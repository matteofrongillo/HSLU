\documentclass{article}

\usepackage{Paper}
\renewcommand\theadfont{\bfseries}
\setcellgapes{2pt}

\pdftitle{Team1_LabReport_UT}

% === TEXT ===
\begin{document}
\thispagestyle{empty}
\hypersetup{citecolor=black}

\begin{minipage}{0.7\textwidth}
    \vspace*{-.8cm} \hspace*{-0.3cm}
    \includegraphics[width=.5\textwidth]{media/hslu-logo.png}
\end{minipage}

\vspace*{1.5cm}

\textbf{\huge Lab report}\\[.75cm]
\begin{center}
    \textbf{\huge Analysis and identification of Plastics}
    
    \textbf{\huge by Soxhlet extraction and ATR-IR}\\[1cm]
    
    \includegraphics[width=.65\textwidth]{media/front_report.png}\\
\end{center}

\vfill

\setlength{\intextsep}{0pt}%
\begin{wrapfigure}{r}{\textwidth}
    \textbf{\Large Principle of Sustainable Environmental Systems\\[.5cm]
    {\large
    Dr. Rebecca Ravotti\\
    Dr. Davide Toniolo\\
    Dr. Macarena San Martín Ruiz\\
    Lecturers}}
    \vspace{-2.1cm}
\end{wrapfigure}

\phantom{}\\[-1cm]

\begin{flushright}
        \large
        \textbf{Team 1}\\
        Bürli Norman\\
        Frongillo Matteo\\
        Murali Arjun\\
        Neukom Yannik\\
        Rossi Anthony
\end{flushright}
\wrapfill
\newpage
\tableofcontents
\pagebreak

\section{Introduction}
\label{sec:introduction}
Plastic is a material found in every environment, and its use is
indispensable in everyday life, present in consumer products and
industrial applications. However, plastics often contain additives
that modify their properties, such as their appearance or performance.
Analyzing and understanding the composition of plastics is essential
for mitigating pollution and protecting nature so that their
environmental hazards can be effectively minimized.

This experiment focuses on the extraction of admixtures from plastics
and on the spectroscopic analysis of polymers in order to determine
their chemical composition and the functional groups of the
molecules. By combining these two techniques, it is possible to
evaluate and study the environmental impact of polymers and their
additives.

\subsection{Soxhlet extraction}
\begin{wrapfigure}{l}{.4\textwidth}
    \includegraphics[width=.4\textwidth]{media/soxhlet_extractor.png}
    \caption{Soxhlet extractor components}
\end{wrapfigure}

\phantom{}

Soxhlet extraction is a chemical separation technique based on
extraction cycles. The technique consists of the separation of
soluble compounds from solid materials and relies on the differences
in solubility between the analyte and the solid matrix; that is the
portion of the sample that remains insoluble in the chosen solvent.

A Soxhlet extractor is composed of a round-bottom flask in which the
solvent is boiled (1,2) heated up by an isomantle (8), a thimble (3)
needed for holding the solid sample that contains the analyte to be
extracted, which in turn is contained in the Soxhlet body (4).
A siphon (6) and a side arm (7) connect the flask to the body,
enabling the solvent circulation throughout the system.

A Soxhlet cycle is made up of solvent boiling, vapor condensation,
percolation through the sample, and siphoning of the extract back
into the boiling flask. A condenser with running water (5) is used to
liquefy the rising solvent vapors and drip back onto the sample.
\wrapfill

\vspace*{-1.5cm}

\subsection{Attenuated Total Reflectance-Fourier Transform Infrared Spectroscopy (ATR-IR)}
Fourier transform infrared spectroscopy (FTIR) consists of a broadband infrared source.
To measure all frequencies simultaneously, the beam must first pass through an
interferometer to encode the light \parencite{FTIR13}.
Afterwards, the infrared light is sent to an ATR extension to probe the sample.

Attenuated total reflection extension (ATR) consisting of a crystal with a high refractive
index \parencite{ATR}. The extension holds the sample in place with
maximum contact to the crystal.\\[1.5ex]

\begin{figure}[ht!]
    \centering
    \includegraphics[width=.45\textwidth]{media/FTIR_crystal.png}
    \caption{Graphical representation of a single bounce ATR \parencite{Specac}}
\end{figure}
\newpage

The infrared beam from the FTIR source arrives and enters the crystal at a specific angel,
is internally reflected many times, and an evanescent wave forms.
The evanescent wave will penetrate out of the crystal and into the sample at a depth of
around 0.5 - 2 micrometers \parencite{ATR}. Chemical bonds from the sample are excited at
specific frequencies and absorb energy.

Finally, a detector collects the returning beam, and a Fourier Transform is performed to
calculate the final absorbance graph, visually showing at which frequencies the material
has been absorbing energy \parencite{FTIR4}. 

\section{Materials and Methods}
\subsection{Material for Soxhlet extraction method}
\begin{minipage}[t]{0.43\textwidth}
    \begin{itemize}
        \item Soxhlet extractor body
        \item Glass reflux condenser
        \item Round-bottom flask (RBF)
        \item Hot plate with heating controller
    \end{itemize}
\end{minipage}
\hfill
\begin{minipage}[t]{0.57\textwidth}
    \begin{itemize}
        \item Support and clamps
        \item Cooling water supply and tubing (secure with clamps)
        \item Polymer sample impregnated with rhodamine B (record mass)
        \item Solvents
    \end{itemize}
\end{minipage}

\subsubsection{Rhodamine B}
Rhodamine B is an organic molecule that exists as the chloride salt,
and it is notable for its intense fluorescence and use in tracing and
dye applications \parencite{NCBI2025_RhodamineB}. It is widely used
in fluorescence microscopy and as a tracer in hydrodynamic studies
because of its bright emission and sensitivity to the environment.

\vspace*{.5cm}
\begin{figure}[ht!]
    \centering
    \includegraphics[width=.45\textwidth]{media/rhodamine.png}
    \caption{Rhodamine B}
\end{figure}

\subsection{Material for ATR-IR}
\begin{itemize}
    \item Plastic samples
    \item Agilent Technologies Cary 630 FTIR Spectrometer
    \item Isopropyl alcohol
\end{itemize}

\subsection{Procedure}
The laboratory work was divided into two main phases: determining the most suitable
solvent for the extraction of rhodamine B from a provided polystyrene-based plastic
sample (matrix) using a Soxhlet extractor, followed by the identification of plastics
and their functional groups in samples collected along the street between the HSLU T\&A
building and the Lucerne Lake.

\subsubsection{Soxhlet extraction}
In this experiment, additives were extracted from a polystyrene-based plastic sample
containing rhodamine B using a Soxhlet extractor, then further identified and
characterized. Before the extraction, a small piece of the polymer was tested with
different solvents to select one that dissolved the dye but did not dissolve the plastic
matrix. Once a suitable solvent had been identified, the required volume was measured and
transferred into a clean round-bottom flask (RBF), filling half of its volume. The mass
of the polymer sample was recorded, and the sample was placed into the extraction thimble,
which was then positioned inside the Soxhlet body. The Soxhlet extraction was then carried
out using a standard assembly consisting of a round-bottom flask (RBF), a Soxhlet body,
and a water-cooled reflux condenser. The RBF, filled with the selected solvent, served as
the boiling reservoir, while the polymer sample was placed in an extraction thimble
positioned within the Soxhlet chamber. The system was heated to establish a gentle,
continuous reflux. Under these conditions, solvent vapor is condensed in the condenser
and percolates through the sample in repeated cycles. Each siphon event returns the
enriched solvent to the RBF, enabling continuous extraction as the dye progressively
accumulated in the boiling flask.

\subsubsection{ATR-FTIR}
The composition of two different pieces of plastic waste was identified using the Agilent
Technologies Cary 630 FTIR Spectrometer. Before the samples were measured, the crystal was
thoroughly cleaned with isopropyl alcohol. This was followed by a background measurement to
calibrate the baseline. The chosen sample was then placed under the spectrometer, and
a measurement was recorded. Using the built-in library of reference materials, the most
likely match was chosen and plotted next to the measured sample. In the results given by
the spectrometer, the most likely match appears as a blue line, while the measured sample
appears as a red line. As the spectrometer can only measure one side of the sample, if
necessary, the sample was flipped, and the other side was measured due to possible
differences in materials. Finally, before proceeding to the next measurements, the crystal
was always cleaned using isopropyl alcohol. 

\vspace*{.5cm}
\begin{figure}[ht!]
    \begin{minipage}[t]{0.48\textwidth}
        \centering
        \includegraphics[width=.8\textwidth]{media/ftir_twix.png}
        \caption{Measurement procedure for the sample ``Twix inside''}
    \end{minipage}
    \hfill
    \begin{minipage}[t]{0.48\textwidth}
        \centering
        \includegraphics[width=.8\textwidth]{media/ftir_twix2.png}
        \caption{Measurement procedure for the sample ``Twix outside''}
    \end{minipage}
\end{figure}

\vspace*{.5cm}
Out of the 17 trash samples collected, the Twix wrapper and the white foam were chosen as
they were found next to the shoreline of the lake, where their potential environmental
impact is greatest. The location of these samples can be seen in \autoref{pic:locations}
under the points 12 and 13, next to the Seerosenplatz.

\newpage
\section{Results}
\label{sub:results}
The following section reports the results obtained from Soxhlet extraction and ATR-FTIR
analysis.

\subsection{Selection of the Soxhlet solvent}
In order to determine the best solvent for the Soxhlet extraction of the rhodamine B
from the polystyrene-based samples, four candidate solvents were tested and categorized
according to their properties and their effectiveness on the matrix and the dye:

\vspace*{.5cm}
\begin{table*}[ht!]
    \caption{Candidate solvents properties and effectiveness}
    \renewcommand{\arraystretch}{2}
    \begin{tabular}{c c c c c c}
    \toprule
    \textbf{Solvent} & \textbf{Solubility of matrix} & \makecell{\textbf{Solubility}\\\textbf{of dye}} &\makecell{\textbf{Boiling point}\\\textbf{[$^\circ$C]}} &\makecell{\textbf{Toxicity}} &\makecell{\textbf{Polarity}} \\
    \midrule
    Hexane & Insoluble & \makecell{No\\extraction} & 69 & \makecell{High:\\neurotoxic} & \makecell{Non-polar} \\[1.5ex]
    Diethyl ether & Slight swelling & \makecell{No\\extraction} & 35 & \makecell{High:\\peroxide-forming} & \makecell{Low polarity} \\[1.5ex]
    Acetone & Swelling & \makecell{No\\extraction} & 56 & \makecell{Moderate:\\irritant} & \makecell{Moderately\\polar} \\[1.5ex]
    Ethanol & Insoluble & \makecell{Strong\\extraction} & 78 & \makecell{Low\\toxicity} & \makecell{Polar} \\
    \bottomrule
    \end{tabular}
\end{table*}

\pph{Scales used}
Matrix solubility: Insoluble -- Slight swelling -- Swelling -- Partially soluble -- Soluble\\
Dye solubility: No extraction -- Weak -- Moderate -- Strong -- Complete extraction

\vspace*{.5cm}
\begin{figure}[ht!]
    \centering
    \includegraphics[width=.6\textwidth]{media/solvents.png}
    \caption{Result of the solvents effectiveness experiment}
\end{figure}

\subsection{Soxhlet extraction}
Throughout the Soxhlet extraction, a progressive change in the appearance of both the
solvent and the matrix was observed. At the beginning of the process, the ethanol in the
round bottom flask looked completely colorless, and the polystyrene samples were brilliant
pink due to the presence of rhodamine B. With the progression in the extraction cycles,
the ethanol gradually stained due to the extraction of the additive in the sample,
turning the solvent pinker. Simultaneously, the polymer-based sample reduced its color
intensity every siphoning cycle. The experiment finished before the first cycle due to
time constraints, nonetheless, assuming between 20 and 60 cycles, with a cycle duration
of 5 to 10 minutes, the expected result is bright-pink colored ethanol due to the
presence of dissolved rhodamine B in the solvent, and a bleached sample compared to its
initial state.

\newpage
\subsection{ATR-FTIR analysis}
The results from the two different samples are listed, each with a picture of the sample,
the measured absorbance graph (in red) compared to the reference graph from the library
(in blue), and a table containing information about the functional groups and possible
compounds. These were determined using the reference table provided in the lab (\autoref{table:functionalgroups}).

\subsubsection{Sample 1: White foam}
Sample 1 is a white foam found in nature, near a road in the village of Horw.

\vspace*{.5cm}
\begin{figure}[ht!]
    \centering
    \includegraphics[width=.4\textwidth]{media/whitefoam.png}
    \caption{Collection of Sample 1}
\end{figure}

\vspace*{.5cm}
\begin{figure}[ht!]
    \centering
    \includegraphics[width=.8\textwidth]{media/whitefoam_analysis.png}
    \caption{Sample 1 ATR-FTIR analysis}
\end{figure}

\vspace*{1cm}
\begin{table*}[ht!]
    \centering
    \caption{Sample 1 results}
    \renewcommand{\arraystretch}{2}
    \begin{tabular}{c c c}
    \toprule
    \textbf{Peak [cm$^{-1}$]} & \textbf{Functional groups} & \textbf{Possible compound}\\
    \midrule
    2900 & CH$_3$ Valenz. & Alkane\\
    2820 & CH$_3$ Valenz. & Methylether\\
    1480 & CH$_3$ and CH$_2$ deform. & \makecell{Hydrocarbons/\\Esters}\\
    710 & \chemfig{C-H} Deform. & \makecell{Monosubstituted\\Benzenes}\\
    \bottomrule
    \end{tabular}
\end{table*}

\newpage
\subsubsection{Sample 2: Inside of a Twix bar wrapper}
Sample 2 is a Twix bar wrapper found beside the railway in the village of Horw.

\vspace*{.5cm}
\begin{figure}[ht!]
    \centering
    \includegraphics[width=.4\textwidth]{media/twix_inside.png}
    \caption{Sample 2 during the analysis}
\end{figure}

\vspace*{.5cm}
\begin{figure}[ht!]
    \centering
    \includegraphics[width=.8\textwidth]{media/twix_inside_analysis.png}
    \caption{Sample 2 ATR-FTIR analysis}
\end{figure}

\vspace*{1cm}
\begin{table*}[ht!]
    \centering
    \caption{Sample 2 results}
    \renewcommand{\arraystretch}{2}
    \begin{tabular}{c c c}
    \toprule
    \textbf{Peak [cm$^{-1}$]} & \textbf{Functional groups} & \textbf{Possible compound}\\
    \midrule
    2910 & CH$_3$ Valenz. & Alkane\\
    1450 & CH$_3$ and CH$_2$ deform. & \makecell{Hydrocarbons/\\Esters}\\
    1380 & CH$_3$ Deform. & Hydrocarbons\\
    \bottomrule
    \end{tabular}
\end{table*}

\newpage
\subsubsection{Sample 3: Outside of a Twix bar wrapper}
Sample 3 is a Twix bar wrapper found beside the railway in the village of Horw.

\vspace*{.5cm}
\begin{figure}[ht!]
    \centering
    \includegraphics[width=.4\textwidth]{media/twix_outside.png}
    \caption{Collection of Sample 3}
\end{figure}

\vspace*{.5cm}
\begin{figure}[ht!]
    \centering
    \includegraphics[width=.8\textwidth]{media/twix_outside_analysis.png}
    \caption{Sample 3 ATR-FTIR analysis}
\end{figure}

\vspace*{1cm}
\begin{table*}[ht!]
    \centering
    \caption{Sample 3 results}
    \renewcommand{\arraystretch}{2}
    \begin{tabular}{c c c}
    \toprule
    \textbf{Peak [cm$^{-1}$]} & \textbf{Functional groups} & \textbf{Possible compound}\\
    \midrule
    3300 (broad) & \chemfig{N-H} Valenz. & Amine/Amide\\
    2940 & CH$_3$ Valenz. & Alkane\\
    1630 & \chemfig{C=O} Valenz. & \makecell{Primary carbon\\acid amice}\\
    1550 & NO$_2$ Valenz. & Nitroalkane\\
    \bottomrule
    \end{tabular}
\end{table*}

\newpage
\section{Discussion}
This section evaluates litter analysis in Horw and examines the results obtained from the
Soxhlet extraction and ATR-FTIR measurements.

\subsection{Litter in Horw}
During the excursion, a survey was conducted in which trash samples were collected across
the region of Horw in Luzern, spanning the area from the HSLU Campus down to Lake Lucerne.

\vspace*{.5cm}
\begin{figure}[ht!]
    \centering
    \includegraphics[width=.85\textwidth]{media/locations.png}
    \caption{Samples locations}
    \label{pic:locations}
\end{figure}

\vspace*{.5cm}
In total, 17 different samples were collected, and their respective images can be found in
the appendix. Out of the 17 trash samples collected, the Twix wrapper and the white foam
were chosen as they were found next to the shoreline of the lake, where their potential
environmental impact is greatest. The location of these samples can be seen in \autoref{pic:locations}
under the points 12 and 13, next to the Seerosenplatz.

This amount of litter collected is relatively little, which can be attributed to a couple of
reasons. These include the availability of bins, regular cleaning and maintenance by the
Horw municipality \parencite{Horw}, and a culture that promotes responsible public
behavior and environmental awareness \parencite{HansmannBinder2020}.

\newpage
\subsubsection{Questions}
The excursion required providing responses to seven questions related to litter
and environmental conditions in Horw:
\begin{itemize}
    \item Why might there be little to no litter in this area?
    \item What does this say about local waste management and environmental awareness?
    \item What might happen to this ecosystem if more people started visiting every day?
    \item What are potential hidden pollutants (microplastics, chemical runoff, etc.) even if no visible litter is present?
    \item How could community behavior or infrastructure (bins, signage, etc.) influence cleanliness?
    \item How could environmental education maintain or improve this standard?
    \item Record observations about waste bins, signage, or evidence of maintenance and report findings. How many waste bins did you see from HSLU until the lake?
\end{itemize}

The following points address the seven questions collectively.

\subsubsection{Availability of bins}
In total, around 50 private bins and 15 public bins were seen on the route from HSLU to
the lake. This means that people looking for a place to dispose of trash do not have to
walk far to find a bin. Additionally, these bins were never overflowing, indicating that
the Horw municipality empties them regularly. These factors encourage people to
correctly dispose of their litter instead of throwing it onto the ground.

\subsubsection{Regual cleaning and maintenance}
The fact that none of the bins were overflowing, and that regular road cleaning is carried
out, indicates that the Horw municipality takes responsibility for local waste management
and demonstrates high environmental awareness. Additionally, the municipality of Horw
states that they are responsible for duties including: the operation and maintenance of
streets, paths and public squares; the upkeep of green spaces, public playgrounds, and
bathing/rest areas; the cleaning and maintenance of waste-collection points; and the daily
emptying of public litter bins \parencite{Horw}.

However, an increase in the number of daily visitors would lead to more littering and
greater pollution, if strict waste disposal and education measures are not maintained.
This can increase the workload on the municipality and have negative impacts on the
ecosystem. For instance, litter along the lake shores can degrade the habitat quality and
enter the food chain of the local wildlife.

\subsubsection{Public culture}
In general, the Swiss culture encourages proper disposal of litter. A study in 2020
surveying 1206 participants across Switzerland found that Swiss residents generally have
high environmental awareness and hold themselves responsible for environmental
consequences. Additionally, some of the most frequently performed behaviors to combat
environmental pollution consisted of avoiding littering and correctly sorting waste for
recycling \parencite{su12208547}.

This could be further improved by introducing environmental education including, but not
limited to, workshops, school programs, and local awareness campaigns to emphasize the
importance of proper waste disposal and appreciation of the lake ecosystem. Additionally,
properly placed bins, clear signage, and community involvement in maintaining a clean
neighborhood can help promote cleanliness.

\subsubsection{Invisible trash}
Despite the absence of visible trash, many invisible elements still contribute to
dangerous pollution and need to be taken into consideration.
One of the major concerns is microplastics, which comprise plastic particles smaller
than five millimetres and often barely visible. A study conducted by the Swiss
Federal Institute of Aquatic Science and Technology, Eawag found that freshwater snails
that ingested nanoplastics with their food were unable to reproduce \parencite{D2EN00101B}.
This is concerning due to the presence of the lake ecosystem nearby. Chemical runoff,
detergents, fertilizers, or sediments along the lake are other concerns.
These pollutants can accumulate over time in the ecosystem and in wildlife. 

\newpage
\subsection{Soxhlet extraction: results interpretation}
After the experiment for the decision of the solvent, ethanol was chosen due to its
capability of removing rhodamine B effectively from the polystyrene sample without
dissolving or damaging the plastic matrix. Ethanol is a polar alcohol molecule, leading
to the conclusion that the dye is also partly polar. The polystyrene, which is a non-polar
polymer, was not dissolved in the solvent. In contrast, diethyl ether and acetone caused
the polymer to swell or soften, indicating partial solubility, but failed to extract the
dye. Hexane, a strongly non-polar molecule, dissolved neither the matrix nor the dye. 

With the change in color of the solvent and the matrix after the first extraction cycle
in the Soxhlet experiment, the efficiency of the solvent with respect to the dye was
demonstrated. Due to the low viscosity of the ethanol while boiling, the interaction
with the matrix and the rhodamine B was enhanced, lowering consequently the cycle time. 

\subsubsection{Questions}
\begin{enumerate}
    \item Why does operating at the solvent’s boiling point almost always improve extraction efficiency?
    \item[R:] The boiling temperature enhances mass transfer due to the increased
        solubility of the solute and the decreased viscosity and surface tension of the
        solvent. The solvent interacts more effectively with the matrix because of the
        reduced thickness of the boundary layer. Soxhlet continuous reflux helps to renew
        the solvent and keep it warm, allowing for a faster and more complete extraction \parencite{Dmitrienko2024}.\\

    \item In what ways can Soxhlet extraction reduce fresh solvent consumption compared with repeated batch extractions?
    \item[R:] In Soxhlet extraction, the same portion of the solvent continuously boils
        and condenses, removing the necessity of replacing the solvent after each
        extraction phase, as required in repeated batch extraction. Moreover, the solvent
        in the thimble is automatically and continuously replenished, ensuring that it
        never becomes fully saturated.\\

    \item Which other characteristics of rhodamine can help us with its identification?
    \item[R:] Rhodamine B can be identified by its pink-purple color and by its strong
        fluorescence, and is easily excited under UV light.\\
        
    \item What makes a solvent good for extraction?
    \item[R:] The polarity of the solute plays a key role in selecting the most suitable
        solvent, as the solvent’s polarity should match the solute’s (“like dissolves like”).
        In organic extractions, the nature of the phytochemical constituents
        (bioactive compounds naturally present in plants) is also important. Moreover,
        the thermal stability and boiling point of the solvent are essential properties
        to consider \parencite{Zhang2018}.
\end{enumerate}

\subsection{ATR-FTIR analysis: results interpretation}
This section provides an interpretation of the ATR-FTIR results, outlining the methodology
used, the material identification process, and the environmental implications of the findings.

\subsubsection{Questions}
For the ATR-FTIR analyses, the following questions were asked:
\begin{enumerate}
    \item What is the purpose of using ATR-FTIR in environmental analysis?
    \item What does the absorbance spectrum represent?
    \item Compare your measured spectrum with reference spectra from the library what materials were identified?
    \item Discuss the environmental implications of finding plastics at the lake shore.
    \item How could the presence of these materials impact ecosystem health and aquatic life?
\end{enumerate}

\subsubsection{Methodology}
To make a risk assessment of an environmental sample, the material must first be
determined. Then further analysis can proceed. ATR-FTIR is an excellent choice for
two main reasons: firstly, it is portable for use in the field, and secondly, it is
relatively inexpensive and reliable. 

As discussed in \autoref{sec:introduction}, the ATR-FTIR uses a system of infrared
beams, a crystal, and a detector to generate an absorbance graph of the sample tested.
The graph shows which frequencies of infrared radiation were absorbed by the material and
by what amount. It can have broad bands or pointy peaks. This pattern is used as a
fingerprint to identify the material in question.

\subsubsection{Identification}
After the measurement of each sample, the computer automatically calculates the best match
for the given absorbance graph. However, it could be wrong, so it is always a good habit
to check and verify if the peaks match the sample accurately. The data presented in
\autoref{sub:results} are now evaluated to figure out what the best-fitting material is
for each of the two samples:

\pph{Sample 1: White foam}
Low-density Polyethylene, this is very plausible, due to it matching the measured graph
nearly perfectly. Each frequency lines up exactly, and the magnitude of the peak is spot
on. This result can be trusted.

\pph{Sample 2: Inside of a Twix bar wrapper}
Polyethylene, this is also a very good match: all the frequency peaks, and
magnitudes match. This result can be trusted.

\pph{Sample 3: Outside of a Twix bar wrapper}
Nylon 6 Poly(caprolactam) Pellets, here it is harder to justify a 100\% match, the
frequencies match almost perfectly, but in the high-frequency region the magnitudes are
very different. In the low-frequency region, some minor differences can be seen. All in
all, this should still be a match, but caution is still advised.

\subsubsection{Impact on the environment}
Polyethylene is very stable and can persist in the environment for a long time. This means
that it will spread from its initial deposition to many other areas. This is accelerated
by the fragmentation of the material into smaller pieces \parencite{YAO2022113933}.  

These smaller pieces, called microplastics, can stay in the environment for many years.
They lead to a range of dangerous side effects, many still unknown. One dangerous effect
is the bioaccumulation \parencite{nature}, where microplastics are ingested by organisms,
build up in their tissues, and are then transported up the food chain. For a human, this
could lead to an increased risk in cardiovascular disease \parencite{nature}. In aquatic
life, such as Lake Lucerne and its population of fish and plants, this poses a great
risk for their reproduction and growth over longers periods of time.

Nylon microplastics can also have a devastating impact on marine life. The feeding of
primary and secondary producers is potentially hindered due to these microplastics.
If these populations decrease, a ripple effect could destabilize the entire food chain,
leading to a reduced total biomass and diminished overall ecosystem health \parencite{Cole2019}. 

\newpage
\section{Conclusions}
In conclusion, the combined use of Soxhlet extraction and ATR-FTIR spectroscopy was an
effective approach for studying the chemical composition of plastics especially in the
context of waste management. The ATR-FTIR experiment identified the presence of compounds
such as polyethylene, low density polyethylene and Nylon 6 Poly(caprolactam) Pellets in
the trash samples, which can be harmful to the environment. Polyethylene and nylon
persist in the environment for a very long time and fragment into microplastics,
harming the local ecosystem.  

The Soxhlet extraction successfully showed its effectiveness in dissolving certain
chemicals contained within materials through the usage of rhodamine B dye.
Nevertheless, tthe collected samples were not further analyzed using Soxhlet
extraction, as ATR-FTIR already provided sufficient information.

Throughout the route from HSLU to the lakeshore, relatively little litter was found.
This can be assosiacted with the fact that multiple public and private bins were found,
none of which were overflowing. This also indicates that the Horw municipality takes
responsibility for local waste management and has a high environmental awareness.
Additionally, the Swiss culture in general promotes environmental awareness and holding
one another accountable for environmental consequences.

\newpage
\appendix
\section{Appendix}
\subsection{Inventory of samples collected}

\begin{figure}[h!]
\centering
\begin{minipage}{0.32\textwidth}
    \centering
    \caption*{1. Wrapper 1}
    \includegraphics[width=\textwidth]{media/s1.png}
    E8°18'19.0''\; N47°00'39.8''\; (Altitude: 498 m)
\end{minipage}
\hfill
\begin{minipage}{0.32\textwidth}
    \centering
    \caption*{2. Wrapper 2}
    \includegraphics[width=\textwidth]{media/s2.png}
    E8°18'19.1''\; N47°00'39.9''\; (Altitude: 480 m)
\end{minipage}
\hfill
\begin{minipage}{0.32\textwidth}
    \centering
    \caption*{3. Cigarette}
    \includegraphics[width=\textwidth]{media/s3.png}
    E8°18'19.6''\; N47°00'38.9''\; (Altitude: 578 m)
\end{minipage}
\end{figure}

\vspace{.75cm}
\begin{figure}[h!]
\centering
\begin{minipage}{0.32\textwidth}
    \centering
    \caption*{4. White plastic piece}
    \includegraphics[width=\textwidth]{media/s4.png}
    E8°18'20.7''\; N47°00'38.0''\; (Altitude: 491 m)
\end{minipage}
\hfill
\begin{minipage}{0.32\textwidth}
    \centering
    \caption*{5. AA battery}
    \includegraphics[width=\textwidth]{media/s5.png}
    E8°18'19.7''\; N47°00'38.2''\; (Altitude: 525 m)
\end{minipage}
\hfill
\begin{minipage}{0.32\textwidth}
    \centering
    \caption*{6. Zip tie}
    \includegraphics[width=\textwidth]{media/s6.png}
    E8°18'22.3''\; N47°00'35.9''\; (Altitude: 476 m)
\end{minipage}
\end{figure}

\newpage
\begin{figure}[h!]
\centering
\begin{minipage}{0.32\textwidth}
    \centering
    \caption*{7. Liquid proteine wrapper}
    \includegraphics[width=\textwidth]{media/s7.png}
    E8°18'21.0''\; N47°00'32.1''\; (Altitude: 477 m)
\end{minipage}
\hfill
\begin{minipage}{0.32\textwidth}
    \centering
    \caption*{8. Black plastic piece}
    \includegraphics[width=\textwidth]{media/s8.png}
    E8°18'20.9''\; N47°00'29.1''\; (Altitude: 475 m)
\end{minipage}
\hfill
\begin{minipage}{0.32\textwidth}
    \centering
    \caption*{9. Metal bottle cap}
    \includegraphics[width=\textwidth]{media/s9.png}
    E8°18'20.2''\; N47°00'24.3''\; (Altitude: 481 m)
\end{minipage}
\end{figure}

\vspace{.75cm}
\begin{figure}[h!]
\centering
\begin{minipage}{0.32\textwidth}
    \centering
    \caption*{10. Aluminum can 1}
    \includegraphics[width=\textwidth]{media/s10.png}
    E8°18'19.9''\; N47°00'23.0''\; (Altitude: 477 m)
\end{minipage}
\hfill
\begin{minipage}{0.32\textwidth}
    \centering
    \caption*{11. Wrapper 3}
    \includegraphics[width=\textwidth]{media/s11.png}
    E8°18'19.8''\; N47°00'24.4''\; (Altitude: 478 m)
\end{minipage}
\hfill
\begin{minipage}{0.32\textwidth}
    \centering
    \caption*{12. Twix wrapper}
    \includegraphics[width=\textwidth]{media/s12.png}
    E8°18'19.8''\; N47°00'24.4''\; (Altitude: 478 m)
\end{minipage}
\end{figure}

\newpage
\begin{figure}[h!]
\centering
\begin{minipage}{0.32\textwidth}
    \centering
    \caption*{13. White foam}
    \includegraphics[width=\textwidth]{media/s13.png}
    E8°18'19.5''\; N47°00'24.2''\; (Altitude: 483 m)
\end{minipage}
\hfill
\begin{minipage}{0.32\textwidth}
    \centering
    \caption*{14. Wrapper 4}
    \includegraphics[width=\textwidth]{media/s14.png}
    E8°18'19.3''\; N47°00'23.5''\; (Altitude: 449 m)
\end{minipage}
\hfill
\begin{minipage}{0.32\textwidth}
    \centering
    \caption*{15. Evliya wrapper}
    \includegraphics[width=\textwidth]{media/s15.png}
    E8°18'23.6''\; N47°00'42.1''\; (Altitude: 487 m)
\end{minipage}
\end{figure}

\vspace{.75cm}
\begin{figure}[h!]
\centering
\begin{minipage}{0.32\textwidth}
    \centering
    \caption*{16. Aluminum can 2}
    \includegraphics[width=\textwidth]{media/s16.png}
    E8°18'23.6''\; N47°00'42.1''\; (Altitude: 487 m)
\end{minipage}
\hfill
\begin{minipage}{0.32\textwidth}
    \centering
    \caption*{17. Plastic bottle cap}
    \includegraphics[width=\textwidth]{media/s17.png}
    E8°18'21.8''\; N47°00'46.0''\; (Altitude: 491 m)
\end{minipage}
\hfill
\begin{minipage}{0.32\textwidth}
    \phantom{}
\end{minipage}
\end{figure}

\newpage
\subsection{ATR-FTIR analysis}

\renewcommand{\arraystretch}{1.8}
\begin{longtable}{ccl}
    \caption{Inftrared absorption bands for Functional groups identification} \\
    \toprule
    \textbf{Wavenumber (cm\(^{-1}\))} & \textbf{Vibration} & \textbf{Compounds} \\
    \midrule
    \endfirsthead
    \toprule
    \textbf{Wavenumber (cm\(^{-1}\))} & \textbf{Vibration} & \textbf{Compounds} \\
    \midrule
    \endhead
    3700--3600    & O--H stretching & Alcohols, phenols, acids \\
    3500--3300    & O--H stretching (broad) & Alcohols, phenols, acids \\
    3550--3350    & N--H stretching (unassociated) & Primary/secondary amines, amides \\
    3500--3100    & N--H stretching (associated) & Primary/secondary amines, amides \\
    3300--3270    & $\equiv$C--H stretching & Monosubstituted alkynes \\
    3350--3150    & NH$_3^+$ stretching (broad) & Aminohydrochlorides \\
    3300--2500    & O--H stretching (very broad) & Carboxylic acids \\
    3100--3000    & =C--H stretching & Aromatics, alkenes \\
    3000--2800    & C--H stretching & Alkanes, cycloalkanes \\
    2962, 2872    & CH$_3$ stretching & Alkanes \\
    2926, 2853    & CH$_2$ stretching & Alkanes \\
    2820          & CH$_3$ stretching & Methyl ethers \\
    2300--2100    & C$\equiv$X stretching (X = C, N, O) & Alkynes, nitriles \\
    2190--2100    & C$\equiv$C stretching & 1,2-disubstituted alkynes \\
    2245--2220    & C$\equiv$N stretching & Nitriles \\
    2140--1970    & C$\equiv$C stretching & Monosubstituted alkynes \\
    1900--1660    & C=O stretching & Carbonyl compounds \\
    1850--1800    & C=O stretching & Acid halides \\
    1840--1780    & C=O stretching & Acid anhydrides (2 bands) \\
    1780--1650    & C=O stretching & Saturated carboxylic acids \\
    1760--1700    & C=O stretching & Saturated esters \\
    1740--1710    & C=O stretching & \makecell[l]{Aldehydes/ketones; $\alpha,\beta$-unsaturated\\and aromatic esters} \\
    1745          & C=O stretching & Cyclopentanone \\
    1715          & C=O stretching & Cyclohexanone \\
    1715--1680    & C=O stretching & $\alpha,\beta$-unsaturated and aromatic aldehydes \\
    1690--1660    & C=O stretching & $\alpha,\beta$-unsaturated and aromatic ketones \\
    1680--1630    & C=O stretching & Primary amides (Amide I) \\
    1660--1600    & C=C stretching & Aromatics, alkenes \\
    1650--1620    & NH$_2$ deformation & Primary amides (Amide II) \\
    1650--1580 & N--H deformation & Primary and secondary amines \\
    1630--1615 & H--O--H deformation & Crystalline water \\
    1630--1590 & Ring vibration & Aromatics \\
    1560       & NO$_2$ stretching & Nitroalkanes \\
    1520       & NO$_2$ stretching & Aromatic nitro compounds \\
    1518       & --- & Aromatic nitro compounds \\
    1500--1480 & Ring vibration & Aromatics \\
    1480--1430 & CH$_3$, CH$_2$ deformation & Hydrocarbons, esters \\
    1420--1340 & O--H deformation & Alcohols, phenols, carboxylic acids \\
    1390--1370 & CH$_3$ deformation & Hydrocarbons \\
    1360--1030 & C--N stretching & Amides, amines \\
    1350--1420 & NO$_2$ stretching & Aliphatic and aromatic nitro compounds \\
    1290--1050 & C--O stretching & Ethers, alcohols \\
    1250--1180 & C--O stretching & Saturated esters \\
    1200--600  & C--H deformation / ring & \makecell[l]{Alkanes, cycloalkanes, alkenes,\\substituted aromatics} \\
    970--960   & C--H deformation & 1,2-disubstituted alkenes (trans) \\
    965--885   & --- & Monosubstituted alkenes \\
    915--905   & C--H deformation & 1,3-disubstituted benzenes \\
    830--750   & --- & --- \\
    815--750   & C--H deformation & 1,2-disubstituted aromatics \\
    855--825   & C--H deformation & 1,4-disubstituted aromatics \\
    770--735   & C--H deformation & 1,2-disubstituted benzenes \\
    740--680   & C--H deformation & Monosubstituted benzenes \\
    710--690   & --- & Alkanes with \(>\)4 CH$_2$ groups \\
    670--650   & C--H deformation & 1,2-disubstituted alkenes (cis) \\
    690--610   & C--H deformation & Benzene \\
    620--490   & C--I stretching & Aliphatics \\
    700--500   & C--Br stretching & Aliphatics \\
    \bottomrule
    \label{table:functionalgroups}
\end{longtable}


\newpage
\listoffigures

\listoftables

\printbibliography

\section*{Declarations on the use of AI tools}
\begin{itemize}
    \item ``\textit{ChatGPT 5.1}'' was used to enhance vocabulary.\\
        {\color{darkgray!95}{\textit{All original sentences originate from our own ideas and were refined with the support of this tool.}}}\\
        \url{https://chatgpt.com/}
    \item ``DeepL'' was used as a spell-checker.\\
        \url{https://www.deepl.com}
    \item ``Google Gemini'' was used as a fact-checker.\\
        \url{https://gemini.google.com/app}
\end{itemize}

\end{document}