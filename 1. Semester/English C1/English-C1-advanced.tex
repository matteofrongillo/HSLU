\documentclass{article}

\usepackage[fleqn]{amsmath}
\usepackage{amssymb}
\usepackage{hyperref}
\usepackage{url}
\usepackage{graphicx}
\usepackage{geometry}
\usepackage{babel}
\usepackage{enumitem}
\usepackage{parskip}
\usepackage{chemfig}
\usepackage{pdfpages}
\usepackage{xcolor}
\usepackage{tikz}
\usepackage{fancybox}
\usepackage{makecell}
\usepackage{pgfplots}
\usepackage{soul}
\usepackage{ulem}
\usepackage{wrapfig}
\usepackage{subcaption}
\usepackage[T1]{fontenc}
\usepackage{esvect}
\usetikzlibrary{arrows}
\usetikzlibrary{decorations.pathreplacing}
\pgfplotsset{compat=1.17}

\geometry{
    a4paper,
    total={170mm, 257mm},
    left=20mm,
    top=20mm
}

\hypersetup{
    colorlinks=true,
    linkcolor=black,
    urlcolor=blue,
    pdftitle={System engineering in EES}
}

\newcommand{\figbox}[1]{ 
    \begin{figure*}[ht!]        
        \begin{center}            
            \fbox{#1}        
        \end{center}    
    \end{figure*}
}

\newcommand{\wrapfill}{
    \par
    \ifnum \value{WF@wrappedlines} > 0
        \addtocounter{WF@wrappedlines}{-1}%
        \null\vspace{
            \arabic{WF@wrappedlines}
            \baselineskip
        }
        \WFclear
    \fi
    \phantom{}
}

\newcommand{\cfig}[1]{%
  \begin{figure*}[ht!]%
    \centering%
    #1%
  \end{figure*}%
}

\newcommand{\difference}{\,\backslash\,}
\newcommand{\rem}{\underline{Remark}: }
\newcommand{\nots}{\underline{Notation}: }
\newcommand{\prf}{\underline{Proof}: }
\newcommand{\exs}{\underline{Example}: }
\newcommand{\defs}{\underline{Definition}: }
\newcommand{\wrn}{\underline{Warning}: }
\newcommand{\sht}{\ |\ }

% === TEXT ===
\title{\textbf{English C1 Advanced\\ HSLU, Semester 1}}
\author{Matteo Frongillo}
\date{\today}

\begin{document}

\maketitle
\tableofcontents
\pagebreak

\section{Course overview}
\subsection{Exam overview}
\begin{itemize}
    \item Use of English and Reading: \textbf{MEP (30\%)};
    \item Writing: \textbf{MEP (30 \%)};
    \item Listening: \textbf{Semester performance (20\%)};
    \item Speaking: \textbf{Semester performance (20\%)}.
\end{itemize}


\newpage
\section{Past tenses}

\subsection{Simple tenses}
\subsubsection{Past simple}
Past simple is used for:
\begin{itemize}
    \item Finished past event at a specific past point in time. 
\end{itemize}

\subsection{Continuous tenses}
Are focused on ongoing actions at past or present time.

\subsubsection{Past continuous}
Past continuous is used to:
\begin{itemize}
    \item Express something was ongoing at a specific past point;
    \item Focus on an ongoing action in the past that is ``crossed''
        by another past action.
\end{itemize}

\subsection{Perfect tenses}
Most often an action stretching over time that started in the past
and continues either up until now (present perfect) or up until a
past point (past perfect)

\subsubsection{Past perfect simple}
Past perfect simple is used to:
\begin{itemize}
    \item Express something happened before another past time;
    \item Compare two past events.
\end{itemize}

\subsection{Perfect and continuous tenses}
\subsubsection{Past perfect continuous}
Past perfect continuous is used for:
\begin{itemize}
    \item Past actions in progress up until another past point;
    \item Past actions in progress that are interrupted or unfinished.
\end{itemize}

\section{Passive forms}
Passive is used to say what happened to the subject.

Passive sentences are formed with ``to be'' in the appropriate tense
+ the past participle (+ed).

\subsection{Impersonal passive}
\subsubsection{Form}
When we use one of these verbs:

believe, claim, report, say, think, understand, know, consider,
estimate, expect, be rumoured, be reputed, allege;

we have to structure the sentence as follow:\\

\begin{flushleft}
    \begin{tabular}{|m{0cm}>{\centering\arraybackslash}m{1cm}|>{\centering\arraybackslash}m{4.5cm}|>{\centering\arraybackslash}m{6cm}|>{\centering\arraybackslash}m{2cm}|}
        \hline \rule{0pt}{15pt}
        & It + & \textbf{"be"} + (required tense) & \textbf{reporting verb} + (past participle) & that... \\ 
        \hline
    \end{tabular}
\end{flushleft} \phantom{}

\begin{flushleft}
    \begin{tabular}{|>{\centering\arraybackslash}m{1.5cm}|>{\centering\arraybackslash}m{4cm}|>{\centering\arraybackslash}m{5.5cm}|>{\centering\arraybackslash}m{4cm}|}
        \hline
        subject + & \textbf{"be"} + (required tense) & \textbf{reporting verb} + (past participle) & main verb (to-infinitive $\rightarrow$ present or past) \\ 
        \hline
    \end{tabular}
\end{flushleft}

\section{Linking words}

\section{Inversion}
After certain words and phrases the \textbf{word order is inverted}.
This kind of inversion is mainly found in formal speech and writing.

\subsection{Application of the inversion}
\subsubsection{Negation adverbs}
The negative adverbs \textbf{never (before/again), rarely, seldom,
barely/hardly/scarcely...when/before, no sooner...than, nowhere,
little (with a negative meaning)}.

\subsubsection{Negation}


\section{Formal letters}

\section{Reported speech}


\end{document}
