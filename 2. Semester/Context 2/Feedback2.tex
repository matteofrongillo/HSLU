\documentclass{article}

\usepackage{Paper}
\pdftitle{MatteoFrongillo_Feedback2}

\title{\textbf{Reflections presentation 2}\\\textbf{Context 2}}
\author{Matteo Frongillo}
\date{\today}

\begin{document}

\maketitle

\pph{Overall}
This presentation showed a marked improvement over the previous one. By opening with a
direct question and systematically addressing it through a clear chain of argumentation,
I kept the audience engaged from beginning to end. I think the connection between the
slides was well done. I did not receive any negative feedback this time, and this
confirmed my improvements, namely a stronger opening and conclusion and a polished delivery.

\pph{Introduction}
I began with a direct and intriguing question so that the audience could take an interest
in my opition, also setting the stage for the chain of arguments that followed. I think
this approach is very effective, and in fact I noticed that the audience was actually
interested. Explicitly linking the question to the summary made people understand the
content of the presentation without reading the bullet points in the summary slide.

\pph{Body}
I believe the body of my presentation was well structured. I touched on the most critical
points of solar thermal panels, moving from the production mismatch to the storage problem
and finally talking about the economic part, which I think was the most important for the
public since an average person without in-depth knowledge about solar thermal will tend to
rely more on the costs and benefits rather than the engineering part of the problem. I am
sure that the powerpoint structure was really the strong point for understanding the
opinion I presented. Also providing an alternative to solar thermal by introducing the
benefits of using photovoltaics I think reinforced my initial opinion.

Also, what I improved for this presentation is that I properly cited the sources of the
images (photographer and project owner) rather than generic websites.

\pph{Conclusion}
The feedback on the conclusion of the last presentation was to improve the effectiveness
and tie directly to the introduction. In this one, I concluded by taking up the initial
question and responding by summarizing my opinion just expressed, so I think I have
fulfilled the request.

\pph{Body language and delivery}
Following the advice given to me last presentation, during this one I mantained a calm
posture, moving toward the blackboard only when necessary. I also removed the keychain
attached to my jeans so that it could no longer annoy with its noise. Finally, I think I
had a firm tone of voice and spoke fluently.

\end{document}