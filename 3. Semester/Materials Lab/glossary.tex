\DeclareAcronym{Amorphous}{
  short = Amorphous,
  long  = Non-crystalline material with no long-range order.
}

\DeclareAcronym{Crystalline}{
  short = Crystalline,
  long  = Material with atoms arranged in a highly ordered microscopic structure{,} forming a crystal lattice that extends in all directions.
}

\DeclareAcronym{Monocrystalline}{
  short = Monocrystalline,
  long  = Material consisting of a single crystal or a continuous crystal lattice with no grain boundaries.
}

\DeclareAcronym{Polycrystalline}{
  short = Polycrystalline,
  long  = Material composed of many crystallites of varying size and orientation.
}

\DeclareAcronym{Anisotropy}{
  short = Anisotropy,
  long  = Direction-dependent properties of a material {\color{red}(Monocrystalline and polycrystalline with texture)}
}

\DeclareAcronym{Isotropy}{
  short = Isotropy,
  long  = Direction-independent properties of a material {\color{red}(Amorphous)}
}

\DeclareAcronym{Quasiisotropy}{
  short = Quasi-isotropy,
  long  = Approximate isotropy in polycrystalline materials with random grain orientation {\color{red}(Polycrystalline without texture)}
}

\DeclareAcronym{Polymorphism}{
  short = Polymorphism / Allotropy,
  long  = Ability of a material to exist in more than one form or crystal structure.
}

\DeclareAcronym{Homogeneous}{
  short = Homogeneous,
  long  = Uniform composition and properties throughout the material.
}

\DeclareAcronym{Heterogeneous}{
  short = Heterogeneous,
  long  = Non-uniform composition and properties throughout the material.
}

\DeclareAcronym{Alloy}{
  short = Alloy,
  long  = A mixture of two or more elements{,} where at least one element is a metal.
}

\DeclareAcronym{Dislocation}{
  short = Dislocation,
  long  = A linear defect in the crystal structure where there is an irregularity in the arrangement of atoms.
}

\DeclareAcronym{Vacancy}{
  short = Vacancy,
  long  = A point defect in a crystal lattice where an atom is missing from its regular lattice site.
}

\DeclareAcronym{Slip}{
  short = Slip,
  long  = Large displacement of one part of a crystal relative to another part along crystallographic planes and directions.
}

\DeclareAcronym{HCF}{
  short = HCF,
  long = High-cycle fatigue. It occurs when materials are subjected to stresses much lower than their yield strength{,} at a high number of cycles.
}

\DeclareAcronym{LCF}{
  short = LCF,
  long = Low-cycle fatigue. It happens when materials are subjecter to higher stresses{,} typically exceeding the yield strength{,} at a smaller number of cycles.
}

\DeclareAcronym{Poisson}{
  short = Poisson's ratio $\nu$,
  long = The ratio of transverse strain to longitudinal strain in a material under uniaxial loading.
}

\DeclareAcronym{Young's modulus}{
  short = Young's modulus $E$,
  long = The ratio of normal stress to longitudinal strain in the elastic range of a material.
}

\DeclareAcronym{Shear modulus}{
  short = Shear modulus $G$,
  long = The ratio of shear stress to shear strain in the elastic range of a material.
}

\DeclareAcronym{Phase}{
  short = Phase,
  long = A region of material that is chemically and structurally uniform.
}

\DeclareAcronym{Brittle}{
  short = Brittle,
  long = Material that fractures without significant plastic deformation.
}

\DeclareAcronym{Hardening}{
  short = Hardening,
  long = The process of increasing a material's hardness and strength through various methods such as heat treatment or work hardening.
}

\DeclareAcronym{Toughness}{
  short = Toughness,
  long = The ability of a material to absorb energy and plastically deform without fracturing.
}

\DeclareAcronym{Q+T}{
  short = Q+T,
  long = Quenching and tempering. A heat treatment process that involves rapid cooling (quenching) followed by reheating (tempering) to improve mechanical properties.
}

\DeclareAcronym{Isothermal transformation}{
  short = Isothermal transformation,
  long = A phase transformation that occurs at a constant temperature.
}

\DeclareAcronym{CHD}{
  short = CHD,
  long = Case hardening depth. The depth to which a material has been hardened by surface treatment processes.
}

\DeclareAcronym{SHD}{
  short = SHD,
  long = Surface hardening depth. The depth to which the surface of a material has been hardened.
}

\DeclareAcronym{NHD}{
  short = NHD,
  long = Nitriding hardening depth. The depth to which a material has been hardened by nitriding.
}

\DeclareAcronym{Pig iron}{
  short = Pig iron,
  long = High-carbon iron produced in a blast furnace{,} used as a raw material for making steel and cast iron.
}

\DeclareAcronym{Crude steel}{
  short = Crude steel,
  long = Refined steel with $<$ 2\% carbon that has been produced but not yet refined or processed into finished products.
}

\DeclareAcronym{Mild steel}{
  short = Mild steel,
  long = Low-carbon steel with a carbon content of approximately 0.05\% to 0.25\%{,} known for its ductility and weldability.
}

\DeclareAcronym{Stainless steel}{
  short = Stainless steel,
  long = Corrosion-resistant steel alloy containing a minimum of 10.5\% chromium.
}

\DeclareAcronym{Austenite}{
  short = Austenite ($\gamma$-Fe),
  long = Face-centered cubic (FCC) phase of iron{,} stable at high temperatures and soluble up to 2\% carbon.
}

\DeclareAcronym{Ferrite}{
  short = Ferrite ($\alpha$-Fe),
  long = Body-centered cubic (BCC) phase of iron{,} stable at room temperature and low C solubility.
}

\DeclareAcronym{Pearlite}{
  short = Pearlite,
  long = A two-phase lamellar microstructure consisting of alternating layers of ferrite and cementite{,} formed during the slow cooling of austenite.
}

\DeclareAcronym{Martensite}{
  short = Martensite,
  long = A hard{,} brittle phase formed by the rapid quenching of austenite{,} characterized by a body-centered tetragonal (BCT) structure.
}

\DeclareAcronym{Bainite}{
  short = Bainite,
  long = Strong{,} ductile microstructure formed in steels at temperatures between those that form pearlite and martensite{,} consisting of a mixture of ferrite and carbides.
}

\DeclareAcronym{Cementite}{
  short = Cementite (Fe$_3$C),
  long = A hard{,} brittle intermetallic compound of iron and carbon{,} forming part of the microstructure in steels and cast irons.
}

\DeclareAcronym{Carburizing}{
  short = Carburizing,
  long = A heat treatment process that enriches the surface layer of a low-carbon steel with carbon to increase its hardness.
}

\DeclareAcronym{Nitriding}{
  short = Nitriding,
  long = A heat treatment process that introduces nitrogen into the surface of a steel to form hard nitrides{,} enhancing surface hardness and wear resistance.
}

\DeclareAcronym{Tempering}{
  short = Tempering,
  long = A heat treatment process that reduces brittleness and increases toughness in quenched steels by reheating to a temperature below the eutectoid temperature.
}

\DeclareAcronym{Quenching}{
  short = Quenching,
  long = A rapid cooling process used to harden steel by transforming austenite into martensite.
}

\DeclareAcronym{Ladle}{
  short = Ladle,
  long = A large container used to hold and transport molten metal during steelmaking and casting processes.
}

\DeclareAcronym{Carbide}{
  short = Carbide,
  long = A compound composed of carbon and a less electronegative element{,} often forming hard materials used in cutting tools and abrasives.
}

\DeclareAcronym{Brittleness}{
  short = Brittleness,
  long = The tendency of a material to fracture with little to no plastic deformation.
}

\DeclareAcronym{Cottrell atmosphere}{
  short = Cottrell atmosphere,
  long = A cluster of intestitial atmos (e.g. C{,} N) around a dislocation in BCC metals{,} causing dislocation pinning and higher yield stresses.
}

\DeclareAcronym{Ductility}{
  short = Ductility,
  long = The ability of a material to undergo plastic deformation before fracture.
}

\DeclareAcronym{Fatigue}{
  short = Fatigue,
  long = Progressive structural damage caused by repeated or fluctuating stress below the static strength of the material.
}

\DeclareAcronym{Hardness}{
  short = Hardness,
  long = Resistance of a material to localized plastic deformation{,} typically measured by indentation tests.
}

\DeclareAcronym{Impact Toughness}{
  short = Impact Toughness,
  long = The ability of a material to absorb energy under sudden loading before fracture.
}

\DeclareAcronym{Coherent}{
  short = Coherent,
  long = A condition in which the atomic planes of two phases{,} such as a precipitate and its matrix{,} are continuous across their interface{,} resulting in lattice alignment and elastic strain without dislocations.
}

\DeclareAcronym{Annealed}{
  short = Annealed,
  long = A heat treatment process in which a metal is heated and slowly cooled to reduce internal stresses{,} soften the material{,} and improve ductility and machinability.
}