\documentclass{article}

\usepackage[fleqn]{amsmath}
\usepackage{amssymb}
\usepackage{hyperref}
\usepackage{url}
\usepackage{graphicx}
\usepackage{geometry}
\usepackage{babel}
\usepackage{enumitem}
\usepackage{parskip}
\usepackage{chemfig}
\usepackage{pdfpages}
\usepackage{xcolor}
\usepackage{tikz}
\usepackage{fancybox}
\usepackage{makecell}
\usepackage{pgfplots}
\usepackage{soul}
\usepackage{ulem}
\usepackage{wrapfig}
\usepackage{subcaption}
\usepackage[T1]{fontenc}
\usepackage{esvect}
\usetikzlibrary{arrows}
\usetikzlibrary{decorations.pathreplacing}
\pgfplotsset{compat=1.17}

\geometry{
    a4paper,
    total={170mm, 257mm},
    left=20mm,
    top=20mm
}

\hypersetup{
    colorlinks=true,
    linkcolor=black,
    urlcolor=blue,
    pdftitle={Environmental chemistry and biology}
}

\newcommand{\figbox}[1]{ 
    \begin{figure*}[ht!]        
        \begin{center}            
            \fbox{#1}        
        \end{center}    
    \end{figure*}
}

\newcommand{\wrapfill}{
    \par
    \ifnum \value{WF@wrappedlines} > 0
        \addtocounter{WF@wrappedlines}{-1}%
        \null\vspace{
            \arabic{WF@wrappedlines}
            \baselineskip
        }
        \WFclear
    \fi
    \phantom{}
}

\newcommand{\cfig}[1]{%
  \begin{figure*}[ht!]%
    \centering%
    #1%
  \end{figure*}%
}

\newcommand{\difference}{\,\backslash\,}
\newcommand{\rem}{\underline{Remark}: }
\newcommand{\nots}{\underline{Notation}: }
\newcommand{\prf}{\underline{Proof}: }
\newcommand{\exs}{\underline{Example}: }
\newcommand{\defs}{\underline{Definition}: }
\newcommand{\wrn}{\underline{Warning}: }
\newcommand{\sht}{\ |\ }


% === TEXT ===
\title{\textbf{Environmental chemistry and biology \\ HSLU, Semester 1}}
\author{Matteo Frongillo}

\begin{document}

\maketitle
\tableofcontents
\pagebreak

\part*{Preamble (Week 0)}
\section{Agenda}
\subsection{Environmental concepts and definitions}
\begin{itemize}
    \item Definition of Environment and the exchange of materials and energy;
    \item Definition of Environmental chemistry;
    \item Definition of Environmental biology;
    \item Environmental chemistry, biology and related fiels.
\end{itemize}

\subsection{Chemical processes in the environment}
\begin{itemize}
    \item Chemical structure and environmental behavior;
    \item Partitioning of organic substances in the environment;
    \item Chemical transformations under environmental conditions.
\end{itemize}

\section{Learning objectives}
We should be able to:
\begin{itemize}
    \item define the term ``Environment'';
    \item define the term ``Environmental chemistry'';
    \item define the term ``Environmental biology'';
    \item know the physical and chemical properties,
        defining the environemntal behavior of a substance;
    \item apply the concept of partitioning to analyze and understan
        the behavior of an organic substance in the environment, with
        the provided values.
\end{itemize}

\section{Keywords}
{\itshape Environment, Environmental Chemistry, Environmental Biology, Partitioning, Structure (of a molecule),
Chemical Transformation, biotic, abiotic}

\section{Introduction SW 0}
\subsection{Chemical structure and Environmental behavior}
We consider two groups of properties of chemicals:
\subsubsection{Physical properties}
\begin{itemize}
    \item Vapor pressure (mp, bp);
    \item Solubility (H$_2$O, ...);
    \item Acid / Base strength (pK$_a$, pK$_b$);
    \item Partition coefficients (e.g. K$_{OW}$).
\end{itemize}

These properties describe \textbf{Dispersion} in different compartiments $\Rightarrow$ Mobility and Toxicity.

\subsubsection{Chemical properties $\rightarrow$ Reactivity}
\begin{itemize}
    \item Functional groups (-OH, -NH$_2$, ...);
    \item Electronic substituent effects (push/pull of electrons);
    \item Reaction mechanisms.
\end{itemize}

These properties describe \textbf{Transformation} of products $\Rightarrow$ Degradation

\subsection{Partitioning of organic substances in the environment}
It explains how easily a chemical can spread in the environment:
\begin{enumerate}
    \item air $\leftrightarrow$ water:
        \begin{itemize}
            \item vapor pressure (``evaporation rate'');
            \item water solubility.
        \end{itemize}
\end{enumerate}

1. air ↔ water:
• vapor pressure (“evaporation rate”)
• water solubility
2. water ↔ soil:
• adsorption (“sticking to particles”)
• water solubility
3. soil ↔ air:
• adsorption
• vapor pressure
4. all phases ↔ biota:
• fat solubility (“lipophilicity”)

\subsection{Chemical transformations under environmental conditions}
There are two fundamental pathways:

\subsubsection{Abiotic}
In compartments Air, Water, Soil $\rightarrow$ \textbf{Energy source: temperature, light}
\begin{enumerate}
    \item Hydrolysis;
    \item Oxidation;
    \item Reduction;
    \item Photochemical reactions.
\end{enumerate}

\subsubsection{Biotic}
In organisms $\rightarrow$ \textbf{Catalyst: enzymes}
\begin{enumerate}
    \item Oxidation;
    \item Reduction;
    \item Hydrolysis;
    \item Secundary reactions.
\end{enumerate}

\newpage
\part{Week 1}
\section{Agenda}
\begin{enumerate}
    \item \textbf{Overview of the Anthrosphere and Environmental Impact}
        \begin{itemize}
            \item Anthrosphere: definition and impact;
            \item Pollutants and hazardous waste;
            \item Anthrosphere: primary source of pollutants;
            \item Pollutants overview;
            \item Point sources vs. Nonpoint sources;
        \end{itemize}
    \item \textbf{Earth system and pollution dynamics}:
        \begin{itemize}
            \item Earth system and chemicals;
            \item 5-key aspects of pollutants;
        \end{itemize}
    \item \textbf{Important pollutant categories}
        \begin{enumerate}[label=\arabic*.]
            \item Heavy metals;
            \item Greenhouse gases;
            \item Particulate matter (PM);
            \item Persistent organic pollutants (POPs);
            \item Chlorofluorocarbons (CFCs);
            \item Polycyclic aromatic hydrocarbons (VOCs);
            \item Environmentally persistent pharmaceutical pollutants (EPPPs);
            \item Plastics.
        \end{enumerate}
\end{enumerate}

\newpage
\section{Anthrosphere}
Definition:

Impact:

\subsection{Human primary sources}
\cfig{\includegraphics*[width=.8\textwidth]{media/anthrosphere.png}}
\begin{enumerate}
    \item Industrial activities:
    \item Transportation:
    \item Agriculture:
    \item Energy production:
    \item Urban development:
    \item Deforestation and Land Use changes:
    \item Household activities:
    \item Waterwaste and Sewage:
\end{enumerate}

\section{Pollutants and hazardous}
\subsection{Pollutants}
...

\subsection{Hazardous waste}
...

\subsection{Pollution overview}
\cfig{\includegraphics*[width=.8\textwidth]{media/pollution-overview.png}}

\textbf{Air pollution and climate changes}:
...

\textbf{Water pollution and Ecosystem impact}:
...

\textbf{Land degradation}:
...

\textbf{Forests and Farmland}:
...

\section{Point sources vs. Nonpoint sources}
\cfig{\includegraphics*[width=.8\textwidth]{media/Point-and-nonpoint.png}}

...

\section{Earth system and Pollution dynamics}
\subsection{Earth system and chemicals}
...

\subsection{5-key aspects of pollutants}
\cfig{\includegraphics*[width=\textwidth]{media/pollutant aspects.png}}

\begin{enumerate}
    \item Origin: source of the chemical or pollutant;
    \item Transport: distribution of the pollutant;
    \item Reactions: trasformation of the pollutant;
    \item Effects: impact of the pollutant;
    \item Fate: whereabouts of the pollutant.
\end{enumerate}

\section{Important pollutant categories}
\subsection{Heavy metals}
Heavy metals are material with very high densities.

\subsubsection{Sources}
...

\subsubsection{Reactions}
...

\subsubsection{Effect}
...

\subsubsection{Fate}
... 

\subsection{Greenhouse gases (GHGs)}
GHGs are gases which reflects UVs, impeding them to exit from the ozone
and therefore dissipate their heat in it.

\cfig{\includegraphics*[width=.8\textwidth]{media/Treibhauseffekt-Graphik.jpg}}

\subsubsection{Sources}
...

\subsubsection{Reactions}
...

\subsubsection{Effect}
...

\subsubsection{Fate}
... 

\subsection{Particulate matter (PM)}
PM is made by microscopic particles which can depositate inside the
respiratory system of animals and create health problems, such as asthma
or cardiovascular issues.

\subsubsection{Sources}
...

\subsubsection{Reactions}
...

\subsubsection{Effect}
...

\subsubsection{Fate}
... 

\subsection{Persistent organic pollutants (POPs)}
POPs are chemicals which have a very long degradation time. They are called
``forever chemicals''.

\subsubsection{Halogenated organic compounds (HOCs)}
...

\subsubsection{Per- and Polyfluoroalkyl substances (PFASs)}
...

\subsubsection{Both categories}
\begin{itemize}
    \item Reactions:
    \item Effect:
    \item Fate:
\end{itemize}

\subsection{Chlorofluorocarbons (CFCs)}
CFCs are gases ...

\subsubsection{Sources}
...

\subsubsection{Reactions}
...

\subsubsection{Effect}
...

\subsubsection{Fate}
... 

\subsection{Polycyclic aromatic hydrocarbons (PAHs)}
PAHs are gases created by incompleted combustions, such as tobacco smoke
or grilled food.

\subsubsection{Sources}
...

\subsubsection{Reactions}
...

\subsubsection{Effect}
...

\subsubsection{Fate}
... 

\subsection{Volatile organic compounds (VOCs)}
VOCs are gases with a low molecular weight which can evaporate at room temperature.

\subsubsection{Sources}
...

\subsubsection{Reactions}
...

\subsubsection{Effect}
...

\subsubsection{Fate}
... 

\subsection{Environmentally persistent pharmaceutical pollutants (EPPPs)}
EPPPs are pharmaceutical chemicals with a complex structure, which gives to
molecules a big stability and a slow biodegradability.

\subsubsection{Sources}
...

\subsubsection{Reactions}
...

\subsubsection{Effect}
...

\subsubsection{Fate}
... 

\subsection{Plastics}
Plactics are chemicals made by long chains of carbon...

\subsubsection{Sources}
...

\subsubsection{Reactions}
...

\subsubsection{Effect}
...

\subsubsection{Fate}
... 


\end{document}
