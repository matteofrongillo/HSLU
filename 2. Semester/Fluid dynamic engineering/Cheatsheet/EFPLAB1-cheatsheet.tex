\documentclass{article}

\usepackage{Engineering}
\usepackage{multicol}
\usepackage{tcolorbox}
\usepackage{titlesec}
\usepackage{titling}
\usepackage{etoolbox}
\usepackage{tabularx}

% === METADATA ===
\pdftitle{EFPLab1 Cheatsheet}

% === HEADER/FOOTER ===
\usepackage{fancyhdr}
\pagestyle{fancy}
\fancyhf{}
\lhead{Matteo Frongillo}
\chead{\nouppercase{\leftmark}}
\rhead{\thepage}

\renewcommand{\sectionmark}[1]{\markboth{#1}{}}

% === CUSTOM BOX COMMANDS ===
\definecolor{BoxBG}{HTML}{F0F8FF}
\definecolor{BoxBorder}{HTML}{3B83BD}

\newtcolorbox{theorybox}[1]{
  colback=BoxBG,
  colframe=BoxBorder,
  fonttitle=\bfseries,
  left=1mm,right=1mm,top=1mm,bottom=1mm,
  boxrule=0.8pt,
  arc=1.5mm,
  title={\begingroup\intheoryboxtitletrue #1\endgroup\intheoryboxtitlefalse}
}

\newtcolorbox{examplebox}[1]{
  colback=gray!10!white,
  colframe=gray!80!black,
  coltitle=white,
  fonttitle=\bfseries,
  left=1mm,right=1mm,top=1mm,bottom=1mm,
  boxrule=0.8pt,
  arc=1.5mm,
  title={#1}
}

\newtcolorbox{formula}[1]{
  colback=red!10!white,
  colframe=red!90!black!75,
  fonttitle=\bfseries,
  left=1mm,right=1mm,top=1mm,bottom=1mm,
  boxrule=0.8pt,
  arc=1.5mm,
  title={#1}
}

% === SECTION TITLE FORMATTING ===
\newif\ifintheoryboxtitle
\intheoryboxtitlefalse

\titleformat{\section}
  {\ifintheoryboxtitle\color{white}\else\color{BoxBorder}\fi\Large\bfseries}
  {\thesection}{0.5em}{}

\titleformat{\subsection}
  {\ifintheoryboxtitle\color{white}\else\color{BoxBorder}\fi\bfseries}
  {\thesubsection}{0.5em}{}

\titleformat{\subsubsection}
  {\ifintheoryboxtitle\color{white}\else\color{BoxBorder}\fi\small\bfseries}
  {\thesubsubsection}{0.5em}{}

% === DOCUMENT ===
\begin{document}
\begin{multicols}{2}
\setlength{\columnsep}{1pt}

% === CONTENT ===
\section{Fluids as energy carriers}
\subsection{Fluid state variables and properties}

\begin{theorybox}{Formulas}
    \subsubsection{State variables}
    \textbf{Density}
    \begin{equation}
        \rho \triangleq \dfrac{m}{V} \left[\dfrac{kg}{m^3}\right] 
    \end{equation}

    \textbf{Specific volume}
    \begin{equation}
        v \triangleq \dfrac{V}{m} = \dfrac{1}{\rho} \left[\dfrac{m^3}{kg}\right]
    \end{equation}

    \subsubsection{Viscosity}
    \textbf{Kinematic viscosity}
    \begin{equation}
        \nu \triangleq \dfrac{\eta}{\rho} \left[\dfrac{m^2}{s}\right]
    \end{equation}

    \textbf{Dynamic viscosity}
    \begin{equation}
        \eta \triangleq \nu\cdot\rho \left[Pa\cdot s = \dfrac{Ns}{m^2} = \dfrac{kg}{m \cdot s}\right]
    \end{equation}

    \subsubsection{Real and ideal fluid}
    \begin{tabularx}{\linewidth}{@{}X@{\hspace{.1766cm}}X@{}}
        \textbf{Real fluid} & \textbf{Ideal fluid} \\
        variable density $\left(\Delta \rho \neq 0\right)$ & incompressible $\left(\Delta \rho = 0\right)$ \\
        friction $\left(\eta > 0, \nu > 0\right)$ & frictionless $\left(\eta=0, \nu=0\right)$ \\
    \end{tabularx}
    \\
    \subsubsection{Compressibility}
    \textbf{Mach number}
    \begin{equation}
        M \triangleq \dfrac{u}{c}
    \end{equation}
    where:
    \begin{itemize}
        \item $M$ is the Mach number [-]\\
            $M \lesssim 0.3$: incompressible flow
        \item $u$ is the flow velocity [m/s]
        \item $c$ is the speed of sound in the fluid [m/s]
    \end{itemize}
    and:
    \begin{itemize}
        \item $c_{\text{w}}^{20^\circ} = 1484$ m/s
        \item $c_{\text{a}}^{20^\circ} = 343$ m/s
    \end{itemize}
\end{theorybox}

\subsection{Laminar and turbulent flow}
\begin{formula}{Reynolds number}
    \begin{equation}
        Re = \dfrac{v\cdot L}{\nu} = \dfrac{\rho\cdot v\cdot L}{\eta} \left[-\right]
    \end{equation}
    where:
    \begin{itemize}
        \item $v$ is the mean flow velocity [m/s]
        \item $L$ is the characteristic length [m]
    \end{itemize}
    \begin{examplebox}{Re values}
    \begin{itemize}
        \item $Re < 2000$: laminar flow
        \item $Re \simeq 2300$: critical point
        \item $2000 < Re < 4000$: transitional regime
        \item $Re \geq 4000$: turbulent flow
    \end{itemize}
    \end{examplebox}
\end{formula}

\vfill
\columnbreak

\subsection{Pressure and velocity}
\begin{theorybox}{Pressure}
    \subsubsection{Total pressure}
    In addition to the static pressure $p_{\rm stat}$, there is also the
    dynamic pressure $p_{\rm dyn}$ and the total pressure $p_{\rm tot}$:
    \begin{equation}
        p_{\rm tot} = p_{\rm stat} + p_{\rm dyn}
    \end{equation}

    \subsubsection{Absolute pressure}
    Absolute pressure $p_{\rm abs}$ refers to the pressure in a vacuum
    $p_{\rm vaacum}=0 Pa$ while relative pressure $p_{\rm rel}$ can refer to
    any chosen reference pressure $p_{\rm ref}$.
    \begin{equation}
        p_{\rm abs} = p_{\rm rel} - p_{\rm ref}
    \end{equation}

    \subsubsection{Velocity}
    Velocity is a vector quantity:
    \begin{equation}
        \vec{v} = \left(v_x v_y v_z\right)
    \end{equation}

    The magnitude is given by:
    \begin{equation}
        v = \sqrt{v_x^2 + v_y^2 + v_z^2}
    \end{equation}
\end{theorybox}

\subsection{Curvature pressure formula}
\begin{examplebox}{Deflection motion of a fluid element around a blunt body}
    \includegraphics[width=\textwidth]{media/pressure_curvatur.png}
    \vspace*{-0.6cm}
    \begin{equation}
        \dfrac{dp}{dn} = -\rho\cdot\dfrac{v^2}{R}
    \end{equation}
\end{examplebox}

\vfill
\phantom{}
\end{multicols}

\newpage
\begin{multicols}{2}
\setlength{\columnsep}{1pt}

\section{Mass conservation}
\subsection{Continuity equation / Mass conservation}
\begin{theorybox}{Continuity equation}
    \includegraphics[width=\textwidth]{media/ContinuityBild1.png}
    \subsubsection{Steady mass-flow}
    \begin{equation}
    \dot m_{\rm in} = \dot m_{\rm out}
    \end{equation}

    \subsubsection{Incompressible fluid}
    \begin{equation}
    \dot m = \rho\,\dot V
    \quad\Longrightarrow\quad
    \dot V_{\rm in} = \dot V_{\rm out}
    \end{equation}

    \subsubsection{Streamline theory}
    \begin{equation}
    \dot V = \bar v\,A
    \quad\Longrightarrow\quad
    \bar v_{\rm in}\,A_{\rm in} = \bar v_{\rm out}\,A_{\rm out}
    \end{equation}
\end{theorybox}

\section{Energy conservation}
\subsection{Fluid mechanical energy conservation}
\begin{theorybox}{Derivation of the Bernoulli equation}
\vspace*{-0.3cm}
    \begin{equation}
        \dot{m}_1 \left(\dfrac{p_1}{\rho} + \dfrac{v_1^2}{2} + gz_1\right) = \dot{m}_2 \left(\dfrac{p_2}{\rho} + \dfrac{v_2^2}{2} + gz_2\right)
    \end{equation}

    This derivation is based on the assumption that the system has:
    
    \begin{minipage}[t]{0.48\linewidth}
        \begin{itemize}
            \item steady flow
            \item ideal fluid
            \item adiabatic process
        \end{itemize}
    \end{minipage}
    \hfill
    \begin{minipage}[t]{0.48\linewidth}
        \begin{itemize}
            \item no work in or out of the system
            \item 1D streamline flow
        \end{itemize}
    \end{minipage}
\subsubsection{Energy flow}
\vspace*{-0.5cm}
    \begin{align}
        \frac{dE}{dt} =\ 
        &\underbrace{\sum P + \sum \dot{Q}}_{
            \substack{
                \text{Energy flow} \\ \text{across system boundary}
            }
        } \notag \\
        &+ \underbrace{\sum_{in} \left[\dot{m}^{\swarrow} \cdot \left(h^{\swarrow} + \frac{v^{2\swarrow}}{2} + g z^{\swarrow}\right)\right]}_{
            \substack{
                \text{Energy transfer} \\ \text{mass in}
            }
        } \notag \\
        &- \underbrace{\sum_{out} \left[\dot{m}^{\nearrow} \cdot \left(h^{\nearrow} + \frac{v^{2\nearrow}}{2} + g z^{\nearrow}\right)\right]}_{
            \substack{
                \text{Energy transfer} \\ \text{mass out}
            }
        }
    \end{align}

    \vspace*{-0.5cm}
    \subsubsection{Outflow formula according to Torricelli}
    \begin{equation}
        gz_1 = \frac{v_2^2}{2} \Longrightarrow v_2 = \sqrt{2g\Delta z}
    \end{equation}
\end{theorybox}

\newcolumn

\subsection{Bernoulli equation}
\begin{formula}{Specific energy equation}
    \includegraphics[width=\textwidth]{media/Bernoulli.png}
    \vspace*{-0.3cm}
    \begin{equation}
        \dfrac{p_1}{\rho} + \dfrac{v_1^2}{2} + gz_1 = \dfrac{p_2}{\rho} + \dfrac{v_2^2}{2} + gz_2 = {\rm const.} \left[\dfrac{J}{kg}\right]
    \end{equation}

    \subsubsection{Alternative forms}
    \textbf{Pressure equation}
    \begin{equation}
        p_1 + \dfrac{\rho v_1^2}{2} + \rho g z_1 = p_2 + \dfrac{\rho v_2^2}{2} + \rho g z_2 = {\rm const.} \left[Pa\right]
    \end{equation}

    \textbf{Height equation}
    \begin{equation}
        \dfrac{p_1}{\rho g} + \dfrac{v_1^2}{2g} + z_1 = \dfrac{p_2}{\rho g} + \dfrac{v_2^2}{2g} + z_2 = {\rm const.} \left[m\right]
    \end{equation}
\end{formula}

\begin{theorybox}{True energy equation}
    The Bernoulli equation states that the sum of these energies is constant along a streamline.
    \subsubsection{Pressure energy}
    \begin{equation}
        E_p = m\cdot \frac{p}{\rho} \left[J\right]
    \end{equation}

    \subsubsection{Kinetic energy}
    \begin{equation}
        E_{\rm kin} = m\cdot \frac{v^2}{2} \left[J\right]
    \end{equation}

    \subsubsection{Potential energy}
    \begin{equation}
        E_{\rm pot} = m\cdot g\cdot z \left[J\right]
    \end{equation}

\subsubsection{Energy conservation}
\vspace*{-0.5cm}
\begin{align}
    E_{p,1} + E_{\rm kin,1} + E_{\rm pot,1} &= E_{p,2} + E_{\rm kin,2} + E_{\rm pot,2} \notag \\[1.5ex]
    m\left(\dfrac{p_1}{\rho}+\dfrac{v_1^2}{2}+gz_1\right) &= m\left(\dfrac{p_2}{\rho}+\dfrac{v_2^2}{2}+gz_2\right)
\end{align}
\end{theorybox}

\subsection{Hydrostatics}
\begin{formula}{Fundamental law of hydrostatics}
    \begin{equation}
        p = p_0 + \rho g h = {\rm const.} \left[\rm Pa\right]
    \end{equation}
    derived from:
    \begin{equation}
        p = p_0 + \dfrac{F_g}{A} = p_0 + \dfrac{mg}{A} = p_0 + \dfrac{\rho hAg}{A}
    \end{equation}
\end{formula}

\vfill
\phantom{}
\end{multicols}

\newpage
\begin{multicols}{2}
\setlength{\columnsep}{1pt}

\subsection{Venturi effect experiment}
\begin{examplebox}{Venturi effect}
    \begin{theorybox}{Height - pressure difference at $\dot{V} = 6$ l/s}
        \includegraphics[width=\textwidth]{media/venturi.png}
    \end{theorybox}


    \begin{formula}{Relative static pressure $p_{\rm rel}$}
        \includegraphics[width=\textwidth]{media/venturi_relative.png}
        \vspace*{-0.3cm}
        \begin{equation}
            p_{\rm rel} = p_{\rm hydro} = \rho g \left(h-h_{\rm ref}\right)
        \end{equation}
    \end{formula}
    \begin{formula}{Dynamic pressure $p_{\rm dyn}$}
        \includegraphics[width=\textwidth]{media/venturi_dyn.png}
        \vspace*{-0.3cm}
        \begin{equation}
            p_{\rm dyn} = \rho\dfrac{v^2}{2}
        \end{equation}
    \end{formula}
    \begin{formula}{Dynamic pressure $v$}
        \includegraphics[width=\textwidth]{media/venturi_velocity.png}
        \vspace*{-0.3cm}
        \begin{equation}
            v=\dfrac{\dot{V}}{A}
        \end{equation}
    \end{formula}
    \begin{formula}{Pressure difference $\Delta p$}
        \includegraphics[width=\textwidth]{media/venturi_pressure.png}
        \vspace*{-0.3cm}
        \begin{equation}
           \Delta p = p_{\rm NoFric} - p_{\rm real} \Longrightarrow p_V \sim v^2
        \end{equation}
    \end{formula}
\end{examplebox}

\vfill
\columnbreak
\begin{examplebox}{Venturi effect}
    \begin{theorybox}{Measurament points}
        \includegraphics[width=\textwidth]{media/venturi_points.png}
    \end{theorybox}
        \begin{theorybox}{Measurament shear flow}
        \includegraphics[width=\textwidth]{media/venturi_flow.png}
    \end{theorybox}
\end{examplebox}

\subsection{Contraction coefficient}
\begin{theorybox}{Outflow contraction coefficient $\alpha$}
    \begin{center}
        \includegraphics[width=.75\textwidth]{media/contraction.png}
    \end{center}
    \begin{equation}
        \alpha = \dfrac{A_{\rm actual}}{A_{\rm opening}} = \dfrac{\pi}{2 + \pi} \approx 0.611 [-]
    \end{equation}
\end{theorybox}

\subsection{Energy line diagram}
\begin{theorybox}{Ideal fluid energy line diagram}
    \includegraphics[width=\textwidth]{media/EGL_Duese_EN.PNG}
\end{theorybox}

\begin{theorybox}{Extended energy line diagram}
    \includegraphics[width=\textwidth]{media/04_energyLineDiagram_vGB.png}
\end{theorybox}

\vfill
\end{multicols}

\newpage
\begin{multicols}{2}
\setlength{\columnsep}{1pt}

\subsection{Extended Bernoulli equation}
\begin{formula}{Extension of the Bernoulli equation}
    \vspace*{-0.4cm}
    \begin{equation}
        \dfrac{p_1}{\rho} + \dfrac{v_1^2}{2} + gz_1 + e_A = \dfrac{p_2}{\rho} + \dfrac{v_2^2}{2} + gz_2 + e_V \left[\frac{J}{kg}\right]
    \end{equation}
\end{formula}

\subsubsection{Additional terms}
\begin{theorybox}{Work term $e_A$}
    If energy is added to the fluid along a streamline from point 1 to point 2
    (eg. a pump), the total energy at point 2 becomes higher than at point 1.\\

    \textbf{Sign convention}\\
    $e_A > 0$:work is done on the fluid\\
    \textrightarrow\ energy is added to the fluid (eg. pump);\\

    $e_A < 0$: work is done by the fluid\\
    \textrightarrow\ energy is extracted from the fluid (eg. turbine).
\end{theorybox}

\begin{theorybox}{Loss term $e_V$}
    The effects of a viscous fluid along a stramline from point 1 to point 2 are taken into
    account by the loss term $e_V$.
\end{theorybox}



\vfill
\phantom{}
\newcolumn

ciao

\vfill
\phantom{}
\end{multicols}



















\end{document}