\subsection{Safety}
{\small Research by Frongillo Matteo}

In the context of designing a disposable camping tent, safety requirements are essential to ensure
the product can be used in crowded and temporary environments such as music festivals.
In particular, the adoption of sustainable materials like mycelium requires compliance with safety
standards, especially regarding mechanical performance and fire safety. These aspects are fundamental
to protect users, reduce fire-related risks, and provide an ecological solution that minimizes waste
after use.

\subsubsection{Fire resistance of mycelium}

The fire resistance of mycelium-based composites is particularly advantageous for disposable camping
tents used during mass events like music festivals, where the risk of fire is higher. Compared to
synthetic materials, mycelium composites produce less smoke and toxic gases during combustion, reducing
health risks for participants in case of emergency. The mixture with silica-rich substrates, such as
rice hulls, gives mycelium fire-retardant properties, ensuring greater safety in crowded settings typical
of festivals.

\fullcite{myc-safety}

\subsubsection{Environmental impact and safety}

A key aspect of mycelium for the patent of a disposable tent is its low environmental impact, both during
use and after disposal. Unlike traditional materials such as synthetic foams, mycelium composites release
minimal amounts of CO2 during combustion. This is particularly relevant for a product designed for
temporary use in outdoor settings, as it reduces air pollution and the risk of exposure to toxic substances
during unforeseen events. Moreover, mycelium is biodegradable, eliminating the problem of waste left behind
after festivals.

\fullcite{myc-environment}

\subsubsection{Mechanical safety properties}

The mechanical properties of mycelium are another advantage for disposable camping tents. Thanks to the
compressive strength and structural stability provided by fungal fibers, mycelium tents can withstand
varying environmental conditions during festivals. This ensures that the structure remains solid and safe
for the entire duration of the event, providing both physical protection and fire safety.

\fullcite{myc-mechanical}

\subsubsection{Safety in camping applications}

For the patent of a disposable mycelium tent, it is crucial to meet safety standards for tents, such as
the DIN EN ISO 5912:2020 standard, which focuses on mechanical performance and fire safety. The use of
mycelium meets these requirements and offers additional benefits like biodegradability, making the product
ideal for eco-conscious camping. After use, mycelium tents can naturally biodegrade, helping to reduce the
environmental impact and the amount of waste generated during festivals.

\fullcite{tent-safety}


UTILITIES:

Improving the Physical and Mechanical Properties of Mycelium-Based Green Composites Using Paper Waste
https://www.mdpi.com/2073-4360/16/2/262

Mechanical, Physical, and Chemical Properties of Mycelium-Based Composites Produced from Various Lignocellulosic Residues and Fungal Species
https://www.mdpi.com/2309-608X/8/11/1125

Mechanical, physical and chemical characterisation of mycelium-based composites with different types of lignocellulosic substrates
https://journals.plos.org/plosone/article?id=10.1371/journal.pone.0213954

A review of recent advances in fungal mycelium based composites
https://link.springer.com/article/10.1007/s43939-024-00084-8

THERMAL DEGRADATION AND FIRE REACTION PROPERTIES OF MYCELIUM COMPOSITES
https://www.nature.com/articles/s41598-018-36032-9

DIN ISO (Already in)