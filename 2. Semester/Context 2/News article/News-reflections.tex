\documentclass{article}

\usepackage{Paper}
\pdftitle{MatteoFrongillo_News-reflections}

\title{\textbf{News article -- reflections}\\\textbf{Context 2}}
\author{Matteo Frongillo}
\date{April 24, 2025}

\begin{document}

\maketitle
\pph{Question a}
As an AI, I used ChatGPT 4.5 because, after reading several papers, I saw that it was
optimized for generating text, especially newspaper headlines. I also used it because,
having paid the monthly subscription to OpenAI for ChatGPT, I had access to it.

I could have used other versions of ChatGPT or other AIs such as Claude or DeepSeek, but
these are highly optimized for math and coding tasks, so I would not have used their full
potential if I had asked them to generate a newspaper article.

\pph{Question b}
I admit to using ChatGPT many times as a text-writing assistant since I am still learning
English. I always reread and analyze the answers that the AI generates for me, firstly,
because I want to see the words it used and how, and secondly, because I want to make sure
that it has not misinterpreted the prompt or invented anything.

\pph{Question c}
I instructed ChatGPT that all the answers it gives me should be written in diplomatic
English and that the answers generated should be pragmatic. My English writing is still
quite immature, so I think reading what the AI generates in a high register is good
training for me. So yes, my way of writing differs a lot from ChatGPT’s, but it is my own
choice.

In any case, when I have to use a sentence or paragraph that ChatGPT generates, I always
try to rewrite it in my own words so that the reading is more natural and so that I can
double-check what it has written.

\end{document}