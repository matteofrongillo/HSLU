\documentclass{article}

\usepackage[fleqn]{amsmath}
\usepackage{amssymb}
\usepackage{hyperref}
\usepackage{url}
\usepackage{graphicx}
\usepackage{geometry}
\usepackage{babel}
\usepackage{enumitem}
\usepackage{parskip}
\usepackage{chemfig}
\usepackage{pdfpages}
\usepackage{xcolor}
\usepackage{tikz}
\usepackage{fancybox}
\usepackage{makecell}
\usepackage{pgfplots}
\usepackage{soul}
\usepackage{ulem}
\usepackage{wrapfig}
\usepackage{subcaption}
\usepackage[T1]{fontenc}
\usepackage{esvect}
\usetikzlibrary{arrows}
\usetikzlibrary{decorations.pathreplacing}
\pgfplotsset{compat=1.17}

\geometry{
    a4paper,
    total={170mm, 257mm},
    left=20mm,
    top=20mm
}

\hypersetup{
    colorlinks=true,
    linkcolor=black,
    urlcolor=blue,
    pdftitle={System engineering in EES}
}

\newcommand{\figbox}[1]{ 
    \begin{figure*}[ht!]        
        \begin{center}            
            \fbox{#1}        
        \end{center}    
    \end{figure*}
}

\newcommand{\wrapfill}{
    \par
    \ifnum \value{WF@wrappedlines} > 0
        \addtocounter{WF@wrappedlines}{-1}%
        \null\vspace{
            \arabic{WF@wrappedlines}
            \baselineskip
        }
        \WFclear
    \fi
    \phantom{}
}

\newcommand{\cfig}[1]{%
  \begin{figure*}[ht!]%
    \centering%
    #1%
  \end{figure*}%
}

\newcommand{\difference}{\,\backslash\,}
\newcommand{\rem}{\underline{Remark}: }
\newcommand{\nots}{\underline{Notation}: }
\newcommand{\prf}{\underline{Proof}: }
\newcommand{\exs}{\underline{Example}: }
\newcommand{\defs}{\underline{Definition}: }
\newcommand{\wrn}{\underline{Warning}: }
\newcommand{\sht}{\ |\ }


% === TEXT ===
\title{\textbf{Systems Engineering in\\Environmental and Energie Systems\\ HSLU, Semester 1}}
\author{Matteo Frongillo}

\begin{document}

\maketitle
\tableofcontents
\pagebreak

\part{Week 1}

\section{Waste to energy (WtE)}
Waste to energy refers to a variety of treatment technologies that convert waste to electricity,
heat, fuel or other usable materials, as well as a range of residues.

\setlength{\intextsep}{0pt}%
\begin{wrapfigure}{r}{.5\textwidth}
    \includegraphics[width=.5\textwidth]{media/WtE.png}
    \vspace{-2.6cm}
\end{wrapfigure}

\phantom{}

WtE can occur through a number of processes\\such as:
\begin{itemize}
    \item incineration;
    \item gasification;
    \item pyrolysis;
    \item anaerobic digestion;
    \item landfill gas recovery.
\end{itemize}
\wrapfill

\subsection{Incineration plants}
Thermal waste to energy, also known as incineration with energy recovery, is a major waste
treatment method and the most widely adopted technology that dominates the global WtE market.

An incineration plant is a waste management facility designed to burn solid waste at high temperatures,
converting in into ash, gases and heat.

Incineration plays a crucial role in modern waste management strategies, contibuting to
environmental protection and resource recovery.

\begin{itemize}
    \item Volume reduction of waste (90\%);
    \item Energy production: waste has the same energy value as wood chips or lignite;
    \item Reduction of contaminant spectrum: bacteria, viruses and problematic organic
        compounds are destroyed at or over 1'000 $^\circ$C;
    \item Low pollutant emissions (modern incineration plants);
    \item Avoidance of methane emissions that results from direct deposition of organic
        waste in landfills;
    \item Chemical stability of residues/slag (compared to biological processes);
    \item Possibility of recovering metals from slag (80kg of iron, 20kg of aluminium and 2kg of copper are recovered per each tonne of slag).
\end{itemize}

\subsubsection{4-key Components of an incineration plant}
\begin{enumerate}
    \item Waste handling system: responsible for loading, sorting and feeding waste into
        the combustion chamber;
    \item Combustion chamber: primary unit where waste is incinerated at temperature
        between 800 and 1'200 $^\circ$C;
    \item Air pollution control system: filters and scrubbers to minimize harmful emissions
        and comply with environmental regulations;
    \item Energy recovery system: utilizes the generated heat to produce electricity or heat
        for district heating.
\end{enumerate}

\newpage
\subsubsection{Incineration plant scheme}
\phantom{}
\cfig{\includegraphics*[width=.9\textwidth]{media/incineration-plant-scheme.png}}
\phantom{}

\subsection{Types of wastes that can be converted into energy}
\subsubsection{Municipal Solid Waste (MSW)}
Municipal wastes can be converted into energy by thermochemical or biological technologies.

At the landfill sites, gasses produced by the natural decomposition of MSW can be collected,
scrubbed and cleaned before feeding into internal combustion engines or gas turbines to
generate heat and power.

The organic fraction of MSW can be biochemically stabilizen in an anaerobic digester to
obtain biogas (for heating and power) as well as fertilizer.

\subsubsection{Wood Wastes}
Wood processing industries primarly include sawmilling, plywood, wood panel, furniture,
building component, flooring, particle board, moulding, jointing and craft industries.



\subsection{Current status of waste to energy}
IMMAGINE

\subsubsection{Waste production per person in CH}
In Switzerland, people produce 700kg of waste per person per year.

\subsection{Waste hierarchy}
The hierarchy helps us rethink our relationship with waste based on
five priorities ranked in terms of what is best for the environment:

\begin{enumerate}
    \item Produc prevention;
    \item Preparing for re-use;
    \item Recycle;
    \item Recovery;
    \item Waste disposal.
\end{enumerate}

\subsection{Advantages and disadvantages of the Swiss system}
\subsubsection{Advantages}
...

\subsubsection{Disadvantages}
...

\section{System thinking}
\subsection{Benefits}
Rigorous way of integrating: people, purposes, process and performance and:
\begin{itemize}
    \item ...
\end{itemize}

\subsection{Feedback loops}
...

\section{Case study part 1}
...

\part{Week 2}
\section{Situation analysis and system thinking}








\end{document}
