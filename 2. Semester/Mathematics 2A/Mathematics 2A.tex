\documentclass{article}

\usepackage{Mathematics}
\pdftitle{Mathematics 2A}

% === TITLE ===
\title{\textbf{Mathematics 2A \\ HSLU, Semester 2}}
\author{Matteo Frongillo}
\date{}

% === TEXT ===
\begin{document}

\maketitle
\tableofcontents
\pagebreak

\part{Differential Equations Theory}

\section{Introduction}
A \textbf{differential equation} is an equation in which derivatives of an unknown function appear. For example, consider the simple differential equation
\figbox{$\dfrac{dH}{dG}=H$}
\rem{This equation asserts that the instantaneous rate of change of \(H\) with respect to \(G\) equals \(H\) itself. Its general solution is \(H(G)=\mathcal{C}e^G\) with \(\mathcal{C}\) an arbitrary constant.}

\section{Separation of Variables}
For a separable differential equation of the form
\[
\frac{dy}{dx}=f(x)\,g(y),
\]
we rewrite it as
\[
\frac{dy}{g(y)}=f(x)\,dx.
\]
\figbox{$\displaystyle \int\frac{dy}{g(y)}=\int f(x)\,dx$}
\rem{After integration, one typically obtains an implicit solution that can be solved (if possible) for \(y\).}
\wrn{Ensure that \(g(y)\neq 0\) on the interval of interest.}

\section{Linear Differential Equations}
A first-order linear differential equation can be written in the standard form
\[
y'+p(x)y=q(x).
\]
Its general solution is given by
\[
y=y_h+y_p,
\]
where \(y_h\) is the general solution of the homogeneous part
\[
y'+p(x)y=0,
\]
and \(y_p\) is any particular solution of the full inhomogeneous equation.
\figbox{$\dm y_h=A\exp\left({-\int p(x)dx}\right)$}
\rem{The principle of superposition applies to the homogeneous equation; that is, any linear combination of solutions is again a solution.}

\section{Exponential Growth and Decay}
Many natural processes obey the simple law
\[
\frac{dP}{dt}=kP.
\]
Its general solution is
\[
P(t)=P(0)e^{kt}.
\]
\figbox{$P(t)=P_0e^{kt}$}
\rem{This model applies not only to population growth but also to radioactive decay (with \(k<0\)).}

\section{Graphical Representation: Slope Fields}
A slope field (or direction field) helps visualize the behavior of solutions of a differential equation by drawing, at selected points \((x,y)\), short line segments whose slope is given by the value of \(f(x,y)\) in
\[
y'=f(x,y).
\]
For example, for the differential equation
\[
y'=y,
\]
the slope at each point is simply the \(y\)-value. The following TikZ figure illustrates a portion of this slope field.

\begin{center}
\begin{tikzpicture}[scale=0.8]
  \draw[->] (-3,0) -- (3,0) node[right] {\(x\)};
  \draw[->] (0,-2) -- (0,2) node[above] {\(y\)};
  % Draw sample slope segments for y' = y
  \foreach \x in {-2,-1,...,2} {
    \foreach \y in {-1.5,-0.5,...,1.5} {
      \pgfmathsetmacro{\slope}{\y}
      \pgfmathsetmacro{\dx}{0.2}
      \pgfmathsetmacro{\dy}{\slope*0.2}
      \draw (\x-\dx,\y-\dy) -- (\x+\dx,\y+\dy);
    }
  }
\end{tikzpicture}
\end{center}
\rem{For \(y'=y\), the slope at each point equals its \(y\)-coordinate. Thus, solution curves such as \(y=\mathcal{C}e^x\) naturally emerge from the field.}

\part{Mathematical Formulary}

\section{Lines and Linear Functions}
\subsection{Slope and Equation of a Line}
\figbox{$m=\displaystyle \frac{y_2-y_1}{x_2-x_1}$}
\figbox{$y-y_1=m(x-x_1)$}
\figbox{$y=mx+b$}
\rem{These formulas describe the fundamental properties of straight lines in the Cartesian plane.}

\section{Exponents and Logarithms}
\subsection{Working with Exponents}
\figbox{$a^x\cdot a^t=a^{x+t}$}
\figbox{$\displaystyle \frac{a^x}{a^t}=a^{x-t}$}
\figbox{$(a^x)^t=a^{xt}$}
\subsection{Definition of the Natural Logarithm}
\figbox{\(y=\ln x\Longleftrightarrow e^y=x\)}
\rem{For instance, \(\ln 1=0\) because \(e^0=1\).}
\subsection{Logarithmic Identities}
\figbox{\(\ln(AB)=\ln A+\ln B\)}
\figbox{\(\ln\left(\frac{A}{B}\right)=\ln A-\ln B\)}
\figbox{\(\ln A^p=p\ln A\)}

\section{Distances and Midpoint Formulas}
\subsection{Distance Formula}
\figbox{\(D=\sqrt{(x_2-x_1)^2+(y_2-y_1)^2}\)}
\subsection{Midpoint Formula}
\figbox{\(\left(\frac{x_1+x_2}{2},\,\frac{y_1+y_2}{2}\right)\)}

\section{Quadratic Equations}
\figbox{\(ax^2+bx+c=0\quad\Rightarrow\quad x=\frac{-b\pm\sqrt{b^2-4ac}}{2a}\)}

\section{Factoring Special Polynomials}
\figbox{\(x^2-y^2=(x+y)(x-y)\)}
\figbox{\(x^3+y^3=(x+y)(x^2-xy+y^2)\)}
\figbox{\(x^3-y^3=(x-y)(x^2+xy+y^2)\)}

\section{Conic Sections}
\subsection{Circles}
\figbox{\((x-h)^2+(y-k)^2=r^2\)}
\subsection{Ellipses}
\figbox{\(\frac{x^2}{a^2}+\frac{y^2}{b^2}=1\)}
\subsection{Hyperbolas}
\figbox{\(\frac{x^2}{a^2}-\frac{y^2}{b^2}=1\)}
\rem{The asymptotes of a hyperbola are given by \(y=\pm\frac{b}{a}x\).}

\section{Geometric Formulas}
\subsection{Conversion Between Radians and Degrees}
\figbox{\(\pi\text{ radians}=180^\circ\)}
\subsection{Circle Geometry}
\figbox{\(A=\pi r^2,\quad C=2\pi r\)}
\subsection{Sector of a Circle}
\figbox{\(A=\frac{1}{2}r^2\vartheta,\quad s=r\vartheta\) \(\vartheta\) in radians.}
\subsection{Volumes and Surface Areas of Solids}
\begin{itemize}
  \item Sphere: \(V=\frac{4}{3}\pi r^3,\quad A=4\pi r^2\).
  \item Cylinder: \(V=\pi r^2h\).
  \item Cone: \(V=\frac{1}{3}\pi r^2h\).
\end{itemize}

\section{Trigonometric Functions and Identities}
\subsection{Definitions}
For a right triangle with hypotenuse \(r\) and legs \(x\) and \(y\):
\figbox{\(\sin\vartheta=\frac{y}{r},\quad \cos\vartheta=\frac{x}{r},\quad \tan\vartheta=\frac{y}{x}\)}
\subsection{Fundamental Identity}
\figbox{\(\sin^2\vartheta+\cos^2\vartheta=1\)}
\subsection{Angle Sum and Difference Formulas}
\figbox{\(\sin(A\pm B)=\sin A\cos B\pm \cos A\sin B\)}
\figbox{\(\cos(A\pm B)=\cos A\cos B\mp \sin A\sin B\)}
\subsection{Double Angle Formulas}
\figbox{\(\sin(2A)=2\sin A\cos A\)}
\figbox{\(\cos(2A)=2\cos^2A-1=1-2\sin^2A\)}

\section{Binomial Expansions}
The binomial expansion for \((x+y)^n\) is given by
\figbox{\((x+y)^n=x^n+n\,x^{n-1}y+\frac{n(n-1)}{2!}x^{n-2}y^2+\cdots+y^n\)}
\rem{For \((x-y)^n\), the signs alternate accordingly.}

\section{Differentiation Rules}
\begin{enumerate}
  \item \((f(x)\pm g(x))' = f'(x)\pm g'(x)\).
  \item \((k\,f(x))' = k\,f'(x)\).
  \item \((f(x)g(x))' = f'(x)g(x)+f(x)g'(x)\).
  \item \(\left(\frac{f(x)}{g(x)}\right)'=\frac{f'(x)g(x)-f(x)g'(x)}{(g(x))^2}\).
  \item \((f(g(x)))' = f'(g(x))\cdot g'(x)\).
  \item \(\frac{d}{dx}(x^n)=nx^{n-1}\).
  \item \(\frac{d}{dx}(e^x)=e^x\).
  \item \(\frac{d}{dx}(a^x)=a^x\ln a,\quad a>0\).
  \item \(\frac{d}{dx}(\ln x)=\frac{1}{x}\).
  \item \(\frac{d}{dx}(\sin x)=\cos x\).
  \item \(\frac{d}{dx}(\cos x)=-\sin x\).
  \item \(\frac{d}{dx}(\tan x)=\frac{1}{\cos^2 x}\).
  \item \(\frac{d}{dx}(\arcsin x)=\frac{1}{\sqrt{1-x^2}}\).
  \item \(\frac{d}{dx}(\arctan x)=\frac{1}{1+x^2}\).
\end{enumerate}

\section{Integration Rules}
\begin{enumerate}
  \item \(\displaystyle \int (f(x)\pm g(x))\,dx=\int f(x)\,dx\pm\int g(x)\,dx\).
  \item \(\displaystyle \int k\,f(x)\,dx=k\int f(x)\,dx\).
  \item \(\displaystyle \int f(g(x))g'(x)\,dx=\int f(w)\,dw,\quad w=g(x)\).
  \item \(\displaystyle \int u(x)v'(x)\,dx = u(x)v(x)-\int u'(x)v(x)\,dx.\)
\end{enumerate}

\section{Taylor Series Expansions}
The Taylor series of \(f(x)\) about \(x=a\) is
\[
f(x)=f(a)+f'(a)(x-a)+\frac{f''(a)}{2!}(x-a)^2+\frac{f'''(a)}{3!}(x-a)^3+\cdots.
\]
Important examples include:
\begin{align*}
e^x &= 1+x+\frac{x^2}{2!}+\frac{x^3}{3!}+\cdots,\\[1ex]
\cos x &= 1-\frac{x^2}{2!}+\frac{x^4}{4!}-\frac{x^6}{6!}+\cdots,\\[1ex]
\sin x &= x-\frac{x^3}{3!}+\frac{x^5}{5!}-\frac{x^7}{7!}+\cdots,\\[1ex]
\frac{1}{1-x} &= 1+x+x^2+x^3+\cdots\quad(|x|<1),\\[1ex]
(1+x)^p &= 1+px+\frac{p(p-1)}{2!}x^2+\frac{p(p-1)(p-2)}{3!}x^3+\cdots.
\end{align*}

\section{Complex Numbers and Euler's Formula}
A complex number \(z\) is written as
\[
z=x+yj,\quad x,y\in\mathbb{R}.
\]
Its magnitude is
\[
|z|=\sqrt{x^2+y^2},
\]
and its conjugate is
\[
\bar{z}=x-yj.
\]
Euler's formula states that
\[
e^{jt}=\cos t+j\sin t,
\]
so any complex number can be written in polar form as
\[
z=re^{j\varphi},\quad r\ge0,\; -\pi<\varphi\le \pi.
\]

\end{document}
