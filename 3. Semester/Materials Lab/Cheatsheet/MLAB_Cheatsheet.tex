\documentclass[9pt, landscape, a4paper]{article}

\usepackage{fontspec}
\renewcommand{\normalsize}{\fontsize{9}{11}\selectfont}
\usepackage[english]{babel}
\babelfont{rm}{Noto Sans}
\babelfont{sf}{Noto Sans}

\usepackage{amsmath}
\usepackage{amssymb}
\usepackage{hyperref}
\usepackage{url}
\usepackage{graphicx}
\usepackage{geometry}
\usepackage{enumitem}
\usepackage{parskip}
\usepackage{chemfig}
\usepackage{pdfpages}
\usepackage{xcolor}
\usepackage{tikz}
\usepackage{fancybox}
\usepackage{makecell}
\usepackage{pgfplots}
\usepackage{ulem}
\usepackage{wrapfig}
\usepackage{subcaption}
\usepackage{esvect}
\usepackage{siunitx}
\usepackage{mathtools}
\usepackage{varwidth}
\usepackage{multicol}
\usepackage{titlesec}
\usepackage{bm}
\usepackage{forest}

\usetikzlibrary{arrows}
\usetikzlibrary{arrows.meta}
\usetikzlibrary{decorations.pathreplacing}
\pgfplotsset{compat=1.18}
\definecolor{darkgreen}{rgb}{0.0, 0.6, 0.0}

% ============= GEOMETRY =============
\geometry{
    a4paper,
    landscape,
    left=0.5cm,
    right=0.5cm,
    top=1.5cm, 
    bottom=0.5cm,
    footskip=0cm
}

\setlist{nosep, leftmargin=*}

\AtBeginDocument{
  \setlength{\abovedisplayskip}{3pt}
  \setlength{\belowdisplayskip}{3pt}
  \setlength{\abovedisplayshortskip}{0pt}
  \setlength{\belowdisplayshortskip}{0pt}
}

% ============= HEADER & TITLES =============
\usepackage{fancyhdr}
\pagestyle{fancy}
\fancyhf{}
\rhead{\small \vspace*{.1cm}Materials Lab Cheatsheet}
\lhead{\small \vspace*{.1cm}Matteo Frongillo}
\cfoot{\small \vspace*{.4cm}\thepage}
\renewcommand{\headrulewidth}{0pt}
\setlength{\headsep}{4pt}

\titlespacing*{\part}{0pt}{2pt}{1pt}
\titlespacing*{\section}{0pt}{2pt}{1pt}
\titlespacing*{\subsection}{0pt}{2pt}{1pt}
\titlespacing*{\subsubsection}{0pt}{1pt}{1pt}

\titleformat{\part}{\normalfont\large\bfseries\color{black}}{\thepart.}{0.5em}{}
\titleformat{\section}{\normalfont\normalsize\bfseries\color{red!80}}{\thesection.}{0.5em}{}
\titleformat{\subsection}{\normalfont\normalsize\bfseries\color[HTML]{007AFF}}{\thesubsection}{0.5em}{}
\titleformat{\subsubsection}{\normalfont\small\bfseries\color[HTML]{007355}}{\thesubsubsection}{0.5em}{}

% ============= META DATA =============
\newcommand{\pdftitle}[1]{\hypersetup{
  colorlinks=true,
  linkcolor=black,
  urlcolor=blue,
  pdftitle={#1},
  pdfauthor={Matteo Frongillo}
}}

% ============= RENEW COMMANDS =============

\makeatletter
\renewcommand{\author}[1]{%
  \def\@author{%
    #1\\
    \parbox{\textwidth}{%
      \centering
      \small Last update: \today
    }
  }%
}
\makeatother

\let\emptyset\varnothing

\makeatletter
\pgfmathdeclarefunction{arctan}{1}{%
    \begingroup
    \pgfmathparse{rad(atan(#1))}%
    \endgroup
}
\makeatother

% ============= NEW COMMANDS =============

\newcommand{\figbox}[1]{ 
    \fbox{\begin{varwidth}{\textwidth} #1 \end{varwidth}}        
}

\newcommand{\wrapfill}{
    \par
    \ifnum \value{WF@wrappedlines} > 0
        \addtocounter{WF@wrappedlines}{-1}%
        \null\vspace{
            \arabic{WF@wrappedlines}
            \baselineskip
        }
        \WFclear
    \fi
    \phantom{}
}

\ExplSyntaxOn
\DeclareDocumentCommand{\integral}{d[] d[] d[] d[]}
{
    \IfNoValueTF{#3}{
        \IfNoValueTF{#4}{
            \displaystyle \int #1\,d#2
        }{
            \text{Error: either 2 or 4 arguments must be provided}
        }
    }{
        \IfNoValueTF{#4}{
            \text{Error: either 2 or 4 arguments must be provided}
        }{
            \displaystyle
            \int\limits
                \IfNoValueF{#1}{\c_math_subscript_token{#1}}
                ^{\IfNoValueTF{#2}{}{#2}}
                #3\, d#4
        }
    }
}
\ExplSyntaxOff

\newcommand*\circled[1]{\tikz[baseline=(char.base)]{
            \node[shape=circle,draw,inner sep=1pt] (char) {\small #1};}}

\makeatletter
\newcommand*{\rom}[1]{\expandafter\@slowromancap\romannumeral #1@}
\makeatother

\newcommand{\difference}{\,\backslash\,}
\newcommand{\rem}{\underline{Remark}: }
\newcommand{\nots}{\underline{Notation}: }
\newcommand{\note}{\underline{Note}: }
\newcommand{\prf}{\underline{Proof}: }
\newcommand{\exs}{\underline{Example}: }
\newcommand{\defs}{\underline{Definition}: }
\newcommand{\wrn}{\underline{Warning}: }
\newcommand{\sht}{\ |\ }
\newcommand{\pph}[1]{\paragraph{#1} \phantom{}\\}
\newcommand{\dm}{\displaystyle}
\newcommand{\rad}{^{\mathrm{c}}}
\newcommand{\dsum}[2]{\displaystyle \sum_{#1}^{#2}}
\newcommand{\dx}[2]{\dfrac{{\mathrm{d}}^{#1}{#2}}{\mathrm{d}x^{#1}}}
\newcommand{\cel}{$^\circ$C }

% ============= DOCUMENT CONTENT =============

\begin{document}
\pdftitle{MLAB-Cheatsheet}
\vspace*{-.5cm}
\setlength{\columnsep}{0.5cm}

\begin{multicols*}{3}
\vspace*{-.5cm}

\part{Physical metallurgy}
\section{Classes and properties}
\begin{center}
    \begin{tabular}{|c|l|}
    \hline Class & Typical properties\\
    \hline \makecell{Metal\\Alloys} & \makecell[l]{1) Thermal + electric conductivity\\2) Ductility\\3) Castable\\4) Reflective}\\
    \hline Ceramics & \makecell[l]{1) High T resistance (High E, low $\alpha$)\\2) High compression strength\\3) Thermal + electric insulator\\4) Wear resistance}\\
    \hline Polymers & \makecell[l]{1) Cheap\\2) Thermal + electric insulator\\3) Corrosion resistance\\4) Moldable}\\
    \hline
    \end{tabular}
\end{center}

\subsection{Structural model of metals}
\[\phi = \dfrac{\text{Volume occupied by atoms in unit cell}}{\text{Total volume of unit cell}}\]
\vspace*{0cm}

\subsubsection{FCC (Face-centered cubic)}
\begin{minipage}{0.3\linewidth}
    \centering
    \includegraphics[width=.65\linewidth]{media/fcc.png}
\end{minipage}%
\begin{minipage}{0.7\linewidth}
    \begin{itemize}
        \item Packing efficiency: $\phi \approx 74\%$
        \item Many slip systems (12)
        \item Closest pack direction
    \end{itemize}
\end{minipage}

\subsubsection{BCC (Body-centered cubic)}
\begin{minipage}{0.3\linewidth}
    \centering
    \includegraphics[width=.65\linewidth]{media/bcc.png}
\end{minipage}%
\begin{minipage}{0.7\linewidth}
    \begin{itemize}
        \item Packing efficiency: $\phi \approx 68\%$
        \item Many slip systems (6)
        \item Not closest pack direction
        \item Cottrell atmosphere
    \end{itemize}
\end{minipage}

\subsubsection{HCP (Hexagonal close-packed)}
\begin{minipage}{0.3\linewidth}
    \centering
    \includegraphics[width=.65\linewidth]{media/hcp.png}
\end{minipage}%
\begin{minipage}{0.7\linewidth}
    \begin{itemize}
        \item Packing efficiency: $\phi \approx 74\%$
        \item Very few slip systems (3)
        \item Closest pack direction
    \end{itemize}
\end{minipage}

\subsection{Structural model of ceramics}
\begin{itemize}
    \item Ionic bonding, complex crystal structures (ceramics), amorphous (glasses)
    \item Brittle, but high chemical and thermal resistance
    \item Insulators
    \item Wear-resistant (e.g. ferro-/piezoelectricity)
\end{itemize}

\subsection{Structural model of polymers}
\begin{itemize}
    \item Macromolecules (10$^3$ to 10$^5$ C atoms)
    \item Weaker intermolecular bonds but strong atomic bond in molecular chain
    \item Electrically and thermally insulating
    \item Cheap, moldable, massive waste problem, pollutant
    \item Matrix for many composite materials (no recycling)
\end{itemize}

\subsection{Amorphous and crystalline materials}
\begin{center}
    \includegraphics[width=.9\linewidth]{media/mono-poly-amorph.png}
\end{center}

\subsubsection{Amorphous materials}
\begin{itemize}
    \item No crystal lattice (e.g. quartz glass, polymers)
    \item Atomic distances defined by chemical bonds
    \item Bond angles are variable
\end{itemize}
\vspace*{-.2cm}
\textbf{Field}: inorganic classes (also Gorilla glass), metallic glasses (ferrous
transformer sheet metal), amorphous plastics (PMMA - plexiglass, COC, ...)

\subsubsection{Crystalline materials}
\begin{itemize}
    \item Crystal lattice (e.g. metals, ceramics, quartz)
    \item Atomic distances and bonding angles are defined
\end{itemize}

\subsubsection{Monocrystalline materials}
Only for special applications, expensive.\\
\textbf{Field}: single-crystal turbine blades (T $>$ 1000$^\circ$C, creep-resistant),
semiconductors, MEMS components made of silicon (gyroscopes in smartphones, accelerometers),
optical elements (laser crystals, $\lambda$/4 plates, crystals for frequency doubling of lasers).

\subsubsection{Polycrystalline materials}
Most metal components are polycr. (made of many grains/crystals).

\subsection{Directional dependence}
\subsubsection{Anisotropy and Isotropy}
\begin{itemize}
    \item \textbf{Anisotropic}: properties depend on the direction\\(eg. single crystals, wood, composites)
    \item \textbf{Isotropic}: properties do not depend on direction\\(eg. polycr. metals, amorphous materials)
    \item \textbf{Quasi-isotropic}: properties microscopically depend on the direction but not macroscopically
\end{itemize}

\subsection{Polymorphism (Allotropy)}
\subsubsection{Iron}
\begin{itemize}
    \item $\alpha$-Fe (ferrite, BCC) $\rightarrow$\ below 911\cel
    \item $\gamma$-Fe (austenite, FCC) $\rightarrow$\ 911\cel to 1392\cel
    \item $\delta$-Fe (ferrite, BCC) $\rightarrow$\ 1932\cel to 1536\cel
\end{itemize}

\subsubsection{Carbon}
\begin{center}
    \includegraphics[width=\linewidth]{media/carbon_poly.png}
\end{center}

\subsubsection{Other materials}
\begin{itemize}
    \item Titanium: HCP $<$ 880\cel, BCC $>$ 880\cel
    \item Shape memory allory (eg. NiTi)
    \item Zirconia (high crack resistance due to phase transformation toughening)
    \item Ferro- and piezoelectric materials (eg. PZT, quartz, ...)
\end{itemize}



\subsubsection{Shape memory alloy (SMA)}
\begin{minipage}{0.5\linewidth}
    \begin{center}
        \includegraphics[width=.9\linewidth]{media/Nitinol.png}
    \end{center}
\end{minipage}
\begin{minipage}{0.5\linewidth}
    NiTi is a SMA used for screen lock of tablet notebooks, medtech and
    spectacle frames.
\end{minipage}%

\subsection{Microstructure and phases}
\subsubsection{Homogeneous microstructure}
They have only one phase and crystal structure. A phase can be
either crystalline or amorphous (eg. only iron crystals).

\subsubsection{Heterogeneous microstructure}
They have multiple phases and many types of crystal structures
(eg. graphite and iron).

\subsection{Alloys}
An alloy is a metallic material of at least 2 types of atoms:
\vspace*{-.2cm}
\begin{itemize}
    \item Metal + metal (iron-nickel, gold-silver, tin-lead, aluminum-copper)
    \item Metal + non-metal (iron-carbon, nickel-phosphorus)
\end{itemize}

\subsubsection{Microstructure of alloys}
\begin{itemize}
    \item \textbf{Solid solution crystal}: homogeneous, single-phase, only one type of crystal
    \item \textbf{Mix of different crystal types}: heterogeneous, multi-phase
\end{itemize}

\section{Metal structures and crystal lattice defects}
\subsection{Lattice defects}
\subsubsection{0-dimensional defects}
Point defects due to vacancies and impurity atoms.

\subsubsection{1-dimensional defects}
Line defects due to dislocations. Edge dislocations insert an
extra half-plane of atoms in the crystal, distorting the nearby
planes of atoms.

\subsubsection{2-dimensional defects}
Surface defects due to grain boundaries. Crystal growth starts at
multiple locations within the molten metal\\
\textbf{Crystallization from a melt}: (1) homogeneous melt, (2)
nucleation of crystals, (3) crystal growth surrounded by residual melt,
(4) fully solidified polycrystalline structure with grain boundaries.

\subsubsection{3-dimensional defects}
Volume defects due to precipitations, inclusions, voids, cracks in
the crystal structure.

\section{Elastic and plastic deformation}
\subsection{Elastic deformation}
The coefficient of thermal expansion $\alpha$ is inversely proportional
to: Young's Modulus $E$, Bonding energy, Melting temp.

\subsection{Elastic constants of isotropic materials}
\subsubsection{Elastic stress, strain, and Young's modulus}
\[\varepsilon_x = \frac{\sigma_x}{E} \Longleftrightarrow \sigma_x = \frac{E}{\varepsilon_x}\]

\subsubsection{Poisson's ratio $\nu$}
When a material is stretched in x, it tends to contract in y, z:
\[\nu = -\frac{\varepsilon_{y,z}}{\varepsilon_x}\]

\subsubsection{Relation between 3 isotropic elastic constants $G$}
\[G = \frac{E}{2(1+\nu)} = \frac{\sigma}{2\varepsilon_x(1+\nu)}\]

\subsection{Plastic deformation in metals}
The plastic deformation is permanent and non-reversible

\subsubsection{At room temp T$_\text{R}$}
\begin{itemize}
    \item Dislocations move on densely packed slip planes in densely packed directions
    \item Smaller slip distances require less external force or energy
\end{itemize}

\subsubsection{At high temps}
The metal creeps, leading to diffusion of atoms, especially at grain boundaries.

\subsection{Simplified dislocation slip model}
\begin{center}
    \includegraphics[width=.84\linewidth]{media/simplified_dislocation_slip_model.png}
\end{center}

\subsection{Slip systems}
\subsubsection{Slip systems in FCC metals}
FCC metals have 12 close packed systems, making them \textbf{soft and highly ductile}
(eg. Au, Ag, Cu, Al, $\alpha$-Fe).

\subsubsection{Slip systems in HCP metals}
HCP metals are closely packed but deform on only one slip plane with 3 slip systems,
resulting in \textbf{limited ductility} (eg. Ti, Zn, Mg).

\subsubsection{Slip systems in BCC metals}
BCC metals have 48 slip systems but are less closely packed, leading
to \textbf{higher strength and lower ductility} (eg. $\alpha$-Fe, Cr,
W, Mo, Ta, Nb).

\subsection{Metals crystal structure vs Ductility}
\begin{center}
    \includegraphics[width=\linewidth]{media/structure_vs_ductility.png}
\end{center}

\subsection{Cottrell atmospheres and Dislocation pinning}
\begin{itemize}
    \item In $\alpha$-Fe with a BCC structure (ferrite), the sites for interstitial atoms are much smaller than $\gamma$-Fe (austenite)
    \item Carbon atmos in ferrite diffuse into the distortion fields near dislocation lines, forming Cottrell atmospheres
    \item Causes upper yield point (R$_\text{eH}$) in tensile tests and brittle fracture at low temps
    \item During plastic deformation, dislocations must first break freefrom the Cottrell atmosphere. 
\end{itemize}

\section{Strenghtening mechanisms}
\begin{center}
    \includegraphics[width=\linewidth]{media/strengthening.png}
\end{center}

\subsection{Solid-solution hardening}
\begin{itemize}
    \item Impurity atoms in a solid solution create lattice distortion fields that impede dislocation motion
    \item Interstitial atoms cause stronger lattice distortions than substitutionals atoms, leading to a greared strengthening effect
    \item A larger atomic radius mismatch and higher impurity concentration both increase the stremgthening effect
    \item \textbf{Result}: increased strength but reduced ductility
\end{itemize}

\subsubsection{0-d: SSH application field}
Al-Mb and Al-Mg alloys (5000 and 3000) for automotive sheet metal,
airplane outer skin, beverage cans;
Structural and stainless steels; Gold jewerly (Au with Ag, Cu, Ni,
Pt, Pd, ...)

\subsection{1-d: Strain hardening (Work hardening)}
\begin{itemize}
    \item Cold work increases dislocation density $\rightarrow$\ dislocation entanglement
    \item Cold working / cold forming: below the recrystallization temp
    \item Hot forming: above the recrystallization temp
    \item Recrystallization temp: $(T_{\mathrm{recr}} \approx 0.4,T_m\ [\text{K}])$
    \item \textbf{Result}: strength increases but ductility decreases
\end{itemize}

\subsubsection{SH application field}
Any cold-formed parts: cold rolled sheet metal, neckline holes in
sheet metal of a washing machine drum, cold pressed steels.

\subsection{2-d: Grain boundary hardening}
\begin{itemize}
    \item Smaller grains increase strength: grain boundaries hinder dislocation slip
    \item Not suitable at high temp: grain growth reduces strength and ductility
    \item \textbf{Result}: strength and ductility both increased (statistical grain orientation, Schmid's law)
\end{itemize}

\subsubsection{GBH application field}
High-strength fine-grain or even Q+T structural steels:
cranes, oil platforms, bridges, shipbuilding.

\subsection{3-d: Precipitation hardening}
\begin{itemize}
    \item Increase in strength by coherence of the precipitates and size and number of the precipitates
    \item \textbf{Result}: increase in strength but decrease in ductility
\end{itemize}

\subsubsection{Effect of precipitate coherence on strength}
\textbf{Coherent}: lattice planes match, strong strain fields in the matrix $\rightarrow$\ large strength increase.\\
\textbf{Semi-coherent}: misfit dislocations at the interface reduce lattice strain $\rightarrow$\ strength decreases.\\
\textbf{Incoherent}: lattice planes mismatch; little matrix distortion\\$\rightarrow$\ small strength increase.

\subsubsection{PH application field}
High-strength aluminum alloys for aerospace, maraging steels, stainless PH steels,
highly temp-resistant Ni base superalloys for turbine blades.
\vfill\phantom{}

\newcolumn
\part{Strength and Ductility}
\section{Properties of material}
\begin{center}
    \includegraphics[width=\linewidth]{media/propertiesofmaterial.png}
\end{center}

\subsection{Failure hypotesis and Material testing methods}
\begin{center}
    \includegraphics[width=\linewidth]{media/failures.png}
\end{center}

\section{Tensile test}
\subsection{Engineering stress and stress conditions}
\subsubsection{Normal vs Engineering vs Shear stress}
Engineering stress is the force $F$ acting on the original
cross-sectional area $S_0$.
Normal stress is the normal force $F_N$ that acts
perpendicularly to $S_0$.
The shear stress $\tau$ is the force $F_Q$ parallel to $S_0$.
\[\sigma=\frac{F}{S_0}\quad;\quad \sigma_N = \frac{F_N}{S_0}\quad;\quad \tau = \frac{F_Q}{S_0}\quad;\quad \sigma = E\cdot \varepsilon\]

\subsubsection{Engineering strain $\varepsilon$}
Engineering strain is the ratio of the change in length to the
original length of the material under load:
\[\varepsilon = \frac{\Delta L}{L_0}\]

\subsection{Stress-strain behavior of metals}
\begin{center}
    \includegraphics[width=\linewidth]{media/stressstrain.png}
\end{center}

\newcolumn
\subsubsection{Yield stress R}
\begin{itemize}
    \item \textbf{R$_\text{eH}$}: Upper Yield point
    \item \textbf{R$_\text{p0.2}$}: 0.2\% Yield Stress
    \item Without max yield point: $\sigma_\text{max}$ = R$_\text{p0.2}$
\end{itemize}

\subsubsection{Yound's modulus}
Slope of the linear-elastic region: \[E=\frac{\Delta\sigma}{\Delta\varepsilon}\]

\subsubsection{Tensile strength R$_\text{m}$}
It correspondes to the stress at the max of the stress-strain curve.
With max yield point: $\sigma_\text{max}$ = R$_\text{m}$; upper yield point $R_\text{eH}$.

\subsection{Fracture strain A}
Plastic strain at fracture is defined with respect to the initial
specimen length $L_0$. (eg. $L_0 = 50$mm $\rightarrow A_{50\text{mm}}$)

\subsubsection{Uniform strain A$_\text{g}$}
It corresponds to the plastic strain at maximum load before necking
begins.

\subsubsection{Contraction at fracture Z}
\[Z = \frac{\Delta S}{S_0}\]

\section{Other quasi-static mechanical tests}
\subsection{Bending test}
A specimen is placed on two supports, loaded in bending (typically at midspan) until failure, and the maximum bending stress is calculated.

\subsection{Torsion test}
A cylindrical specimen is clamped and twisted by applying torque until yielding or fracture, and the torsion strength is taken from the peripheral shear stress.

\subsection{Creep and relaxation tests (High temps)}
\begin{itemize}
    \item Creep: time-dependent deformation under constant stress
    \item Relaxation: decreasing stress under constant strain
    \item At high T, strength becomes time- and strain-rate-dependent
    \item Strain changes with time under constant load
    \item Creep/relaxation tests assess heat-resistant materials for high-T applications (eg. steels and Ni alloys above 400\cel)
\end{itemize}

\part{Steel - technology and applications}
Stees is a Carbon-iron alloy. Has very good properties: strength,
ductility, toughness, formability, machining, weldability. Can be
heat-treated due its polymorphism.

\section{Steel Technology}
\subsection{Blast furnace and Pig iron}
\begin{itemize}
    \item \textbf{Process}: reduction of iron oxide with coke\\
        Fe$_2$O$_3$ + 3C $\rightarrow$\ 2Fe + 3CO
    \item \textbf{Product}: pig iron (3-5\% C, also Mn, Si, S, P)
\end{itemize}

\subsection{Crude steel production}
\begin{itemize}
    \item Pig iron refined to crude steel by reducing carbon content
    \item Oxygen-Blown Converter (OBC): refining with oxygen
    \item Electric Furnace (EF): melts scrap steel or direct-reduced iron (sponge iron) to crude steel
\end{itemize}

\subsection{Secondary metallurgy}
\begin{itemize}
    \item Crude steel further purified and adjusted to final composition in a ladle furnace
    \item Process:
    \begin{itemize}
        \item Deoxidation (chemical or vacuum)
        \item Advanced purification (electroslag remelting (ESR), vacuum arc remelting (VAR), powder metallurgy (PM))
        \item Alloying additions
    \end{itemize}
\end{itemize}

\subsection{Semi-finished products}
\begin{itemize}
    \item Produced mainly by continuous casting (slabs, billets)
    \item Formed into sheets, plates, wires, rods, pipes, and profiles by hot or cold rolling/drawing
\end{itemize}

\subsection{Important alloying elements}
\begin{center}
    \begin{tabular}{|l|l|}
        \hline \textbf{Function} & \textbf{Alloying elements}\\
        \hline Hardenability & Mn, Cr, Ni, Mo, V, Si\\
        \hline Grain refinement & Al, V, Ti, Nb\\
        \hline Corrosion resistance & Cr (>12\%), Cu, Ni, Si, Mn, Mo, N\\
        \hline Wear/heat resistance & Cr (>1\%), Mo, V, Al, Ti, Nb, Mo\\
        \hline Scale resistance & Cr, Al, Si\\
        \hline 
    \end{tabular}
\end{center}

\section{Microstructure formation}
\subsection{Polymorphism of iron}
\begin{center}
    \includegraphics[width=\linewidth]{media/iron_polymorphism.png}
\end{center}

\subsection{Metastable iron-iron carbide phase diagram}
\subsubsection{Phase diagram}
\begin{center}
    \includegraphics[width=\linewidth]{media/phasediagram_explaination.png}
\end{center}

\newcolumn
\begin{center}
    \includegraphics[width=\linewidth]{media/iron-castiron_diagram.png}
\end{center}

\subsection{Microstructure formation of Steel (Slow Cooling)}
\begin{center}
    \scriptsize
    \begin{tabular}{|c|c|l|l|}
        \hline \textbf{Type} & \textbf{\makecell{Composition\\range wt\% C}} & \textbf{\makecell{Resulting\\microstructure}} & \textbf{Properties}\\
        \hline Hypoeutectic & < 4.3 & \makecell[l]{Primary austenite\\+ledeburite} &
        \makecell[l]{toughness $\uparrow$\\brittleness $\downarrow$\\machinability $\uparrow$}\\
        \hline Eutectic & 4.3 & \makecell[l]{Ledeburite (austenite\\+cementite)} &
        \makecell[l]{castability $\uparrow\uparrow$\\hardness $\uparrow$\\ductility $\downarrow$}\\
        \hline Hypereutectic & > 4.3 & \makecell[l]{Primary cementite\\+ledeburite} &
        \makecell[l]{hardness $\uparrow\uparrow$\\wear $\uparrow\uparrow$\\machinability $\downarrow$\\toughness $\downarrow$}\\
        \hline Hypoeutectoid & < 0.8 & \makecell[l]{Primary ferrite\\+pearlite} &
        \makecell[l]{ductility $\uparrow\uparrow$\\weldability $\uparrow\uparrow$\\formability $\uparrow\uparrow$\\hardness $\downarrow$}\\
        \hline Eutectoid & 0.8 & \makecell[l]{Pearlite (ferrite\\+cementite)} &
        \makecell[l]{hardness $\uparrow$\\toughness $\uparrow$\\ductility $\downarrow$}\\
        \hline Hypereutectoid & > 0.8 & \makecell[l]{Primary cementite\\+pearlite} &
        \makecell[l]{hardness $\uparrow\uparrow$\\wear $\uparrow\uparrow$\\toughness $\downarrow$}\\
        \hline
    \end{tabular}
\end{center}

\subsubsection{Eutectoid steel final microstructure at T$_\text{R}$}
The microstructure is 100\% pearlite, consisting of ferrite lamellae
(soft, ductile phase forming the light layers), and cementite lamellae
(hard, brittle phase forming the dark layers).

\subsubsection{Hypereutectoid steel final microstructure at T$_\text{R}$}
Final components are grain boundary cementite (hard and brittle,
formed before eutectoid reaction) and pearlite (lamellar mixture of
ferrite and cementite).

\subsection{Faster cooling / Quenching}
Fast cooling of steels changes the transformation behavior compared
to equilibrium cooling.

\subsubsection{Fast cooling}
\begin{itemize}
    \item Transformation temps shift to lower values
    \item Pearlite become finer and forms over a wider Temp range
    \item Martensite and bainite can form
\end{itemize}

\subsection{Ferrite}
\begin{itemize}
    \item $\alpha$-iron (BCC), stable at low temps
    \item Very low carbon solubility $\rightarrow$ carbon tends to form cementite
    \item Soft and ductile $\rightarrow$ good formability and toughness
\end{itemize}

\subsection{Austenite}
\begin{itemize}
    \item $\gamma$-iron (FCC), stable at higher temps
    \item High carbon solubility $\rightarrow$ carbon dissolves readily
    \item Parent phase for transf. to pearlite, bainite, and martensite
\end{itemize}

\subsection{Pearlite}
\begin{itemize}
    \item Forms from austenite during slow cooling (eutectoid trans.)
    \item Lamellar mixture of ferrite ($\alpha$) and cementite ($\mathrm{Fe_3C}$)
    \item Higher strength and hardness than ferrite, but lower ductility; finer lamellae give higher strength
\end{itemize}

\subsection{Martensite}
\begin{itemize}
    \item Very fast cooling of austenite produces martensite
    \item Diffusion does not occur, so no ferrite or pearlite
    \item Lattice shear of FCC austenite produces BCT
\end{itemize}

\subsection{Bainite}
\begin{itemize}
    \item Produced by a rapid cooling (250-500\cel)
    \item Combined shear and limited diffusion; cementite precipitate
    \item Higher strength than pearlite, more ductile than martensite
    \item Good for high-performance components and tools
\end{itemize}

\subsubsection{Upper bainite}
\begin{itemize}
    \item Forms at higher bainitic T (350-500\cel)
    \item Ferrite needles with cementite lamellae at ferrite boundaries
\end{itemize}

\subsubsection{Lower bainite}
\begin{itemize}
    \item Forms at lower bainitic T (250-350\cel)
    \item Ferrite needles with fine cementite particles inside the ferrite
\end{itemize}

\section{Heat treatments}
\begin{center}
    \includegraphics[width=\linewidth]{media/heat_treatments.png}
\end{center}

\newcolumn
\part{Hardness and Toughness}
\section{Hardness}
It measures resistance against localized plastic deformation.

\subsection{Hardness testing}
\subsubsection{Common testing methods}
\begin{center}
    \footnotesize
    \begin{tabular}{|l|l|}
        \hline \textbf{Testing method} & \textbf{Application}\\
        \hline Vickers (HV) & Universal application\\
        \hline Rockwell (HCR) & For hard steels\\
        \hline Brinell (HB) & For soft steels and aluminum\\
        \hline Berkovich & For nanoindentation\\
        \hline Shore A, D & For rubber and plastic\\
        \hline 
    \end{tabular}
\end{center}

Approximate relation: R$_\text{m} \approx 3\times$ HB or HV

\subsection{Hardness Testing procedures}
\subsubsection{Indentation depth}
Hardness is measured based on how deep the indenter penetrates
the material.

\subsubsection{Indentation area}
Hardness is determined by the surface area of the indentation left
in the material.

\section{Notch Impact Toughness}
\subsection{Impact Notch Toughness test (Charpy)}
\begin{itemize}
    \item Applied mainly to structural steel (BCC)
    \item Shows the transition temp from ductile to brittle fracture
    \item Determines the absorbed impact energy (``notch toughness'')
    \item Failure hypotesis: crack propagation under sudden loads
\end{itemize}

\subsection{Toughness}
\subsubsection{Material behavior}
\begin{itemize}
    \item \textbf{Tough material}: absorb high energy before fracture
    \item \textbf{Brittle material}: fracture with little energy absorbtion
\end{itemize}

\subsubsection{Energy criterion}
\begin{itemize}
    \item \textbf{Ductile fracture}: if absorbes > 27 J of the impact energy
    \item \textbf{Brittle fracture}: if absorbes < 27 J of the impact energy
\end{itemize}




























































































































































































































































































































































































































\end{multicols*}

\end{document}