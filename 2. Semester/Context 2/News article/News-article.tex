\documentclass{article}

\usepackage{Paper}
\pdftitle{MatteoFrongillo_News-article}

\title{\textbf{News reflection}\\\textbf{Context 2}}
\author{Matteo Frongillo}
\date{\today}

\begin{document}

\maketitle
\section*{News exercise}
\subsubsection*{7 wh-questions}
\begin{center}
    \setlength{\extrarowheight}{0.35cm}
    \begin{tabular}{|l|l|}
        \hline
        \textbf{Question} & \textbf{Short answer} \\ \hline
        Who & Services with AI speech-recognition systems \\ \hline
        What & AI frequently misunderstand certain accents \\ \hline
        When & Currently and in the future \\ \hline
        Where & All over the world, especially in countries where English is not an official language \\ \hline
        How & Due to bias in training datasets lacking diverse linguistic inputs \\ \hline
        Why & Because AI are cheaper and companies don't put enough focus on it \\ \hline
        What are the consequences & User misunderstanding, frustration, hard access to critical services and safety risks \\ \hline
    \end{tabular}
\end{center}


\subsubsection*{Main message}
AI-based speech recognition often amplifies inequality by misunderstanding specific
groups, and we must strive for inclusive data practices to prevent harmful outcomes.

\subsubsection*{Headline}
AI speech bias marginalizes users.

\section*{AI generated news}
\textit{\textbf{AI Speech Recognition Bias Raises Concerns}}

Voice-recognition AI doesn’t understand everyone equally well, often making more mistakes with certain groups. Speakers with strong accents or dialects – including non-native English speakers – face more errors. One study found some voice systems misinterpreted Black Americans nearly twice as often as white Americans.

The root cause is bias in training data. AI models learn from huge speech databases, and if most training voices share the same accent or background, the system struggles with others. In critical settings like 911 calls or health hotlines, a misheard symptom or emergency call could cause dangerous delays, raising concerns about equal access to vital services.

\subsubsection*{Prompt}
I would like to ask you if you can write a short news article based on the report I attached.
Make it no longer than 1000 characters with spaces. Be diplomatic,
supporting numbers or thesis with proofs and/or papers. (Attached, the news article)

\pph{AI reply:}
Sure, I can help write a short news article. To make it as precise and well-supported as possible, could you please clarify:

\begin{enumerate}
    \item Do you want the article to target a specific audience or publication style (e.g. general public, academic readers, tech professionals)?
    \item Should it include quotes or references to specific studies mentioned in the original article or similar peer-reviewed sources?
    \item Do you want the tone to be more neutral, critical, or optimistic about AI development?
\end{enumerate}

Once I have this info, I can get started on the article for you.
\\

\textbf{My answer:}
\begin{enumerate}
    \item general public
    \item it should not include references inside the text itself, but it should report all the references at the end
    \item the tone should be as neutral as possible, in order to be more diplomatic as possible
\end{enumerate}

\end{document}