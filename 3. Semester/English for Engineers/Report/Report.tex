\documentclass{article}

\usepackage{Paper}
\pdftitle{Matteo_Frongillo_EENG_Report_Github_Copilot}

% === TITLE ===
\title{\textbf{Github Copilot and the concept of vibe coding \\ English for Engineers, HS25}}
\author{Matteo Frongillo}
\date{}

% === TEXT ===
\begin{document}
\hypersetup{citecolor=black}
\maketitle
\linespread{1.2}\selectfont

% introduction (AI introduction, what it can do, what problems it solves)
IDEs (Integrated Development Environments) are software applications specially designed
for programmers to write and manage code efficiently. With the rise of AI technologies,
IDEs have started to incorporate predictive coding features, aimed at facilitating the
coding process and improving productivity. Vibe coding refers to the practice of
coding in a relaxed and creative manner, relying on AI predictive autocompletion
suggestions, removing the need of a structured approach before starting to code.
This report explores the concept of vibe coding on Visual Studio Code, a popular IDE,
which has, as default, GitHub Copilot as integrated AI-powered coding assistant \parencite{vscode-overview}.
\vspace*{.25cm}

Studying university subjects efficiently requires optimal time management and workload
distribution. In order to achieve this, paraphrasing and summarizing the study material is
essential, and this is where GitHub Copilot comes into play. As a Visual Studio Code
user, taking notes in LaTeX, which is a typesetting system widely used for technical and
scientific documents \parencite{latex}, saves me time and effort compared to traditional
pen-and-paper methods. GitHub Copilot facilitates my note-taking process by analyzing the
content of the code and predicting the next words that will be most likely written,
with a particular emphasis on accuracy in mathematical and physical formulas.
\vspace*{.25cm}

% potential applications
Vibe coding has the potential to revolutionize the way we approach programming tasks.
By leveraging AI-driven suggestions, developers can focus on the creative aspects of coding,
allowing for a more fluid and intuitive workflow. This approach not only enhances
productivity but also encourages experimentation and exploration of new ideas.
GitHub Copilot, in particular, has demonstrated to be an effective and smart tool for
programmers, as the accuracy of its suggestions and the capability to understand context
improve over time with usage \parencite{vscode-inline}.
\vspace*{.25cm}

% challenges
Despite the numerous benefits of GitHub Copilot, one of the main challenges associated
with vibe coding is the risk of over-reliance on AI-generated suggestions. This can
lead to a lack of problem-solving skills and a partial understanding of the code being
written. Personally, I believe that AI is useful as a writing assistant, but when it
comes to studying and learning, it is essential that I actively write and understand the
material, rather than passively accepting AI-generated content.
\vspace*{.25cm}

% comparison/contrasts
GitHub Copilot gives the opportunity to choose between different AI models for relying
on predictive coding. Among these, ChatGPT-based models stand out for their ability to
understand tasks better, while Anthropic's Claude models are known for performing well
in coding tasks \parencite{github-models}.
\vspace*{.25cm}

% future prospects
Promising advancements in AI technology suggest that integrated AI assistants like
GitHub Copilot will be increasingly adopted in IDEs and browsers. Specifically for
Copilot, new AI models and respective features are expected to be added in the future
\parencite{github-discussion}.
\vspace*{.25cm}

% recommendations
Both creativity and critical thinking should always be prioritized while vibe coding.
While GitHub Copilot enhances productivity and the workflow, it is crucial to keep
an eye out for potential misalignments in the suggestions provided by the AI.
Over reliance on Copilot should be considered carefully, especially when IDEs are used
for working on complex tasks and important projects \parencite{github-guide}.

% conclusion
In conclusion, vibe coding with GitHub Copilot offers a 

\vfill
\setlength{\bibitemsep}{1.2\baselineskip}
\printbibliography[title={References}]

\section*{Declarations about AI tools}
\begin{itemize}
    \item ``GitHub Copilot'' was used to generate words suggestions.\\
        \url{https://github.com/features/copilot}
    \item ``DeepL'' was used as a translator.\\
        \url{https://www.deepl.com}
\end{itemize}




\end{document}
