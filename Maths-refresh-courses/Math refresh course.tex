\documentclass{article}

\usepackage[fleqn]{amsmath}
\usepackage{amssymb}
\usepackage{hyperref}
\usepackage{url}
\usepackage{graphicx}
\usepackage{geometry}
\usepackage{babel}
\usepackage{enumitem}
\usepackage{parskip}
\usepackage{chemfig}
\usepackage{pdfpages}
\usepackage{xcolor}
\usepackage{tikz}
\usepackage{fancybox}
\usepackage{makecell}
\usepackage{pgfplots}
\usepackage{soul}
\usepackage{ulem}
\usepackage{wrapfig}
\usepackage{subcaption}
\usepackage[T1]{fontenc}
\usetikzlibrary{decorations.pathreplacing}
\pgfplotsset{compat=1.17}

\geometry{
    a4paper,
    total={170mm, 257mm},
    left=20mm,
    top=20mm
}

\hypersetup{
    colorlinks=true,
    linkcolor=black,
    urlcolor=blue,
    pdftitle={Math refreshing course}
}

\newcommand{\figbox}[1]{ 
    \begin{figure*}[ht!]        
        \begin{center}            
            \fbox{#1}        
        \end{center}    
    \end{figure*}
}

% === TEXT ===
\title{\textbf{Maths refreshing course \\ HSLU, Semester 1}}
\author{Matteo Frongillo}

\begin{document}

\maketitle
\tableofcontents
\pagebreak

\part{Lesson 1}

\section{Algebraic definitions}
\begin{itemize}
    \item $\mathbb{N} := \text{Natural numbers (including 0)}$
    \item $\mathbb{Z} := \text{Integer numbers}$
    \item $\mathbb{Q} := \text{Rational numbers}$
    \item $\mathbb{R} := \text{Real numbers}$
\end{itemize}

We have that: 
\[
    \mathbb{N} \subset  \mathbb{Z} \subset \mathbb{Q} \subset \mathbb{R} \subset \mathbb{C} 
\]

\section{Prime numbers}
A prime number is a natural number greater than 1 that has no positive
divisors other than 1 and itself.
\figbox{$n \in \mathbb{N},\ n \neq \{0, 1\}$}

\section{Positive powers}
Let $a \in \mathbb{R}, n \in \mathbb{R}^*$ and ${a} \subset \mathbb{R}$, then

\figbox{$
    a^{1} := a \quad | \quad
    a^n = \underbrace{a \cdot a \cdot ... \cdot a}_{n \text{ times}}$
}

\subsection{Property 1}
Let $a, b \in \mathbb{R},\ n,m \in \mathbb{N}$, then \\
\figbox{$a^n \cdot a^m = a^{n+m}$}

\subsection{Property 2}
Let $a,b \in \mathbb{R},\ n \in \mathbb{N}$, then \\
\figbox{$(a \cdot b)^n = a^n \cdot b^n$}

\underline{Notation}: The power $a^n$, $a$ is the base and $n$ is the exponent.

\subsection{Property 3}
Let $a \in \mathbb{R},\ m,n \in \mathbb{N}^*$, then \\
\figbox{$(a^n)^m = a^{n \cdot m}$, which is $\neq a^{(n^m)}$}

\newpage
\section{Fractions}
\underline{Notation 1}: $a \cdot b = a \times b = ab$ \quad | \quad $\frac{a}{b} = a \div b = a : b$

\underline{Notation 2}: ``$a$'' is called numerator, ``$b$'' is called denominator.

\underline{Notation 3}: $\frac{a}{b},\ a,b \in \mathbb{R},\ b \neq 0$

\subsection{Property 1}
Let $a, b \in \mathbb{R}^*$ and $c, d \in \mathbb{R}$, then\\
\figbox{\large $\frac{a}{b} \cdot \frac{c}{d} = \frac{a \cdot c}{b \cdot d}$}

\subsection{Property 2}
Let $a, b \in \mathbb{R}^*$ and $c, d \in \mathbb{R}$, then\\
\figbox{\large $\frac{a}{b} \div \frac{c}{d} = \frac{a}{b} \cdot \frac{d}{c}$}

\subsection{Property 3}
Let $a, b \in \mathbb{R}^*$ and $c, d \in \mathbb{R}$, then\\
\figbox{\large $\frac{a}{b} \pm \frac{c}{d} = \frac{a \cdot d \pm c \cdot b}{b \cdot d}$}

\section{Negative powers}
\subsection{Definition}
\figbox{$\forall a \in \mathbb{R},\ a \neq 0$; \; $a^{-1} := \frac{1}{a} $}

\subsection{Property 4}
Let $\forall n \in \mathbb{N},\ \forall a \in \mathbb{R}$, then\\
\figbox{$a^{-n} = \left(\frac{1}{a}\right)^n$}

This property implies that $\forall z \in \mathbb{Z},\ \forall a \in \mathbb{R},\ z \neq 0$\\
We can compute $a^z$

\subsection{Property 5}
Let $\forall a \in \mathbb{R},\ a \neq 0,\ \forall n,m \in \mathbb{Z}$, then\\
\figbox{$\frac{a^n}{a^m} = a^{n-m}$}

\newpage
\underline{Consequences}:
\begin{enumerate}
    \item Properties 1, 2 and 3 also hold for integer exponents:
        \begin{itemize}
            \item $\forall a \in \mathbb{R},\ \forall n,m \in \mathbb{Z} \Rightarrow a^n \cdot a^m = a^{n+m}$
            \item $\forall b \in \mathbb{R},\ (a \cdot b)^n = a^n \cdot b^n$
            \item $(a^n)^m = a^{n \cdot m}$
        \end{itemize}  
    \item $\forall a \in \mathbb{R}^*,\ a^0 = a^{1-1} = \frac{a^1}{a^1} = 1 \Rightarrow a^0 = 1$
\end{enumerate}

\section{Fractions and percentages (and back)}
$\alpha \in \mathbb{R},\ n \% \text{ of } \alpha \Longleftrightarrow \frac{n}{100} \cdot \alpha$

\newpage
\part{Lesson 2}










\end{document}
