\documentclass{article}

\usepackage{fancyhdr}
\usepackage{setspace}
\usepackage{titlesec}
\usepackage{tocloft}
\usepackage{lipsum}
\usepackage[fleqn]{amsmath}
\usepackage{amssymb}
\usepackage{hyperref}
\usepackage{url}
\usepackage{graphicx}
\usepackage{geometry}
\usepackage{enumitem}
\usepackage{parskip}
\usepackage{chemfig}
\usepackage{pdfpages}
\usepackage{tikz}
\usepackage{fancybox}
\usepackage{makecell}
\usepackage{pgfplots}
\usepackage{soul}
\usepackage{ulem}
\usepackage{wrapfig}
\usepackage{subcaption}
\usepackage[T1]{fontenc}
\usepackage{esvect}
\usepackage{xcolor}
\usepackage[utf8]{inputenc}
\usepackage{csquotes}
\usepackage[style=apa, backend=biber]{biblatex}
\addbibresource{refs.bib}
\usetikzlibrary{arrows}
\usetikzlibrary{decorations.pathreplacing}
\pgfplotsset{compat=1.17}

\geometry{
    a4paper,
    total={170mm, 257mm},   
    left=20mm,
    top=20mm
}   

\hypersetup{
    colorlinks=true,
    linkcolor=black,
    urlcolor=blue,
    pdftitle={Team 31_EAP Semester Performance 1 HS24}
}

\pagestyle{fancy}
\fancyhf{}
\fancyhead[L]{\leftmark}
\fancyhead[R]{\thepage}
\renewcommand{\headrulewidth}{0.4pt}
\newcommand{\figbox}[1]{ 
    \begin{figure*}[ht!]        
        \begin{center}            
            \fbox{#1}        
        \end{center}    
    \end{figure*}
}

\newcommand{\wrapfill}{
    \par
    \ifnum \value{WF@wrappedlines} > 0
        \addtocounter{WF@wrappedlines}{-1}%
        \null\vspace{
            \arabic{WF@wrappedlines}
            \baselineskip
        }
        \WFclear
    \fi
    \phantom{}
}

\newcommand{\difference}{\,\backslash\,}
\newcommand{\rem}{\underline{Remark}: }
\newcommand{\nots}{\underline{Notation}: }
\newcommand{\prf}{\underline{Proof}: }
\newcommand{\exs}{\underline{Example}: }
\newcommand{\defs}{\underline{Definition}: }
\newcommand{\wrn}{\underline{Warning}: }
\newcommand{\sht}{\ |\ }
\newcommand{\pph}[1]{\paragraph{#1} \phantom{}\\}
\newcommand{\dm}{\displaystyle}
\newcommand{\rad}{^{\mathrm{c}}}

\titleformat{\chapter}[block]{\bfseries\Large}{\thechapter.}{1em}{}

%========= TEXT ===========

\begin{document}

\hypersetup{citecolor=black}

\begin{titlepage}
    \begin{flushleft}
        \hspace*{.01cm}
        \includegraphics[width=.3\textwidth]{media/hslu-logo-2.png}

        \vspace*{.01cm}
        \hspace*{.16cm}\includegraphics[width=0.4\textwidth]{media/hslu-svg-logo.png}
    \end{flushleft}

    \vspace*{-.4cm}
    {\huge \textbf{EAP Semester Performance 1}}

    {\Large Herbstsemester 2024}

    \vspace*{3cm}
    \begin{center}
        {\Huge \textbf{My$_{\text{celium}}$Tent}}

        \vspace*{.1cm}
        \textbf{\large The biodegradable tent}

%        \vspace*{.5cm}
%        \includegraphics*[width=.35\textwidth]{media/mytent_logo.png}
    \end{center}

    \vfill
    {\Large \textbf{Team 31}}\\
    {\large \vspace*{.01cm}
        Althaus Simon\\
        \vspace*{.01cm}
        Berner Nic\\
        \vspace*{.01cm}   
        Frongillo Matteo\\
        \vspace*{.01cm}
        McCarthy Benjamin\\
        \vspace*{.01cm}
        Nyamdorj Narandavaa\\
        \vspace*{.01cm}
    }
    
    \vspace{1cm}
    {\large Horw, 14th October 2024}
\end{titlepage}

\tableofcontents
\thispagestyle{empty}

\newpage
\section{Introduction}
The growing concern over environmental sustainability and waste reduction has driven
innovation in the outdoor gear industry. Disposable camping tents, particularly those used
at large-scale events like music festivals, contribute significantly to waste accumulation.
To address this issue, our project focuses on developing a biodegradable tent, named
My$_{\text{celium}}$Tent, which leverages mycelium-based composites as a sustainable
alternative to traditional nylon and plastic fabrics. Mycelium, the root structure of
fungi, offers unique properties such as biodegradability, fire resistance, and minimal
environmental impact. This project aims to create an eco-friendly solution that not only
meets safety and durability standards but also contributes to reducing post-event waste,
aligning with the global shift toward more sustainable and environmentally conscious
products.

\section{Theory chapter}
\subsection{Mycelium}
{\small Research by Berner Nic}

The development of the MyceliumTent is focused on replacing existing
nylon and plastic tent fabrics with a mycelium-based fabric in order
to make it biodegradable.

\subsubsection{Definition}
Mycelium is the underground root network created by a mushroom
organism. Fungi nourish themselves by secreting digestive enzymes to
break down organic material in their surroundings and absorb it
through the cell walls of the hyphae, their root network
(Moore, Ahmadjian, \& Alexopoulos, 2024). Various types of fungi
produce mycelium. For this project, the focus is on a fungus with a
high growth rate, suitable for the production of biocomposites. In
this case, it is possible to use the oyster mushroom
(Nicolcioiu, Popa, \& Matei, 2016), (\cite{alexopoulos2024}).

\subsubsection{Potential of mycelium}
The potential of this material lies in its low carbon footprint, low
energy and processing cost, and biodegradability
(Alaneme et al., 2023, pp. 234--250). The most common use cases in the
industry so far include leather, packaging materials, or composites
used for construction. However, challenges remain, such as the lack
of standardized treatment methods during material development. This
project specifically explores biodegradability, breathability,
durability, water, and fire resistance, as needed to construct a tent
fabric. (\cite{alaneme2023})

\subsection{Environment}
{\small Research by McCarthy Benjamin}

This chapter explores different properties of mycelium bio-composites
(MBC). MBCs are ``composed of an agricultural residue, a non-living
material, colonized by a fungus'' (Amziane, Merta, \& Page, 2023, p. 740).
As of today, the full potential of MBCs has not been found, and
different production processes and growth combinations
(types of fungi and substrates) continue to be tested and compared.

\subsubsection{Mycelium bio-composites}

Some important properties for this product are sound absorption,
thermal conductivity, and moisture buffering value. These properties
vary greatly depending on the substrate-fungus combination. Research
on fifty unprocessed MBCs found that the sound absorption coefficient
differs from 0.5 to 0.95 depending on the frequency, indicating they
are good sound absorbers (Amziane et al., 2023, S. 749). The thermal
conductivity value was found to be between 0.057 -- 0.085
$\frac{W}{m\cdot K}$, and the mean moisture buffering value was
1.632 (Amziane et al., 2023, p. 749). However, no clear values of
water resistance were found, although it is possible to coat the MBC
in biodegradable polyurethanes or beeswax for extra water resistance.
(\cite{amziane2023})

\subsubsection{Mycelium-based leather}

One of the MBC products in use today is mycelium-based leather (MBL).
Research has shown that the order of polyporales fungi, specifically
Fomitella fraxinea, in combination with a substrate made of sawdust
and rice bran, is best suited for the production of MBL
(Raman, Kim, Kim, Oh, \& Shin, 2022). After harvesting, the
composites are plasticized (with a biodegradable mixture) and
hot-pressed, forming a leather-like material. These processes
increase tensile strength, elongation percentage, and reduce water
absorption. MBL has a mean tensile strength of 8.49 MPa, can elongate
up to 58.86\%, and has a water contact angle of up to 129.63$^{\circ}$,
making it hydrophobic (Raman et al., 2022). (\cite{env3})

\subsubsection{Biodegradability}

Depending on the fungi and substrate used, the biodegradation
duration can vary. For example, using Pleurotus ostreatus on a
bamboo-based substrate coated with beeswax for increased water
resistance shows a mass reduction of 64.13\% after two months
(Gan et al., 2022). However, due to insufficient research on the
biodegradability of other MBCs, no definitive information can be
provided on the biodegradability of class-sharing fungi like
polyporales. (\cite{env2})

\subsection{Structure and Setup}
{\small Research by Althaus Simon}

There are several construction options available for tents. To determine the most suitable
option for our purposes, this section evaluates the advantages and disadvantages of the
most common tent designs. Hilleberg, a manufacturer of high-quality expedition tents,
prioritizes ease of use, which aligns with the criteria important for our product.
Therefore, their approach to tent construction serves as a relevant reference for this
analysis. Hilleberg categorizes its tents by label, with the Yellow Label being the most
applicable to our needs, as these tents are designed for use during snow-free months and
in protected environments, or in warmer climates. (\cite{setup1})

The two most common constructions are Tunnel Tents and Dome Tents. 
Tunnel tents provide the best space-to-weight ratio, making them ideal for mobile journeys
where the tent is frequently set up and taken down. Their lighter overall design is
advantageous for users who carry their gear during the day. However, tunnel tents are less
stable in windy or snowy conditions and typically require pegging to remain upright.

In contrast, dome tents offer greater stability, particularly in adverse weather conditions
such as snow or high winds. They are better suited for base camp setups, where they can
remain stationary for extended periods. Some dome tents are freestanding, which eliminates
the need for pegging and makes them useful in terrains like rocky or gravelly soil. Despite
these advantages, dome tents tend to be heavier and provide less space for the weight they
add. (\cite{setup2})

In addition to tunnel and dome tents, instant tents represent another construction option
that has gained popularity in recent years due to improvements in ease of setup. These
tents combine features of both tunnel and dome designs, offering a balance between
spaciousness and stability. Instant tents are particularly advantageous for users seeking
quick and simple setup, often requiring minimal effort. However, they are generally not
designed to withstand harsh environmental conditions such as strong winds or heavy snow,
making them more suitable for mild weather and less extreme environments.
(\cite{outdoorlife2024})

\subsection{Marketing}
{\small Research by Nyamdorj Narandavaa}

The analysis of the marketing of camping tents and outdoor equipment is based on the use
of strategies leveraged to reach and engage consumers globally, in Europe, and Switzerland.
Nowadays, the use of social media is a key focus for product promotion and customer loyalty
through targeted advertising and influencers.

\subsubsection{Worldwide Market}

In 2022, the global camping tent market reached a total value of USD 2.65 billion. The
future of this market is positive; in fact, it is set to grow further to USD 4 billion by
2028. This positive outlook is possible due to the increase in outdoor recreation and
nature tourism. (\cite{expert2023})

\subsubsection{European Market}

According to reported projections, by 2029 the European camping tent market will grow
significantly, reaching a value of USD 1.50 billion. The analysis covers various product
categories, materials, and capacities, highlighting the increasing demand for innovative
and practical solutions in line with new camping trends. (\cite{arizton2024})

\subsubsection{Swiss Market}

The analysis of the camping tent market in Switzerland focuses on growth trends driven by
increased outdoor activities and investment in sustainable materials. Changes in consumer
preferences are also part of the analysis. (\cite{swiss2023})

\subsubsection{Marketing Strategies}

Key marketing strategies for outdoor brands are analyzed based on their impact on their
audiences. In fact, content posted on social media is strongly considered, facilitating
visibility through influencers, who play a key role in product promotion and customer
loyalty. (\cite{mitech2024})

\subsection{Safety}
{\small Research by Frongillo Matteo}

In the development of a disposable tent, ensuring safety is paramount, particularly for
use in crowded and temporary environments such as music festivals. The incorporation of
sustainable materials like mycelium necessitates adherence to safety standards, focusing
on both mechanical performance and fire protection. These aspects are crucial to safeguard
users, mitigate fire-related risks, and offer an eco-friendly solution that minimizes
post-use waste.

\subsubsection{Fire resistance of mycelium}

Mycelium-based composites offer significant advantages in terms of fire resistance, making
them well-suited for mass gatherings where the fire hazard is elevated. Compared to
synthetic alternatives, mycelium produces less smoke and fewer toxic emissions when
exposed to flames, reducing health risks during emergencies. Additionally, the inclusion
of silica-rich substrates, such as rice hulls, enhances its fire-retardant properties,
ensuring greater protection in densely populated festival settings. (\cite{fire})

\subsubsection{Environmental impact and safety}

A major benefit of mycelium in the context of this patent is its minimal environmental
footprint, both during use and after disposal. Unlike conventional synthetic materials,
mycelium composites release negligible amounts of CO2 when combusted. This is especially
important for products designed for temporary outdoor use, as it helps limit air pollution
and reduces the risk of toxic exposure during unexpected incidents. Moreover, the
biodegradable nature of mycelium resolves the issue of waste accumulation post-festival.
(\cite{myc-environment})

\subsubsection{Mechanical safety properties}

Mycelium’s mechanical strength offers another advantage, particularly in ensuring
structural stability under varying environmental conditions typical of festivals. The
compressive strength and resilience provided by fungal fibers ensure that the tent remains
secure and functional throughout its use, offering both physical protection and fire
resistance. (\cite{myc-mechanical})

\subsubsection{Safety in camping applications}

For the patent, it is essential to meet established safety regulations for camping
shelters, such as the DIN EN ISO 5912:2020 standard, which governs both mechanical and
fire safety requirements. The use of mycelium not only satisfies these standards but also
adds the benefit of biodegradability, making it an ideal choice for eco-conscious outdoor
events. After its lifecycle, the tent naturally decomposes, significantly reducing
environmental impact and festival-generated waste. (\cite{tent-safety})

\newpage
\section{References}
\setlength{\bibitemsep}{1.2\baselineskip}
\printbibliography[heading=none]
\vfill

\section{Declarations}
\begin{itemize}
    \item \textit{DeepL} and \textit{ChatGPT 4o} have been used as a spell-checker;\\
        \url{https://www.deepl.com/}
    \item \textit{ChatGPT 4o} and \textit{ChatGPT o1-preview} have been used as the APA 7 citation corrector.\\
        \url{https://chatgpt.com/}
\end{itemize}

\end{document}
