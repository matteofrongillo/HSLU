\documentclass{article}

\usepackage[fleqn]{amsmath}
\usepackage{amssymb}
\usepackage{hyperref}
\usepackage{url}
\usepackage{graphicx}
\usepackage{geometry}
\usepackage{babel}
\usepackage{enumitem}
\usepackage{parskip}
\usepackage{chemfig}
\usepackage{pdfpages}
\usepackage{xcolor}
\usepackage{tikz}
\usepackage{fancybox}
\usepackage{makecell}
\usepackage{pgfplots}
\usepackage{soul}
\usepackage{ulem}
\usepackage{wrapfig}
\usepackage{subcaption}
\usetikzlibrary{decorations.pathreplacing}
\pgfplotsset{compat=1.17}

\usepackage{frongillo}
\usepackage{stellar}

\geometry{
    a4paper,
    total={170mm, 257mm},
    left=20mm,
    top=20mm
}

\hypersetup{
    colorlinks=true,
    linkcolor=black,
    urlcolor=blue,
    pdftitle={Math refreshing course, lession 1}
}

% === TEXT ===
\title{\textbf{Maths refreshing course \\ HSLU, Semester 1}}
\author{Matteo Frongillo}

\begin{document}

\maketitle
\tableofcontents
\pagebreak

\part{Lession 1}

\section{Algebric definitions}
\begin{itemize}
    \item $\mathbb{N} := \text{Natural numbers}$
    \item $\mathbb{Z} := \text{Integral numbers}$
    \item $\mathbb{Q} := \text{Rational numbers}$
    \item $\mathbb{R} := \text{Real numbers}$
\end{itemize}

We have that: 
\[
    \mathbb{N} \subset  \mathbb{Z} \subset \mathbb{Q} \subset \mathbb{R} \subset \mathbb{C} 
\]

\section{Prime numbers}
A prime number is a natural number which can be
devided only by itself or 1\\
\figbox{$n \in \mathbb{N},\ n \neq {0, 1}$}

\section{Positive powers}
Let $a \in \mathbb{R}, n \it \mathbb{R}, n \neq 0$
and ${a} \subset \mathbb{R}$

\[
    3¹ := 3 \\
    3² := 3 \cdot 3 \\
    3^{23} := 3 \cdot 3 \cdot ... \cdot 3, \text{23 times}
\]

\subsection{Property 1}
Let $a, b \in \mathbb{R},\ n,m \in \mathbb{N}$, then \\
\figbox{$a^n \cdot a^m = a^{n+m}$}

\subsection{Property 2}
Let $a,b \in \mathbb{R},\ n \in \mathbb{N}$, then \\
\figbox{$(a \cdot b)^n = a^n \cdot b^n$}

\underline{Notaton}: The power $a^n$, $a$ is the base nad $n$ is the
    exponent.

\subsection{Property 3}
Let $a \in \mathbb{R},\ m,n \in \mathbb{N^*}$, then \\
\figbox{$(a^n)^m = a^{n \cdot m}$, which is $\neq a^{(n^m)}$}

\newpage
\section{Fractions}
\underline{Notation 1}: $a \cdot b = a \times b = ab$;\quad $\frac{a}{b} = a \div b = a : b$

\underline{Notation 2}: $a$ is called numerator, $b$ is called denominator.

\underline{Notation 3}: $\frac{a}{b},\ a,b \in \mathbb{R},\ b \neq 0$

\subsection{Property 1}
Let $a, b, c ,d \in \mathbb{R},\ a,b \neq 0$
\figbox{\large $\frac{a}{b} \cdot \frac{c}{d} = \frac{a \cdot c}{b \cdot d}$}

\subsection{Property 2}
Let $a, b, c ,d \in \mathbb{R},\ a,b \neq 0$
\figbox{\large $\frac{a}{b} \div \frac{c}{d} = \frac{a}{b} \cdot \frac{d}{c}$}

\subsection{Property 3}
Let $a, b, c ,d \in \mathbb{R},\ a,b \neq 0$
\figbox{\large $\frac{a}{b} \pm \frac{c}{d} = \frac{a \cdot b\ \pm\ c \cdot b}{b \cdot d}$}

\section{Negative powers}
\subsection{Definition}
\figbox{$\forall a \in \mathbb{R},\ a \neq 0$; \; $a^{-1} := \frac{1}{a} $}

\subsection{Property 1}


\subsection{Property 2}
















\end{document}