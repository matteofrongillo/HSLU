\documentclass{article}

\usepackage{Mathematics}
\pdftitle{TUT-MATH1A-evaluation}

% === TITLE ===
\title{\textbf{MATH1A Tutoring evaluation \\ TUT, HS25}}
\author{Matteo Frongillo}
\date{}

% === TEXT ===
\begin{document}
\maketitle

\section{Feedback}
I have created a Microsoft Forms survey to collect feedback from the students attending
my MATH1A tutoring sessions. The survey was anonymous with the option to leave their name.
I have received 9 responses in total.

\section{Responses overview}
\subsection{Your name}
-

\subsection{What is your general impression about the MATH1A Tutoring sessions?}
A scale from 1 to 10, where 1 is ``extremely disappointed'' and 10 is ``extremely satisfied''
was used:
\begin{figure}[ht!]
    \centering
    \includegraphics[width=.8\textwidth]{media/Q2.png}
\end{figure}

\subsection{How often did you go to the Tutoring sessions?}
\begin{figure}[ht!]
    \centering
    \includegraphics[width=.8\textwidth]{media/Q3.png}
\end{figure}

\newpage
\subsection{What motivated you to attend the tutoring sessions?}
\begin{itemize}
    \item ``better understanding of class and exercise problems''
    \item ``A lot of exercises and explanation given''
    \item ``Because I know that Maths is not one of my strong suites.''
    \item ``I wanted to get additional guidance through the exercises and homework.''
    \item ``Revising and getting immediate feedback. Being able to ask personal question and receiving an immediate response.''
    \item ``If I’m struggling with a specific topic, I’ll attend the tutoring sessions.''
    \item ``How simple Matteo explains the problems and breaks them down in order for us to understand.''
    \item ``Learning different methods with the tutor to resolve the problems.''
    \item ``Personal knowledge of the tutor and skill in explaining critical topics.''
\end{itemize}

\subsection{How well did the tutorials meet your expectations?}
A scale from 1 to 5, where 1 is ``not met at all'' and 5 is ``fully met'' was used:
\begin{figure}[ht!]
    \centering
    \includegraphics[width=.8\textwidth]{media/Q5.png}
\end{figure}

\subsection{Would you use the tutorials again?}
\begin{figure}[ht!]
    \centering
    \includegraphics[width=.8\textwidth]{media/Q6.png}
\end{figure}

\subsection{About question 6 - If no, why?}
No answer was given for this question.

\newpage
\subsection{How much did you benefit from the tutorial...}
Possible responses were: Very much, Quite a lot, Moderately, Only a little.

\begin{table}[ht!]
    \centering
    \renewcommand{\arraystretch}{1.5} 
    \begin{tabular}{|c|l|l|l|l|l|}
    \hline
        ID& \makecell[l]{Understanding\\of key concepts\\and tasks} & \makecell[l]{Ability to solve\\tasks and work\\efficiently} & \makecell[l]{Motivation for\\the module and\\its assignments} & \makecell[l]{Self-confidence in\\dealing with course\\tasks} & \makecell[l]{General work habits\\and/or\\learning strategies} \\ \hline
        1 & Quite a lot & Very much & Moderately & Quite a lot & Quite a lot \\ \hline
        2 & Quite a lot & Quite a lot & Moderately & Very much & Quite a lot \\ \hline
        3 & Quite a lot & Moderately & Very much & Quite a lot & Quite a lot \\ \hline
        4 & Very much & Quite a lot & Quite a lot & Quite a lot & Moderately \\ \hline
        5 & Quite a lot & Quite a lot & Moderately & Very much & Quite a lot \\ \hline
        6 & Quite a lot & Quite a lot & Quite a lot & Quite a lot & Quite a lot \\ \hline
        7 & Quite a lot & Quite a lot & Quite a lot & Quite a lot & Quite a lot \\ \hline
        8 & Very much & Very much & Very much & Very much & Very much \\ \hline
        9 & Quite a lot & Quite a lot & Quite a lot & Quite a lot & Quite a lot \\ \hline
    \end{tabular}
\end{table}

\subsection{The tutor...}
Possible responses were: Fully true, Mostly true, Partly true, Not true, N.A.

\begin{table}[ht!]
    \centering
    \renewcommand{\arraystretch}{1.5} 
    \begin{tabular}{|c|l|l|l|l|l|}
    \hline
        ID & \makecell[l]{Explains complex\\issues clearly} & \makecell[l]{Addresses questions\\and objections\\carefully} & \makecell[l]{Is well prepared\\for the meetings} & \makecell[l]{Treats me/us\\with kindness\\and respect} & \makecell[l]{Coordinates well\\with the lecturers\\and/or other tutors\\of the course} \\ \hline
        1 & Mostly true & Fully true & Mostly true & Fully true & Fully true \\ \hline
        2 & Mostly true & Mostly true & Fully true & Fully true & Fully true \\ \hline
        3 & Mostly true & Mostly true & Mostly true & Fully true & Fully true \\ \hline
        4 & Fully true & Fully true & Fully true & Fully true & Fully true \\ \hline
        5 & Mostly true & Fully true & Fully true & Fully true & Fully true \\ \hline
        6 & Fully true & Fully true & Fully true & Fully true & Fully true \\ \hline
        7 & Fully true & Fully true & Fully true & Fully true & Fully true \\ \hline
        8 & Fully true & Fully true & Fully true & Fully true & Fully true \\ \hline
        9 & Fully true & Mostly true & Fully true & Fully true & Mostly true \\ \hline
    \end{tabular}
\end{table}

\newpage
\subsection{About yourself}
Possible responses were: Always, Mostly, Seldom, Never.

\begin{table}[ht!]
    \centering
    \renewcommand{\arraystretch}{1.5} 
    \begin{tabular}{|c|l|l|l|l|l|}
    \hline
        ID & \makecell[l]{I attend the\\tutorial on time} & \makecell[l]{I pay attention\\and participate\\actively} & \makecell[l]{I work or study\\together with\\colleagues} & \makecell[l]{I prefer asking\\tutors rather\\than lecturers} & \makecell[l]{I complete all\\recommended tasks\\for this module} \\ \hline
        1 & Mostly & Mostly & Seldom & Always & Mostly \\ \hline
        2 & Always & Mostly & Seldom & Seldom & Mostly \\ \hline
        3 & Always & Mostly & Seldom & Seldom & Mostly \\ \hline
        4 & Always & Mostly & Mostly & Seldom & Mostly \\ \hline
        5 & Always & Mostly & Seldom & Always & Always \\ \hline
        6 & Seldom & Seldom & Always & Mostly & Never \\ \hline
        7 & Mostly & Mostly & Always & Seldom & Always \\ \hline
        8 & Mostly & Always & Always & Seldom & Always \\ \hline
        9 & Mostly & Seldom & Seldom & Seldom & Mostly \\ \hline
    \end{tabular}
\end{table}

\subsection{Your suggestions for improving the tutorials}
Two answers were given:
\begin{itemize}
    \item ``For me the tutorials are quite good as they are''
    \item ``If the maths script excerises were in the correct order with the lectures, the flow would be better becuase then there is no confusion as to what was done in class and what not. Saves the tutor time becuase he doesnt have to filter through all the exercises to find the ones to the relevant topic. Tutoring has been very helpful and the tutor is very organised and put lots of time and effort into creating new exercise sheets for our class which were very helpful.''
\end{itemize}

\end{document}
