\documentclass{article}

\usepackage{Paper}
\pdftitle{MatteoFrongillo_Feedback1}

\title{\textbf{Reflections presentation 1}\\\textbf{Context 1, F2502}}
\author{Matteo Frongillo}
\date{\today}

\begin{document}

\maketitle
\pph{Overall}
I believe that with my presentation I was able to successfully
introduce the concept of Building-Integrated Photovoltaics (BIPV) to
a general audience unfamiliar to the world of photovoltaics. I am
satisfied with how I explained BIPV and its use, emphasizing its
difference between traditional photovoltaic modules (BAPVs), and I
think that not including the physical and chemical workings of this
technology was the correct move to maintain the public's interest. I
think the inclusion of real case studies, including some from
Switzerland, helped illustrate both the practical and aesthetic
potential of BIPV technology. I also noticed interested and amazed
faces during the presentation of real data from a case study, which I
think was the highlight of the presentation.

\pph{Introduction}
Regarding the introduction, reading the points in the table of
contents was not the most engaging approach. This made the
introduction feel more mechanical than inviting. In the future,
I will focus better on creating a more effective and
attention-grabbing introduction, perhaps by asking a direct question
or presenting an unusual fact that can stimulate the audience's
interest from the very beginning, tying it directly with the table of
contents, so as to avoid reading it. An interesting sentence might
have been:

``Energy costs are rising, yet some people manage to save
money by producing almost 50 percent self-consumption of energy with
their homes. Here's what their secret is:''

\pph{Body}
In the body of the presentation, I provided an overview of BIPV
systems, covering the different possible integrations and its
advantages, with lots of real analogous data. I think including
actual pictures of the case studies increased audience
interest, and the choice of the all different
projects (residential building, start-up, university facility, town
square) showed how flexible the use of this PV technology is. The only
improvement I think needs to be made is to cite the sources properly,
citing the photographer or company responsible instead of the website
from which I took the image.

\pph{Conclusion}
The conclusion was a weak point, as I did not include a clear summary
that related back to the introduction or the body of the presentation.
Concluding with “that's it” followed by a brief message encouraging the
audience to look forward to integrated PV in the future was not an
effective ending, making it abrupt and incomplete. This is another
aspect that I will consider improving for the future, adding a
concluding thought or call to action for those listening.

\pph{Body language}
I received useful feedback on my movement.
I tended to move excessively due to nervousness. I will definitely
work on maintaining a more stable posture, keeping one foot firmly
still and moving only when necessary, trying to have my arms
looser and freer. Also, I was told that the noise of the keys I had
on me was annoying, so I will definitely leave them in my backpack
during future presentations. On the positive side, I believe that I made
good eye contact with the audience and managed to convey a good
impression with my facial expressions.

\end{document}