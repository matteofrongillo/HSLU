\documentclass{article}

\usepackage{fancyhdr}
\usepackage{setspace}
\usepackage{titlesec}
\usepackage{tocloft}
\usepackage{lipsum}
\usepackage[fleqn]{amsmath}
\usepackage{amssymb}
\usepackage{hyperref}
\usepackage{url}
\usepackage{graphicx}
\usepackage{geometry}
\usepackage{enumitem}
\usepackage{parskip}
\usepackage{chemfig}
\usepackage{pdfpages}
\usepackage{tikz}
\usepackage{fancybox}
\usepackage{makecell}
\usepackage{pgfplots}
\usepackage{soul}
\usepackage{ulem}
\usepackage{wrapfig}
\usepackage{subcaption}
\usepackage[T1]{fontenc}
\usepackage{esvect}
\usepackage{xcolor}
\usepackage[utf8]{inputenc}
\usepackage{csquotes}
\usepackage[style=apa, backend=biber]{biblatex}
\addbibresource{refs.bib}
\usetikzlibrary{arrows}
\usetikzlibrary{decorations.pathreplacing}
\pgfplotsset{compat=1.17}

\geometry{
    a4paper,
    total={170mm, 257mm},   
    left=20mm,
    top=20mm
}   

\hypersetup{
    colorlinks=true,
    linkcolor=black,
    urlcolor=blue,
    pdftitle={Context 1}
}

\pagestyle{fancy}
\fancyhf{}
\fancyhead[L]{\leftmark}
\fancyhead[R]{\thepage}
\renewcommand{\headrulewidth}{0.4pt}
\newcommand{\figbox}[1]{ 
    \begin{figure*}[ht!]        
        \begin{center}            
            \fbox{#1}        
        \end{center}    
    \end{figure*}
}

\newcommand{\wrapfill}{
    \par
    \ifnum \value{WF@wrappedlines} > 0
        \addtocounter{WF@wrappedlines}{-1}%
        \null\vspace{
            \arabic{WF@wrappedlines}
            \baselineskip
        }
        \WFclear
    \fi
    \phantom{}
}

\newcommand{\difference}{\,\backslash\,}
\newcommand{\rem}{\underline{Remark}: }
\newcommand{\nots}{\underline{Notation}: }
\newcommand{\prf}{\underline{Proof}: }
\newcommand{\exs}{\underline{Example}: }
\newcommand{\defs}{\underline{Definition}: }
\newcommand{\wrn}{\underline{Warning}: }
\newcommand{\sht}{\ |\ }
\newcommand{\pph}[1]{\paragraph{#1} \phantom{}\\}
\newcommand{\dm}{\displaystyle}
\newcommand{\rad}{^{\mathrm{c}}}

\titleformat{\chapter}[block]{\bfseries\Large}{\thechapter.}{1em}{}

%========= TEXT ===========

\begin{document}

\begin{titlepage}
    \begin{flushleft}
        \includegraphics[width=.3\textwidth]{media/hslu-logo-2.png}\\
        \hspace*{.16cm}\includegraphics[width=0.4\textwidth]{media/hslu-svg-logo.png}
    \end{flushleft}

    \vspace*{-.4cm}
    {\huge \textbf{Project Context 1}}

    {\Large Review 2}

    \vspace*{3cm}
    \begin{center}
        {\Huge \textbf{My$_{\text{celium}}$Tent}}

        \vspace*{.1cm}
        \textbf{\large The biodegradable tent}

%        \vspace*{.5cm}
%        \includegraphics*[width=.35\textwidth]{media/mytent_logo.png}
    \end{center}

    \vfill
    {\Large \textbf{Team 31}}\\
    {\large \vspace*{.01cm}
        Althaus Simon\\
        \vspace*{.01cm}
        Berner Nic\\
        \vspace*{.01cm}   
        Frongillo Matteo\\
        \vspace*{.01cm}
        McCarthy Benjamin\\
        \vspace*{.01cm}
        Nyamdorj Narandavaa\\
        \vspace*{.01cm}
    }
    
    \vspace{.75cm}
    {\large
    Author: Frongillo Matteo\\
    Horw, 14th October 2024
    }
\end{titlepage}

\tableofcontents
\thispagestyle{empty}

\newpage
\section{Annotated research}
\subsection{Mycelium}
\subsubsection{What is mycelium}
\subsubsection{Technical informations}
{\small Research by Frongillo Matteo}

The potential of biodegradable materials based on fungal species combined with
lignocellulosic residues is very high. The mechanical, physical and chemical
properties of mycelium are extremely good, promising confidence for the future, which
could see, as an environmentally friendly alternative to synthetic polymers, precisely
mycelium-based materials.\\

\fullcite{techinfo}

\subsubsection{The future of mycelium}
{\small Research by Frongillo Matteo}

The current status of pure mycelium-based materials and its future prospects are a key
focus for research on this material. Its versatile properties are a great strength, making
mycelium the main rival to chemically produced materials.\\

\fullcite{future}

\subsection{Marketing}
{\small Research by Nyamdorj Narandavaa}

The analysis of the marketing of camping tents and outdoor equipment is based on the use
of strategies leveraged to reach and engage consumers globally, in Europe and Switzerland.
Nowadays, the use of social media is a key focus for product promotion and customer
loyalty through targeted advertising and influencers.

\subsubsection{Worldwide}
In 2022, the global camping tent market reached a total value of USD 2.65 billion. The
future of this market is positive; in fact, it is set to grow further to USD 4 billion by
2028. This is possible due to the increase in outdoor recreation and nature tourism.\\

\fullcite{worldwide}

\subsubsection{In Europe}
According to reported projections, by 2029 the European camping tent market will grow
significantly, reaching a value of USD 1.50 billion. The analysis covers various product
categories, materials, and capacities, highlighting the increasing demand for innovative
and practical solutions in line with new camping trends.\\

\fullcite{europe}

\subsubsection{In Switzerland}
The analysis of the camping tent market in Switzerland focuses on growth trends driven by
increased outdoor activities and investment in sustainable materials.
Changes in consumer preferences are also part of the analysis.\\

\fullcite{switzerland}

\subsubsection{Marketing strategies}
Key marketing strategies for outdoor brands are analyzed based on their impact on their
audiences. In fact, content posted on social media is strongly considered, facilitating
one's visibility through influencers, who play a key role on product promotion and
customer loyalty.\\

\fullcite{marketing}

\subsection{Structure}
{\small Research by Althaus Simon}

Tent structures are the focus of this section, particularly those that offer practical and
durable designs. Ease of set-up is also analyzed, providing an overview of suitable
solutions for various usage needs.

\subsubsection{Tent structure}
The manufacturer Hilleberg produces tents in the high-quality and expedition sector.
Hilleberg values the simplest possible handling, which is why it makes sense to look at
the structures and set-ups they use.\\

\fullcite{structure}

\subsubsection{Setup}
A report on various tent structures, all of which promise to take little time to set up.\\

\fullcite{setup}

\subsection{Safety}
\subsubsection{Mycelium safety}
\subsubsection{Tent safety}
{\small Research by Althaus Simon}

The DIN EN ISO 5912:2020 standard provides information on safety requirements for tents,
including mechanical performance and fire safety standards.\\

\fullcite{tent-safety}

\subsection{Environment}
{\small Research by McCarthy Ben}

The topic focuses on mycelium-bonded composites in the environmental context and
analyzes their biodegradability and lifecycle properties.

\subsubsection{Properties of mycelium-bound composites}
This source compares different substrate and fungus combinations to test different
properties of mycelium-bound composites. Some of the properties this source compares are
growth conditions, water and moisture absorption as well as sound absorption. This study
was presented at the International Conference on Bio-Based Building Materials, which is a
conference for information exchange and discussions on research of bio-construction
materials.\\

\fullcite{composites}

\subsubsection{Biodegradation}
This source demonstrates the biocycle for mycelium-bound composites (MBC) with the use of
the fungus Pleurotus ostreatus grown on bamboo microfibers substrate. It shows that the
MBC’s can biodegrade in soil in as little as two months.\\

\fullcite{biodegradation}

\newpage
\setlength{\bibitemsep}{1.2\baselineskip}
\printbibliography


\end{document}  