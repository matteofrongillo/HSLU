\DeclareAcronym{Amorphous}{
  short = Amorphous,
  long  = Non-crystalline material with no long-range order.
}

\DeclareAcronym{Crystalline}{
  short = Crystalline,
  long  = Material with atoms arranged in a highly ordered microscopic structure{,} forming a crystal lattice that extends in all directions.
}

\DeclareAcronym{Monocrystalline}{
  short = Monocrystalline,
  long  = Material consisting of a single crystal or a continuous crystal lattice with no grain boundaries.
}

\DeclareAcronym{Polycrystalline}{
  short = Polycrystalline,
  long  = Material composed of many crystallites of varying size and orientation.
}

\DeclareAcronym{Anisotropy}{
  short = Anisotropy,
  long  = Direction-dependent properties of a material {\color{red}(Monocrystalline and polycrystalline with texture)}
}

\DeclareAcronym{Isotropy}{
  short = Isotropy,
  long  = Direction-independent properties of a material {\color{red}(Amorphous)}
}

\DeclareAcronym{Quasiisotropy}{
  short = Quasi-isotropy,
  long  = Approximate isotropy in polycrystalline materials with random grain orientation {\color{red}(Polycrystalline without texture)}
}

\DeclareAcronym{Polymorphism}{
  short = Polymorphism / Allotropy,
  long  = Ability of a material to exist in more than one form or crystal structure.
}

\DeclareAcronym{Homogeneous}{
  short = Homogeneous,
  long  = Uniform composition and properties throughout the material.
}

\DeclareAcronym{Heterogeneous}{
  short = Heterogeneous,
  long  = Non-uniform composition and properties throughout the material.
}

\DeclareAcronym{Alloy}{
  short = Alloy,
  long  = A mixture of two or more elements{,} where at least one element is a metal.
}

\DeclareAcronym{Dislocation}{
  short = Dislocation,
  long  = A linear defect in the crystal structure where there is an irregularity in the arrangement of atoms.
}

\DeclareAcronym{Vacancy}{
  short = Vacancy,
  long  = A point defect in a crystal lattice where an atom is missing from its regular lattice site.
}

\DeclareAcronym{Slip}{
  short = Slip,
  long  = Large displacement of one part of a crystal relative to another part along crystallographic planes and directions.
}

\DeclareAcronym{HCF}{
  short = HCF,
  long = High-cycle fatigue. It occurs when materials are subjected to stresses much lower than their yield strength{,} at a high number of cycles.
}

\DeclareAcronym{LCF}{
  short = LCF,
  long = Low-cycle fatigue. It happens when materials are subjecter to higher stresses{,} typically exceeding the yield strength{,} at a smaller number of cycles.
}

\DeclareAcronym{Poisson}{
  short = Poisson's ratio $\nu$,
  long = The ratio of transverse strain to longitudinal strain in a material under uniaxial loading.
}

\DeclareAcronym{Young's modulus}{
  short = Young's modulus $E$,
  long = The ratio of normal stress to longitudinal strain in the elastic range of a material.
}

\DeclareAcronym{Shear modulus}{
  short = Shear modulus $G$,
  long = The ratio of shear stress to shear strain in the elastic range of a material.
}