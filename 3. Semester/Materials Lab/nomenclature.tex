\setlength{\nomlabelwidth}{2cm} % width of the symbol column
\renewcommand{\nomlabel}[1]{\makebox[\nomlabelwidth][l]{#1}}

% vertical spacing between items
\setlength{\nomitemsep}{0.5\baselineskip}

% group titles
\renewcommand\nomgroup[1]{%
  \item[\bfseries
    \ifstrequal{#1}{E}{Elastic moduli}{%
    \ifstrequal{#1}{S}{Strength measures}{%
    \ifstrequal{#1}{D}{Ductility measures}{%
    \ifstrequal{#1}{R}{Ratios}{%
    \ifstrequal{#1}{T}{Transformations and Processes abbreviations}{%
    \ifstrequal{#1}{A}{Alloy series}{%
    \ifstrequal{#1}{C}{Coatings and Surface Treatments}{}}}}}}}%
    ]\addvspace{0.7\baselineskip}%
}


% Ductility measures
\nomenclature[D]{$A_\text{g}$}{Uniform strain}
\nomenclature[D]{$A$}{Fracture strain}
\nomenclature[D]{$A_n$}{Elongation measured with \ensuremath{\dm L_0 = n\sqrt{S_0}}}
\nomenclature[D]{$Z$}{Contraction at fracture (elongation at break, in \%)}
\nomenclature[D]{$KV$}{Notch impact energy, in J, indicator of toughness}

% Elastic moduli
\nomenclature[E]{$E$}{Young's modulus \textrightarrow\ 70GPa for aluminum, 210 GPa for steel}
\nomenclature[E]{$G$}{Shear modulus}

% Ratios
\nomenclature[R]{$\nu$}{Poisson's ratio}
\nomenclature[R]{PREN}{Pitting Resistance Equivalent Number = \%Cr + 3.3\%Mo + 16\%N}

% Strength measures
\nomenclature[S1]{$R_{\text{p}0.2}$}{0.2\% yield strength}
\nomenclature[S1]{$R_\text{eH}$}{Upper yield point}
\nomenclature[S1]{$R_\text{m}$}{Tensile strength}
\nomenclature[S1]{HV}{Vickers hardness}
\nomenclature[S1]{HB}{Brinell hardness}
\nomenclature[S1]{HRC}{Rockwell C hardness}

% Transformation / Process abbreviations
\nomenclature[T]{OBC}{Oxygen-Blown Converter}
\nomenclature[T]{EF}{Electric Furnace}
\nomenclature[T]{OHF}{Open Hearth Furnace}
\nomenclature[T]{EKD}{Iron-Iron Carbide Phase Diagram}
\nomenclature[T]{ZTU/CCT}{Continuous Cooling Transformation diagram}
\nomenclature[T]{SHD}{Surface Hardening Depth}
\nomenclature[T]{CHD}{Case Hardening Depth}
\nomenclature[T]{NHD}{Nitriding Hardness Depth}
\nomenclature[T]{PM}{Powder Metallurgy}
\nomenclature[T]{ESR}{Electro Slag Remelting}
\nomenclature[T]{VAR}{Vacuum Arc Remelting}
\nomenclature[T]{Q+T}{Quenching and Tempering}
\nomenclature[T]{HSS}{High-Speed Steel}
\nomenclature[T]{ETG}{Name of the steel grade according to the Swiss standard}
\nomenclature[T]{DRI}{Direct Reduced Iron}
\nomenclature[T]{AW}{Wrought Aluminum}
\nomenclature[T]{AC}{Cast Aluminum}
\nomenclature[T]{IC}{Intercrystalline corrosion}
\nomenclature[T]{SCC}{Stress Corrosion Cracking}
\nomenclature[T]{FSW}{Friction Stir Welding}
\nomenclature[T]{HAZ}{Heat Affected Zone}
\nomenclature[T]{L$_1$}{Hypoeutectoid Steel (0.35\% C)}
\nomenclature[T]{L$_2$}{Eutectoid Steel (0.8\% C)}
\nomenclature[T]{L$_3$}{Hypereutectoid Steel (1.4\% C)}

% Alloy series
\nomenclature[A]{1XXX}{Commercially pure aluminum (high conductivity, soft)}
\nomenclature[A]{2XXX}{Al-Cu alloys (high strength, poor corrosion resistance)}
\nomenclature[A]{3XXX}{Al-Mn alloys (good formability, non-heat-treatable)}
\nomenclature[A]{4XXX}{Al-Si alloys (good castability, used in welding)}
\nomenclature[A]{5XXX}{Al-Mg alloys (good weldability and corrosion resistance)}
\nomenclature[A]{6XXX}{Al-Mg-Si alloys (medium strength, excellent formability)}
\nomenclature[A]{7XXX}{Al-Zn-Mg alloys (very high strength, lower corrosion resistance)}

% Surface Treatments and Coatings
\nomenclature[C]{CAA}{Chromic Acid Anodizing}
\nomenclature[C]{CCC}{Chromate Conversion Coating}
\nomenclature[C]{Cr(VI)}{Hexavalent chromium (toxic, high corrosion resistance)}
\nomenclature[C]{Cr(III)}{Trivalent chromium (eco-friendly)}
\nomenclature[C]{GS}{Sulfuric Acid Anodizing (decorative)}
\nomenclature[C]{GSX}{Sulfuric + Oxalic Acid Anodizing (protective, weat-resistant)}
\nomenclature[C]{TSA}{Tartaric-Sulfuric Acid Anodizing (Cr-free, aerospace use)}
\nomenclature[C]{PUR}{Polyurethane paint or vanish (top protective layer)}
\nomenclature[C]{Zn-NI}{Zinc-Nickel galvanic coating for paint adhesion}
\nomenclature[C]{Phosphating}{Chemical conversion coating for paint adhesion}
\nomenclature[C]{Chromating}{Chemical conversion layer of chromates or chromites}