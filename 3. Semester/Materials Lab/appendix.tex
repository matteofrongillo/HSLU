\appendix
\section*{Appendix}
\addcontentsline{toc}{section}{Appendix}

\subsection*{Indices and Abbreviations}
\begin{description}
\item[Amorphous] Non-crystalline material with no long-range order.
\item[Crystalline] Material with atoms arranged in a highly ordered microscopic structure, forming a crystal lattice that extends in all directions.
\item[Monocrystalline] Material consisting of a single crystal or a continuous crystal lattice with no grain boundaries.
\item[Polycrystalline] Material composed of many crystallites of varying size and orientation.
\item[Anisotropy] Direction-dependent properties of a material {\color{red}(Monocrystalline and polycrystalline with texture)}
\item[Isotropy] Direction-independent properties of a material {\color{red}(Amorphous)}
\item[Quasi-isotropy] Approximate isotropy in polycrystalline materials with random grain orientation {\color{red}(Polycrystalline without texture)}
\item[Polymorphism / Allotropy] Ability of a material to exist in more than one form or crystal structure.
\item[Homogeneous] Uniform composition and properties throughout the material.
\item[Heterogeneous] Non-uniform composition and properties throughout the material.
\item[Alloy] A mixture of two or more elements, where at least one element is a metal.
\end{description}