\documentclass{article}

\usepackage{Engineering}
\pdftitle{Materials Lab}

\newcommand{\qs}{\vspace{2.5cm}} 

% === TEXT ===
\title{\textbf{Materials Lab Questions \\ HSLU, Semester 3}}
\author{Matteo Frongillo}
\date{}

\begin{document}

\maketitle
\tableofcontents
\newpage

\section{Basics and Crystal Structure}
\begin{enumerate}[label=\textbf{1.\arabic*}]
    \item How are Young's modulus, binding energy, and the coefficient of thermal expansion qualitatively related? 
    \qs
    \item What is the plastic deformation of metals at room temperature based on? 
    \qs
    \item In dishwashers, the side walls often consist of stainless ferritic chromium steel (body-centered cubic, BCC). For the bottom, which is subjected to very large plastic deformations during manufacturing, the choice usually falls on a stainless austenitic chromium-nickel steel (face-centered cubic, FCC). Explain the material choice for the tub bottom; please argue using the crystal structure. 
    \qs
    \item Why are FCC and BCC metals easier to plastically deform than the hexagonal close-packed (HCP) titanium? 
    \qs
    \item The strength of metals can be increased by deliberately placing obstacles in the form of lattice defects in the path of dislocations. Name four lattice defects and the strengthening mechanisms caused by them. Order them according to the dimension of the lattice defects (zero to three-dimensional). 
    \qs
    \item Spring elements in mechanical watch movements (setting lever springs, metallic hairsprings) are usually cold-rolled. Why? 
    \qs
    \item Please name and explain three factors that decisively influence solid solution hardening. 
    \qs
    \item Electrical sheets for transformers consist of BCC iron with 2-4\% Si (solid solution, suppression of eddy currents). These sheets often have a Goss texture, i.e., the (100) directions of the crystals lie preferentially in the rolling direction and the \{110\} planes parallel to the sheet surface.
    \begin{enumerate}
        \item Please draw a BCC unit cell. 
        \vspace{4cm} % Extra space for drawing
        \item Draw an electrical sheet with Goss texture in the correct orientation into this unit cell and mark the rolling direction of type (100) in this plane of type \{110\}. 
        \vspace{4cm} % Extra space for drawing
    \end{enumerate}
    \item Drawn copper wires possess a (111) fiber texture, i.e., planes of type \{111\} are oriented perpendicular to the wire axis or the pulling direction of the wire.
    \begin{enumerate}
        \item Draw an FCC unit cell. 
        \vspace{4cm} % Extra space for drawing
        \item Draw a plane of type \{111\} as well as the direction of the wire axis or the pulling direction of the wire of type (111) perpendicular to it into the FCC unit cell. 
        \vspace{4cm} % Extra space for drawing
    \end{enumerate}
\end{enumerate}

\newpage
\section{Alloys}
\begin{enumerate}[label=\textbf{2.\arabic*}]
    \item The aluminum casting alloy EN AC-AlSi10Mg (EN AC-43000) is used for both Sand Casting (S) and Permanent Mold/Die Casting (K). In the cast state, the properties differ. Why is the strength of the alloy higher after permanent mold casting than after sand casting? 
    \qs
    \item The binary system Copper-Nickel is a phase diagram with complete solubility. What three conditions must generally be met for such a binary system to exist? 
    \qs
    \item All Swiss coins from ten centimes to five francs have consisted of a CuNi25 alloy (Cu with 25\% Ni) since 1968. Please describe, using the Copper-Nickel binary system, how the state of aggregation would change with increasing temperature for a five-franc coin if it were heated from room temperature to $1300^{\circ}C$ with a gas burner. 
    \qs
    \item What is the chemical composition of the first crystals of the CuNi25 alloy when this alloy is cooled from the melt? 
    \qs
    \item Describe the state of this CuNi25 alloy at $1160^{\circ}C$. Which phases are present in what proportions (Lever rule)? How are these phases chemically composed? 
    \qs
    \item What is the composition of the last melt when this alloy is cooled? 
    \qs
    \item Please describe what segregations are and to what extent they can occur in this coin alloy? 
    \qs
    \item What disadvantages would segregations have in this coin alloy? 
    \qs
    \item Aluminum-Silicon phase diagram: At the eutectic point, there are classic casting alloys. Why is the alloy with 12.5\% Silicon in the Al-Si binary system a good casting alloy? Please name two arguments. 
    \qs
    \item Briefly describe the structure of the microstructure of the alloy with 12.5\% Si (casting alloy) and the alloy with 35\% Si (piston alloy) at room temperature. 
    \qs
\end{enumerate}

\newpage
\section{Iron-Carbon Diagram}
\begin{enumerate}[label=\textbf{3.\arabic*}]
    \item Please determine, using the lever rule for the hypoeutectoid alloy $L_{1}$ (Fig 3.4), which phases are present in what proportions with what respective chemical composition at $800^{\circ}C$. 
    \qs
    \item Which phases with what chemical composition are present at $800^{\circ}C$ for the eutectoid alloy $L_{2}$? 
    \qs
    \item Please determine, using the lever rule for the hypereutectoid alloy $L_{3}$, which phases are present in what proportions with what respective chemical composition at $800^{\circ}C$. 
    \qs
    \item Please name two applications for hypoeutectoid steels. 
    \qs
    \item Please name one application for hypereutectoid steel. 
    \qs
\end{enumerate}

\newpage
\section{Materials Testing}
\begin{enumerate}[label=\textbf{4.\arabic*}]
    \item Please briefly explain the meaning of the isotropic elastic material parameters: Young's Modulus $E$, Poisson's ratio $\nu$, and Shear modulus $G$. 
    \qs
    \item Please briefly explain the meaning of the plastic material parameters: 0.2\% Proof Stress $R_{p0.2}$, Tensile Strength $R_{m}$, and Elongation at Break $A$. 
    \qs
    \item Which parameter is usually of greater importance for a design engineer: the 0.2\% proof stress $R_{p0.2}$ or the tensile strength $R_{m}$? Please justify your answer. 
    \qs
    \item Using the stress-strain diagrams for EN AW-6082-T6 (Abb. 4.40 and 4.41), read the parameters Young's Modulus $E$, 0.2\% proof stress $R_{p0.2}$, Tensile Strength $R_{m}$, and Elongation at Break $A$ as accurately as possible. 
    \qs
    \item Which steel is a spring steel with high 0.2\% proof stress $R_{p0.2}$: a steel with 0.2\% C or a steel with 0.7\% C and 4\% Si? Please justify your answer. 
    \qs
    \item May an aluminum profile loaded in tension with an initial cross-section $A_{0} = 500~mm^{2}$ be loaded with 150 kN without plastically deforming if its 0.2\% proof stress is 280 MPa? 
    \qs
    \item Which value from the tensile test is decisive for a predetermined breaking point? Justify the choice. 
    \qs
    \item When launching an F/A 18F Superhornet, a release rod (20 mm diameter) holds the aircraft. Engine thrust: $2 \times 98,000$ N. Catapult force: 1,235,000 N. The thrust must not plastically deform the rod, but the catapult must tear it (force $>3 \times$ force at tensile strength). Decide if a steel with $R_{p0.2} = 985$ MPa and $R_{m} = 1075$ MPa meets the requirements. 
    \qs
    \item Excessive magnetic reversal losses occur in an electrical sheet. Which measurement methods would principally be suitable for determining grain size, texture, and chemical composition? 
    \qs
    \item A stainless chromium-nickel steel has discolored. You need to distinguish between a thin temper color (max 10 nm oxide) and organic contamination. If it is oxide, you need the Cr/Fe ratio. EDX penetrates too deep. Which method do you choose? 
    \qs
\end{enumerate}

\newpage
\section{Corrosion}
\begin{enumerate}[label=\textbf{5.\arabic*}]
    \item Please explain the terms Anode, Cathode, and Electrolyte. 
    \qs
    \item Please name four necessary conditions for a corrosion reaction to occur. 
    \qs
    \item \begin{enumerate}
        \item Under what conditions is the nobler half-cell in a corrosion reaction the oxygen electrode? 
        \qs
        \item Under what conditions is the nobler half-cell in a corrosion reaction the hydrogen electrode? 
        \qs
    \end{enumerate}
    \item Why does hydrogen corrosion generally proceed faster at acidic pH values than oxygen corrosion at neutral pH values? 
    \qs
    \item Can the noble metal copper corrode in a humid environment at a neutral pH value? 
    \qs
    \item Zinc layers are used as active corrosion protection on steel (e.g., power pylons). Explain why zinc corrodes instead of iron. Write the anodic and cathodic partial reactions. Which cathodic reaction would occur in an acidic electrolyte? 
    \qs
    \item Case Study: A welded sheet of austenitic stainless steel 1.4301 corrodes along the weld seam. Discuss the cause and suggest two possible solutions. 
    \qs
    \item Case Study: Gold-plated brass plugs are corroding under scratches in the gold layer.
    \begin{enumerate}
        \item Does the gold act as active or passive protection? 
        \qs
        \item Why does the brass corrode strongly under small scratches? 
        \qs
        \item Suggest two measures to avoid this. 
        \qs
    \end{enumerate}
    \item Pitting corrosion occurs in a 1.4301 stainless steel cooking chamber due to chloride concentration ($130$ mg/l). Suggest solutions. 
    \qs
    \item You need a stainless steel for a water-carrying device (15-year life). A supplier offers a cheap high-sulfur (0.1\% S) version and a dearer low-sulfur (0.01\% S) version. Which do you choose and why? 
    \qs
\end{enumerate}

\newpage
\section{Tribology and Surface Technology}
\begin{enumerate}[label=\textbf{6.\arabic*}]
    \item Which is greater – static friction or kinetic friction? 
    \qs
    \item Can the coefficient of friction take values greater than one? Name an example. 
    \qs
    \item Name two advantages and two disadvantages of dry lubrication (graphite/MoS$_2$) vs. wet lubrication (oil). 
    \qs
    \item Which metallic materials are susceptible to adhesive wear? Where is this exploited? 
    \qs
    \item Name and explain three ways to improve coating adhesion. 
    \qs
    \item Name and explain two advantages of electroless nickel deposition over galvanic. 
    \qs
    \item Why are hexavalent chromium electrolytes banned (RoHS)? Name an alternative for galvanized screws. 
    \qs
    \item Why does the oxide layer grow into the surface during aluminum anodizing? Name an alternative to chromic acid anodizing. 
    \qs
    \item Name and explain two advantages of powder coatings over wet paint. 
    \qs
    \item What are PVD and CVD coatings used for? Name three applications. 
    \qs
\end{enumerate}

\newpage
\section{Steel Technology}
\begin{enumerate}[label=\textbf{7.\arabic*}]
    \item What happens to steel in the ESR (Electro-Slag Remelting) process? What are the pros/cons and applications? 
    \qs
    \item Why is martensite not in the iron-carbon diagram? What is the advantage of bainite over martensite? 
    \qs
    \item Which microstructural constituents form if 34CrMo4 is cooled according to curve No. 4 in the TTT diagram (Fig 7.24)? 
    \qs
    \item Is pearlite a phase? 
    \qs
    \item Explain quench and tempering of carbon steels. Difference to austempering? 
    \qs
    \item What is the heat treatment for strong and tough fine-grained structural steels? 
    \qs
    \item Difference between carbonitriding and nitrocarburizing? 
    \qs
    \item How is hardening depth defined for surface hardening vs. case hardening? Difference between surface-hardening steels and case-hardening steels? 
    \qs
    \item Which elements increase hardenability (max surface hardness)? 
    \qs
    \item Which elements increase the depth of hardening? 
    \qs
\end{enumerate}

\newpage
\section{Steel Designation}
\begin{enumerate}[label=\textbf{8.\arabic*}]
    \item Compare C35E and 34CrMo4. Name one advantage of each. Give their chemical compositions. 
    \qs
    \item Derive the short name for material 1.4301 (0.05\% C, 18\% Cr, 10\% Ni). 
    \qs
    \item Which is intended for heat treatment: Quality steel or High-grade steel (Edelstahl)? 
    \qs
    \item Which material numbers apply to unalloyed tool steels, alloyed tool steels, and high-speed steels? 
    \qs
    \item Difference between C45 and C45E? 
    \qs
\end{enumerate}

\newpage
\section{Special Steels}
\begin{enumerate}[label=\textbf{9.\arabic*}]
    \item Why are high-alloy austenitic steels chosen for very low temps ($-50^{\circ}C$)? 
    \qs
    \item How can the high-temperature strength of steels be increased? 
    \qs
    \item Advantage and disadvantage of maraging steels vs. heat-treatable steels? 
    \qs
    \item Which is a good spring steel: 0.2\% C or 0.56\% C + 1.75\% Si? Why? 
    \qs
    \item Name three steel grades for automotive body stiffening (B-pillar, etc.). 
    \qs
    \item 1.4016 (0.7\% C) is hard to weld. Suggest an alternative ferritic stainless steel for a washing machine drum. Which element helps weldability/corrosion? 
    \qs
    \item Pitting in a 1.4301 tank due to chloride ($70$ mg/l). Name two prevention methods. 
    \qs
    \item Advantage of seawater-resistant duplex steels vs. super-austenitic steels? 
    \qs
    \item Pros/cons of cold-work vs. hot-work tool steels? Two applications for each? 
    \qs
    \item Pros of HSS tools vs. ceramic tools? Main elements in HSS? 
    \qs
\end{enumerate}

\newpage
\section{Cast Iron}
\begin{enumerate}[label=\textbf{10.\arabic*}]
    \item Which use stable vs. metastable diagrams? Match: GJL, GS, GJN, GJS, GJV. 
    \qs
    \item Why is austenitic cast steel GX5CrNi19-10 better for cryo tech than ferritic GE300? 
    \qs
    \item Compare Nodular Cast Iron (GJS) vs. Grey Cast Iron (GJL). One advantage and application for each. 
    \qs
    \item Match applications (Crankshaft, Pump housing, Grinding balls) to types (White cast iron, ADI, Austenitic cast iron). 
    \qs
    \item How is the melt treated to create nodular graphite instead of flake graphite? 
    \qs
\end{enumerate}

\newpage
\section{Aluminum}
\begin{enumerate}[label=\textbf{11.\arabic*}]
    \item Why is aluminum a "young" industrial metal (only approx. 100 years)? 
    \qs
    \item Name five properties and resulting applications of aluminum. 
    \qs
    \item Why replace Steel or Copper with Aluminum despite lower strength/conductivity? 
    \qs
    \item List main elements and strengthening mechanisms for EN-AW 2XXX, 3XXX, 5XXX, 6XXX, 7XXX. 
    \qs
    \item What causes the highest strength increase? (Solid solution, Work hardening, Natural aging, Artificial aging). 
    \qs
    \item Which precipitate type gives greatest strength? (Coherent, Semi-coherent, Incoherent). 
    \qs
    \item Compare 2024, 7075, 6061 regarding: Fatigue (T3/T4), Strength (T6), Corrosion, Weldability. 
    \qs
    \item Which binary system are most casting alloys based on? Elements for precipitation hardening? 
    \qs
    \item Name four forming processes and four casting processes for aluminum. 
    \qs
    \item How is EN AW-2024 made corrosion resistant? 
    \qs
\end{enumerate}

\newpage
\section{Non-Ferrous Metals}
\begin{enumerate}[label=\textbf{12.\arabic*}]
    \item Which light alloys offer strength of heat-treatable steel but lower weight? 
    \qs
    \item Why is Titanium corrosion resistant (biocompatible) despite being non-noble? 
    \qs
    \item Name two negative properties of Magnesium preventing it from replacing Aluminum. 
    \qs
    \item Which properties allow Ni-superalloys in turbine blades? How to increase creep strength? 
    \qs
    \item Which Ni alloy is best for maritime atmospheres? 
    \qs
    \item Which Ni alloy is used for soft magnetic cores in quartz watches? 
    \qs
    \item How is high purity Copper produced? 
    \qs
    \item Two applications of Zinc? 
    \qs
    \item Advantage of Niobium/Tantalum over Tungsten/Molybdenum at room temp? 
    \qs
    \item Why can gold wires be friction welded (piezo actuators)? 
    \qs
\end{enumerate}

\newpage
\section{Ceramics and Glass}
\begin{enumerate}[label=\textbf{13.\arabic*}]
    \item Three pros and three cons of ceramics vs. metals. 
    \qs
    \item What is the Weibull modulus? Which has higher modulus: HIP or sintered Al$_2$O$_3$? Effect of sample volume on Weibull plot? 
    \qs
    \item Silicate vs. High-performance ceramic? Why is single-crystal Al$_2$O$_3$ transparent but poly- not? 
    \qs
    \item Three requirements for refractory ceramics. 
    \qs
    \item Components of porcelain? Function of glaze? Advantage of Cordierite? 
    \qs
    \item Properties of Yttrium-stabilized Zirconia (Mechanical \& Electrical). 
    \qs
    \item Decisive advantage of glass-ceramic? 
    \qs
    \item Why is float glass perfectly flat? 
    \qs
    \item Components of window glass? Use for oven window? Alternative? Why is chemically toughened glass stronger? 
    \qs
    \item Structural feature of piezoelectric ceramics? 
    \qs
\end{enumerate}

\newpage
\section{Polymers}
\begin{enumerate}[label=\textbf{14.\arabic*}]
    \item Match: Thermoplast, Thermoset, Elastomer to Cross-link density (Many, Few, None). Give examples. 
    \qs
    \item Structural features increasing strength/melting point of thermoplasts? Which melt $>200^{\circ}C$? 
    \qs
    \item Write a biaxial stress tensor for a pressurized container. How to increase strength in stress direction without fibers? 
    \qs
    \item Higher strength: PE or PS? Why? Copolymer for Styrene to improve toughness? 
    \qs
    \item Advantage of POM vs PA? 
    \qs
    \item Advantage of EPDM vs Classic Elastomers? 
    \qs
    \item Applications of thermosets? 
    \qs
    \item Sources of microplastic? 
    \qs
    \item Problem with oxo-degradable plastics? 
    \qs
    \item Name a biodegradable plastic and application. Why not suitable for all uses? 
    \qs
\end{enumerate}

\end{document}