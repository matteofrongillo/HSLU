\documentclass{article}

\usepackage{Mathematics}
\pdftitle{Linear Algebra}

% === TITLE ===
\title{\textbf{Linear Algebra \\ HSLU, Semester 4}}
\author{Matteo Frongillo}
\date{}

% === TEXT ===
\begin{document}

\maketitle
\tableofcontents
\pagebreak

\section{Vectors}
\subsection{Linear combination}
A sum of scalings of vectors is called a linear combination of the vectors.

Let $\vec{u}$, $\vec{v}$ be vectors, and $a$, $b$ be scalars, $a,b \in \mathbb{R}$, then:
\figbox{$\dm a\cdot\vec{u} + b\cdot\vec{v}$}

Generalizing this to a set of vectors $\vec{u}_1, \dots, \vec{u}_n$, and
scalars $a_1, \dots, a_n$, we have:
\figbox{$\dm \sum_{i=1}^{n} a_i \cdot \vec{u}_i$}

\subsection{Cross product (vector product)}
The cross product of two vectors $\vec{u}$ and $\vec{v}$ is a
vector $\vec{w}$ that is perpendicular to both $\vec{u}$ and $\vec{v}$,
and has a magnitude equal to the area of the parallelogram formed by $\vec{u}$ and $\vec{v}$.

Let $\vec{u} = (x,y,z)$ and $\vec{v}=(t,s,q)$
\figbox{$\dm \vec{u} \times \vec{v} = (yq-sz,\ -(xq-tz),\ xs-yt)$}

\subsection{Unit vectors}
Components of the unit vectors of the vector basis are as follow:
\figbox{$\dm \vec{i} = \vec{e}_x = \begin{bmatrix}
    1 \\ 0 \\ 0
\end{bmatrix} \qquad \vec{j} = \vec{e}_y = \begin{bmatrix}
    0 \\ 1 \\ 0
\end{bmatrix} \qquad \vec{k} = \vec{e}_z = \begin{bmatrix}
    0 \\ 0 \\ 1
\end{bmatrix}$}

with their norms being $\Vert{\vec{i}}\Vert = \Vert{\vec{j}}\Vert = \Vert{\vec{k}}\Vert = 1$












\end{document}