\documentclass{article}

\usepackage{Paper}
\renewcommand\theadfont{\bfseries}
\setcellgapes{2pt}

\pdftitle{Team1_Presentation_Handout}

% === TEXT ===
\begin{document}
\thispagestyle{empty}

\begin{minipage}{0.7\textwidth}
    \vspace*{-.8cm} \hspace*{-0.3cm}
    \includegraphics[width=.5\textwidth]{media/hslu-logo.png}
\end{minipage}

\vspace*{1.5cm}

\textbf{\huge Presentation handout}\\[1.5cm]
\begin{center}
    \textbf{\huge Separation techniques in a WWTP:}
    
    \textbf{\huge Precipitation, Flocculation, and EC}\\[4.5cm]
    
    \includegraphics[width=\textwidth]{media/separation_wwtp.png}\\
\end{center}

\vfill

\setlength{\intextsep}{0pt}%
\begin{wrapfigure}{r}{\textwidth}
    \textbf{\Large Principle of Sustainable Environmental Systems\\[.5cm]
    {\large
    Dr. Macarena San Martín Ruiz\\
    Lecturer}}
    \vspace{-2.1cm}
\end{wrapfigure}

\phantom{}\\[-1cm]

\begin{flushright}
        \large
        \textbf{Team 1}\\
        Bürli Norman\\
        Frongillo Matteo\\
        Murali Arjun\\
        Neukom Yannik\\
        Rossi Anthony
\end{flushright}
\wrapfill
\newpage
\tableofcontents
\pagebreak

\section{Introduction}
Water is in a constant cycle, being renewed and consumed repeatedly.
To minimize the environmental impact of modern civilization,
humans have developed complex systems, Wastewater Treatment Plants
(WWTPs), to purify the wastewater we produce. They ensure safety and
our continued coexistence with nature as we use water as a resource.

Through innovative processes, it has become possible to separate,
extract, and neutralize harmful substances in sewage. These include
large solids such as plastics and food waste, oils and grease,
suspended solids, and organic matter. The subsequent effluent is
clean enough to protect the local ecosystem, while the process
itself enables the recovery of valuable resources such as phosphorus
and energy. 

This handout will explore specific separation techniques using ARA
Rhein as a primary case study. This facility is particularly
relevant due to its mix of communal and industrial treatments.

\subsection{ARA Rhein}
ARA Rhein is one of the major WWTPs in Switzerland. It is located
along the River Rhine, approximately 10 km upstream of Basel.
The treatment plant started operations in 1975 \parencite{pratteln}.
In 2019, the plant was modernized by adding a dissolved air flotation
unit \parencite{DAF_ARA_Rhein}. More investments are being made to
make the plant future-proof, with annual investments of 6 to 10
million Swiss francs \parencite{ara_rhein_abwasserreinigung}.

While industrial wastewater makes up only 40\% of the hydraulic
volume (2.5 billion liters annually), it contributes 90\% of the
pollution load (450’000 PE). In contrast, communal sources account
for the remaining 60\% of the volume (3.5 billion liters) but only
10\% of the pollution load (50,000 PE) \parencite{ara_rhein_sauberes_wasser_2025}. 

To improve energy efficiency, ARA Rhein feeds 6.4 GWh of waste heat
generated by the incineration process into the local heating grid.
That is 17\% of its total energy consumption of 38.2 GWh \parencite{ara_rhein_sauberes_wasser_2025}.
Additionally, ash from the incineration is stored in the Elbisgraben
landfill for future phosphorus recycling. Also, odorous emissions
from communal wastewater are purified to reduce the strain on the
surrounding area \parencite{ara_rhein_abwasserreinigung}.

\section{Separation techniques}
In WWTPs, separation techniques are essential for removing contaminants
at different stages of the process. As each method targets a specific
pollutant, applying them in the correct order and combination is
vital for maximizing the efficiency of the treatment process. At
ARA Rhein, the general flow consists of mechanical treatment,
chemical treatment, biological treatment, sludge treatment,
off-gas treatment, and finally resource recovery and monitoring.
Additionally, the WWTP uses two different process flows for the
communal wastewater and the industrial wastewater.

The communal wastewater is first processed by mechanical treatment,
which consists of a bar screen, a grit chamber and a clarifier that
remove suspended solids. Then it passes through a biological
treatment where microorganisms remove organic compounds. Finally,
the treated water is safely released into the River Rhine, while
the sludge is processed in a separate treatment.

The industrial wastewater is treated slightly differently due to the
characteristics of the contaminants found in the water. Industrial
wastewater often contains fewer big solids and mainly consists of
hazardous chemicals. As a result, the industrial wastewater only
needs minimal mechanical treatment to ensure safe operation.
Importantly, it undergoes chemical treatment and then goes through
an additional biological treatment step. Finally, after this
pretreatment, the stream joins the biological treatment of the
communal wastewater for further cleaning. 

The sludge produced by the biological treatment of both the communal
and industrial wastewater is then thickened and incinerated,
converting it into ash, which is transported to the Elbisgraben landfill
\parencite{ara_rhein_abwasserreinigung}.

A detailed diagram (\autoref{fig:ara_rhein_process}) describes the entire process. 

Electrocoagulation, flocculation, and precipitation have been chosen
from the chemical treatment section for further analysis.
In \autoref{fig:wwtp_process}, a hypothetical wastewater treatment
process visualizes how ARA Rhein could implement the discussed methods. 

\begin{figure}[ht!]
    \centering
    \includegraphics[width=\textwidth]{media/separation_wwtp.png}
    \caption{Hypotetical wastewater treatment process including EC, precipitation, and flocculation.}
    \label{fig:wwtp_process}
\end{figure}

\subsection{Precipitation}
Chemical precipitation in wastewater treatment is a process in which
dissolved pollutants are converted into insoluble solids
(precipitates) through the addition of specific chemicals known as
precipitants \parencite{vsa_faellung_nd, kato_kansha_2024}. In this process,
iron or aluminium salts are typically used as reagents, though
calcium-based compounds may be used for specific dissolved substances.
To remove the precipitates, a physical separation technique such as
filtration must follow \parencite{almawatech_precipitation_2025}.

According to the ARA Rhein treatment scheme, the facility adds a
chemical step for industrial wastewater. In this
stage, the wastewater is neutralized and pre-clarified.
ARA Rhein also has a microflotation system that
acts as a precipitation chamber, using polyaluminium chloride along
with flocculation methods to ensure effective separation.
By combining precipitation with flocculation, a
significant portion of non-soluble matter is removed prior to
biological treatment \parencite{das_ee_ara_rhein_2023}.

The primary advantage of precipitation is the removal of dissolved
contaminants, such as phosphates, which generally cannot be
eliminated through biological or mechanical treatment alone. This
improves effluent quality, allowing ARA Rhein to meet strict
discharge standards. Furthermore, the subsequent biological treatment
process benefits greatly in efficiency, due to the added
concentration of pollutants coming from industrial wastewater sources. 

However, the process has disadvantages. The precipitate and
pollutants that are captured generate significant quantities of
sludge, which needs to be further processed, adding more costs
\parencite{koul_insights_2022}. Moreover, the efficiency and success
of the process heavily depend on controlled pH and reaction
conditions. Small deviations from optimal conditions can cause
incomplete removal or generate excess sludge.

\subsection{Flocculation}
Flocculation is a separation method used to remove tiny, suspended
particles from a liquid by making them clump together into larger,
heavier aggregates called flocs \parencite{iupac_colloid_1972}.

\subsubsection{Flocculation separation procedure}
We consider a system boundary which consists of water containing
impurities in the form of tiny particles which are lightly charged
and do not settle on their own due to electrostatic repulsion.

\begin{figure}[ht!]
    \centering
    \includegraphics[width=0.5\textwidth]{media/flocculation.png}
    \caption{General flocculation process}
    \label{fig:flocculation_process}
\end{figure}

\newpage
A compound such as ferric chloride is added to neutralize the charges
on the particles, making them less likely to repel each other.
Gentle mixing is then applied, and the neutralized particles start
colliding and sticking, forming visible flocs. Once the flocs form,
they can be removed by sedimentation once they settle to the bottom,
or they can be filtered out. \autoref{fig:wwtp_process} illustrates
the general process of flocculation as described above \parencite{ionexchange_flocculation_nd}.

\subsubsection{Application of flocculation at ARA Rhein wastewatertreatment plant}
Flocculation forms part of the chemical treatment steps taken at ARA
Rhein WWTP. Wastewater can contain high amounts of insoluble
phosphorus compounds which can lead to eutrophication
(excessive nutrient enrichment) if released to water bodies \parencite{edmondson_phosphorus_1970}.
ARA Rhein utilizes a flocculation-flotation stage to remove the insoluble impurities.
\autoref{fig:flocculation_ara_rhein} illustrates the control volume, enclosed in a dotted region, of the setup at ARA
Rhein. The incoming water is dosed with a cationic polymer which acts
as the flocculant. The flocculant binds phosphates present in the
water. The mixture then flows into a flotation chamber. In this
stage, tiny air bubbles are injected into the water and get attached
to the flocs, making them buoyant. The flocs then float upwards where
they are skimmed off. This process, as applied above, results in
$>$ 90\% separation of suspended solids.
\vspace{.5cm}
\begin{figure}[ht!]
    \centering
    \includegraphics[width=0.8\textwidth]{media/flotation_tank.png}
    \caption{Flocculation-flotation process at ARA Rhein WWTP}
    \label{fig:flocculation_ara_rhein}
\end{figure}

\subsection{Electrocoagulation}
Electrocoagulation (EC) is a chemical separation technique that
relies on the controlled sacrifice of a less noble metal in a fluid
through the application of electric current to coagulate materials
in water.

\subsubsection{Anode and Cathode}
The anode and cathode are two conductive metal plates (electrodes)
connected to a direct current power source. The anode is positively
charged, while the cathode is negatively charged. 

This positively charged electrode sacrifices itself, oxidizing to
cause contaminants in the water to coagulate via its released ions.
It is typically made of Al or Fe, which releases Al$^{3+}$ or
Fe$^{2+}$ ions upon oxidation. 

On the other hand, the cathode is where the reduction takes place,
transferring electrons to the water molecules and splitting them to
generate gas bubbles, usually hydrogen bubbles, that help float
pollutants to the surface.

\subsubsection{EC reactor and hydrolysis reaction}
In the EC reactor, both electrodes are immersed in water. The anode
oxidizes and causes the metal ions to spread into the wastewater,
whilst the cathode, reduced in the water, releases hydrogen bubbles.
The dissolved metal ions hydrolyze, forming metal hydroxide species,
such as Al(OH)$_3$, that act as direct coagulants \parencite{PHU2025111965}. 

Hydrolysis is a chemical reaction that consists of the scission of a
polymer into monomers, with the addition of a water molecule for each
covalent bond that is split. In electrocoagulation, this reaction
occurs when the anode ions react with water molecules to form
insoluble metal hydroxides. Hydroxides act as the active coagulants
that neutralize charges and catch pollutants to separate them from
the wastewater \parencite{SURESHKUMAR2023639}.
\vspace{.5cm}
\begin{figure}[ht!]
    \centering
    \includegraphics[width=0.6\textwidth]{media/EC.png}
    \caption{Electrocoagulation reactor schematic}
    \label{fig:ec_reactor}
\end{figure}
\vspace{.2cm}

\textbf{Main reaction (with iron electrodes)} \parencite{ECSTransactions_2007}:

Anode dissolution:
\begin{center}
    Fe \textrightarrow\ Fe$^{2+}$ + 2e$^-$
\end{center}

Cathode reaction:
\begin{center}
    2H$_2$O + 2e$^-$ \textrightarrow\ H$_2\uparrow$ + 2OH$^-$
\end{center}

Formation of ferrous hydroxide:
\begin{center}
    Fe$^{2+}$ + 2OH$^-$ \textrightarrow\ Fe(OH)$_2$
\end{center}

Oxidation to ferric hydroxide (Floc):
\begin{center}
    4Fe$^{2+}$ + 10H$_2$O + O$_2$ \textrightarrow\ 4Fe(OH)$_3\downarrow$ + 8H$^+$
\end{center}

Legend:
\begin{itemize}
    \item[$\uparrow$] Release of gas (hydrogen bubbles);
    \item[$\downarrow$] Creation of precipitate (floc).
\end{itemize}

\subsubsection{Potential for ARA Rhein}
Currently, ARA Rhein does not utilize electrocoagulation.
By installing EC reactors between the buffer tanks and flocculation
units, the water cleaning effectiveness would increase. Additionally,
it would improve the separation quality of heavy metals and
pharmaceutical residues. Furthermore, because EC avoids the addition
of counter-ions (like sulfate or chloride found in chemical salts),
it has the potential to produce a denser, more stable floc, thereby
reducing the overall sludge volume, lowering costs, and increasing
capacity.

\section{Conclusion}
In conclusion, reducing the number of pollutants in wastewater is
crucial before reuse or environmental release to minimize
contamination and eutrophication. As seen in this study, the process
of purifying wastewater can vary depending on the nature of the
pollutants present. 

The ARA Rhein case study illustrates two main separation techniques
and a potential third method: flocculation, precipitation, and
electrocoagulation. Precipitation first converts dissolved pollutants
into insoluble solids, followed by flocculation, which aggregates
these fine particles into larger flocs for easier separation.
Furthermore, electrocoagulation was examined as a potential
additional step. 

With the inclusion of electrocoagulation, ARA Rhein could improve the
removal of complex contaminants like heavy metals and pharmaceutical
residues. However, it is important to note that sludge is a
significant by-product of these processes and requires careful
handling and disposal.

\newpage
\pagestyle{plain}
\listoffigures

\printbibliography

\section*{Declarations on the use of AI tools}
\begin{itemize}
    \item ``\textit{ChatGPT 5.1}'' was used to enhance vocabulary.\\
        {\color{darkgray!95}{\textit{All original sentences originate from our own ideas and were refined with the support of this tool.}}}\\
        \url{https://chatgpt.com/}
    \item ``DeepL'' was used as a spell-checker.\\
        \url{https://www.deepl.com}
    \item ``Google Gemini'' was used as a fact-checker.\\
        \url{https://gemini.google.com/app}
\end{itemize}

\newpage
\pagestyle{fancy}
\appendix
\section{Appendix}
\begin{figure}[ht!]
    \centering
    \includegraphics[width=.6\textwidth]{media/wwtp_diagram.png}
    \caption{ARA Rhein WWTP process flow diagram}
    \label{fig:ara_rhein_process}
\end{figure}

\begin{figure}[ht!]
    \centering
    \includegraphics[width=.6\textwidth]{media/ARA_Rheinpng.png}
    \caption{ARA Rhein wastewater treatment plant aerial view}
    \label{fig:ara_rhein_aerial}
\end{figure}

\end{document}