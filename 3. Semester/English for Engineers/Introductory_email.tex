\documentclass{article}

\usepackage{Paper}
\pdftitle{Matteo_Frongillo_Introductory_email}

\title{\textbf{EEG personal introduction}}
\date{}
\geometry{top=5mm, bottom=20mm}

\begin{document}
\maketitle
\vspace*{-1.5cm}

Dear Petrushka,
\setlength{\parskip}{1.5em}

% your course of studies
I am contacting you to share my scholastic background and my actual student life.
\setlength{\parskip}{.3em}

Since last year, I have been enrolled to the Lucerne University of Applied Sciences, where I
am studying the faculty of Energy and Environmental Systems Engineering in order to
become an energy engineer.
\setlength{\parskip}{1em}

% your professional background: apprenticeships and/or internships, work experience 
My school experience started in Claro, a small village located at north of Bellinzona, the
capital of Canton Tessin. There, I have attended the kindergarten and the primary school,
whereas my secondary school took place in Castione.
Dithering about the choice to apply for a high school or an apprenticeship, I jumped into
the latter, starting my first job as a Building Services Designer. The job gave me the
choice to choose between three specializations: heating, ventilation, or plumbing systems,
and I stuck with heating systems. After a long consideration, I also decided to undergo
baccalaureate with the technical matura besides my four working years.
As soon as my apprenticeship ended, I served in the army in the CBRN
(Chemical, Biological, Radiological, and Nuclear) Decontamination Defence recruit school.
Although I did not enjoy the behavior of people in the military, I indeed loved my service.
Since the recruit school ended in November, instead of applying for a school, I decided to
convert my remaining days of military service into civil service.
I had the valuable opportunity to work six months as Research Collaborator in Renewable Energy
at the University of Applied Sciences of Southern Switzerland, in the honorable Institute
of Applied Sustainability to the Built Environment, where I have learnt the theory and the
practical application of photovoltaic modules, finding the motivation to pursue further
university education.
Fascinated by physics, my goal was to study nuclear physics at EPFL, hence I applied for
the passerelle program, which allows students with a technical matura to have access to
universities and polytechnics. However, that program was not fulfilling my expectations,
so I decided to enroll in an applied science university, for instance the HSLU.

% English language foundations including travel, studies and/or work (abroad)
English in Tessin is taught from the third year of secondary school; pretty late in my opinion.
It was not until the year of passerelle that I started to appreciate the language. I had
the chance to study with Delia, an extraordinary student with advanced knowledge in English,
to whom I am very grateful for tutoring me throughout the whole year.
Although I did not particularly enjoy the passerelle year, it gave me the determination to
continue my studies in English.
The hardest part of my journey was obtaining the C1 certificate in English in order
to be accepted by the university, which I finally achieved as a private candidate thanks to
the IELTS exam in May 2025.
Nowadays, I am doing my best to improve my skills, especially in speaking and listening,
which are my weakest points.

% your particular interests in the course
I find all the lexical topics of the course very interesting, especially materials and
energy production, which are closely related to my field of study.
I believe that English for Engineers will be a great opportunity to enhance my
aforementioned weaknesses.
I acknowledge that my vocabulary is still quite limited, consequently my speaking skills
are limited due to the difficulty to find the right words. I consider communication one of
the most important skills in professional life, and I would like to have a good command of it.

% projects and main topics you are working on currently, this semester
My faculty is split into energy systems and environmental systems, and so is my current
semester. The main subjects I am studying are related to sustainable environmental
systems, including waste management, water treatment, and air pollution control, and
to thermodynamics, such as gas dynamics and heat transfer.
I am also part of a project team of nine students for the module PDP1, Production
Development Project 1, where we are discussing solutions to introduce the topic of
energy production and waste management to children between six and sixteen years old who
will attend a summer camp of one week.

% your academic and professional interests and goals
My professional goal is to become an energy engineer, working in the field of renewable
energy, especially within the thermal systems sector. I am also particularly interested
in photovoltaic systems and energy storage, which are growing fields with a lot of
potential.
Gaining the bachelor's degree in Energy and Environmental Systems Engineering is the first
step to achieve my goal, and I am determined to do my best to succeed.

% closing and greetings
I believe this long introduction shows my background and motivation, and I look forward
to growing as both a student and a future engineer.
\setlength{\parskip}{1.5em}

Best regards,\\
Matteo













\end{document}