\documentclass[12pt,a4paper]{report}

\usepackage[a4paper, margin=1in]{geometry}
\usepackage{amsmath,amssymb}
\usepackage{graphicx}
\usepackage[hidelinks]{hyperref}
\usepackage{enumitem}
\usepackage{parskip} 
\usepackage{titlesec}

%%%%%%%%%%%%%%%%%%%%%%%%%%%%%%%%%%%%%%%%%%%%%%%%%%%%%%%%%%%%%%%%%%%%%%%%%%%%%%%
% Title Page
%%%%%%%%%%%%%%%%%%%%%%%%%%%%%%%%%%%%%%%%%%%%%%%%%%%%%%%%%%%%%%%%%%%%%%%%%%%%%%%

\title{Marketing Management\\HSLU, Semester 2}
\author{Matteo Frongillo}
\date{}

\begin{document}
\maketitle
\tableofcontents
\newpage

%%%%%%%%%%%%%%%%%%%%%%%%%%%%%%%%%%%%%%%%%%%%%%%%%%%%%%%%%%%%%%%%%%%%%%%%%%%%%%%
\chapter{Course Overview and Organizational Aspects}
%%%%%%%%%%%%%%%%%%%%%%%%%%%%%%%%%%%%%%%%%%%%%%%%%%%%%%%%%%%%%%%%%%%%%%%%%%%%%%%

\section{Administrative Details and Course Structure}
\begin{itemize}
    \item \textbf{Course Title:} Introduction to Marketing Management
    \item \textbf{Institution:} Institut f\"ur Innovation und Technologie Management
    \item \textbf{Script composed by:} Markus Raschke
    \item \textbf{Important Dates:}
    \begin{itemize}
         \item Semester Start: 18.02.2025
         \item Semester Performance Test: SW09 (16.04.25)
         \item Mock--Exam: SW14 (21.05.2025)
         \item Final Exam: Date TBD by university administration
    \end{itemize}
    \item \textbf{Lecture Format:}
    \begin{itemize}
         \item Lectures are held in--person (not streamed or recorded)
         \item Additional materials and scripts are available on ILIAS
    \end{itemize}
    \item \textbf{Study Guidelines:}
    \begin{itemize}
         \item Actively participate, ask questions, and challenge the content.
         \item Self--study and review the previous lecture’s content before each session.
         \item Engage in group work and discussions as assigned.
    \end{itemize}
\end{itemize}

\section{Course Content and Learning Objectives}
\begin{itemize}
    \item The course covers:
    \begin{itemize}
         \item Fundamentals of Relationship Marketing
         \item Situational Analysis in Marketing Planning
         \item Strategy Formulation (including SWOT, mission, objectives, and portfolio models)
         \item Market Segmentation, Targeting, and Positioning (STP)
         \item Marketing Mix Decisions (Product, Pricing, Distribution, Communication)
         \item Corporate Social Responsibility (CSR) strategies
    \end{itemize}
    \item \textbf{Learning Objective:} Equip students with both the theoretical foundation and practical skills to design and implement strategic marketing plans.
\end{itemize}

%%%%%%%%%%%%%%%%%%%%%%%%%%%%%%%%%%%%%%%%%%%%%%%%%%%%%%%%%%%%%%%%%%%%%%%%%%%%%%%
\chapter{Fundamentals of Relationship Marketing}
%%%%%%%%%%%%%%%%%%%%%%%%%%%%%%%%%%%%%%%%%%%%%%%%%%%%%%%%%%%%%%%%%%%%%%%%%%%%%%%

\section{Evolution of Marketing and Key Definitions}
\begin{itemize}
    \item \textbf{Traditional Marketing:} Focuses on transactions—selling a product to a customer.
    \item \textbf{Relationship Marketing:} Emphasizes building long--term relationships with customers to create lifetime value.
    \item \textbf{Key Quote:} 
        \begin{quote}
            ``Education is not the learning of facts, but training the mind to think.'' 
            \\ \hfill ---- Albert Einstein
        \end{quote}
    \item \textbf{Drucker’s Insight:} Business has only two functions ---- marketing and innovation.
\end{itemize}

\section{The Marketing Concept}
\begin{itemize}
    \item \textbf{Definition:} Achieving corporate goals through meeting and exceeding customer needs better than the competition.
    \item \textbf{Core Elements:}
    \begin{enumerate}[label=\alph*.]
        \item \textbf{Customer Orientation:} Every corporate activity is geared toward customer satisfaction.
        \item \textbf{Integrated Effort:} All staff are responsible for creating customer value.
        \item \textbf{Goal Achievement:} Corporate success follows customer satisfaction.
    \end{enumerate}
\end{itemize}

\section{Transactional vs. Relationship Marketing}
\begin{itemize}
    \item \textbf{Transactional Approach:} Emphasizes the exchange process (sales, one--time interactions).
    \item \textbf{Relationship Approach:} Focuses on continuous engagement, trust, and long--term customer loyalty.
    \item \textbf{Relationship Ladder:} 
    \begin{enumerate}[label=\arabic*.]
        \item Awareness -- Recognizing a potential exchange partner.
        \item Exploration -- Testing the exchange with trial interactions.
        \item Expansion -- Growing interdependency and benefits.
        \item Commitment -- Formal or informal promise of continued exchange.
        \item Dissolution -- Eventual disengagement, which always remains a possibility.
    \end{enumerate}
    Companies can develop specific strategies to nurture customers from initial awareness (through informational and engaging content) to commitment (with personalized services or loyalty programs). This progression helps create sustainable, long-term customer relationships.
\end{itemize}

\section{Relationship Economics}
\begin{itemize}
    \item \textbf{Objective:} Create sustainable profitability via customer retention.
    \item A stronger relationship typically leads to higher lifetime customer value.
\end{itemize}

%%%%%%%%%%%%%%%%%%%%%%%%%%%%%%%%%%%%%%%%%%%%%%%%%%%%%%%%%%%%%%%%%%%%%%%%%%%%%%%
\chapter{Situational Analysis in Marketing Planning}
%%%%%%%%%%%%%%%%%%%%%%%%%%%%%%%%%%%%%%%%%%%%%%%%%%%%%%%%%%%%%%%%%%%%%%%%%%%%%%%

\section{Overview of the Marketing Planning Process}
\begin{itemize}
    \item \textbf{Purpose:} To determine how to provide value to customers and capture that value in return.
    \item \textbf{Main Steps:}
    \begin{enumerate}[label=\arabic*.]
         \item Research and analyze the current situation.
         \item Develop and document objectives, strategies, and programs.
         \item Implement, evaluate, and control marketing activities.
    \end{enumerate}
\end{itemize}

\section{Market Driven vs. Resource Driven Strategies}
\subsection{Resource--Based View (RBV)}
\begin{itemize}
    \item Focus on \textbf{core competencies} and unique, hard--to--copy resources.
    \item Emphasizes internal capabilities as the source of superior performance.
\end{itemize}

\subsection{Market Orientation View (MOV)}
\begin{itemize}
    \item Focus on understanding customer needs and competitive forces.
    \item Emphasizes external market trends and a customer--centric approach.
\end{itemize}

\subsection{Critical Discussion}
\begin{itemize}
    \item A balanced integration of RBV and MOV is essential.
    \item Overemphasis on internal resources may miss market shifts; focusing exclusively on market trends may underutilize unique strengths.
\end{itemize}

\section{Assessing the Internal Marketing Situation}
\begin{itemize}
    \item \textbf{Resources:}
    \begin{itemize}
         \item \textbf{Tangible:} Physical assets, financial resources, technology.
         \item \textbf{Intangible:} Human resources, innovation capability, brand reputation.
    \end{itemize}
    \item \textbf{Core Competencies:} 
         \begin{itemize}
             \item Difficult to duplicate.
             \item Provide access to multiple markets.
             \item Significantly enhance customer value.
         \end{itemize}
    \item \textbf{Marketing Capability Gap:} When identified core competencies do not fully match market needs.
\end{itemize}

\section{Assessing the External Marketing Situation}
\subsection{Macroenvironmental Analysis (PESTEL)}
\begin{itemize}
    \item Factors include:
    \begin{enumerate}[label=\alph*.]
         \item Political/Legal
         \item Economic
         \item Social/Cultural
         \item Technological
         \item Ecological
         \item Legal (often split out for emphasis)
    \end{enumerate}
\end{itemize}

\subsection{Microenvironmental Analysis}
\begin{itemize}
    \item Involves:
    \begin{itemize}
         \item Customers and their buying behavior.
         \item Competitors and their strategies.
         \item Channel members, partners, suppliers, and employees.
    \end{itemize}
\end{itemize}

\subsection{Porter’s Five Forces Analysis}
\begin{itemize}
    \item \textbf{Threat of New Entrants:} Impact of market entry barriers.
    \item \textbf{Threat of Substitutes:} Risks from alternative products or services.
    \item \textbf{Bargaining Power of Buyers:} Influence of consumers on pricing and quality.
    \item \textbf{Bargaining Power of Suppliers:} Influence on production cost and input availability.
    \item \textbf{Competitive Rivalry:} Intensity of competition among existing firms.
\end{itemize}

\section{SWOT Analysis}
\begin{itemize}
    \item \textbf{Internal Factors:}
    \begin{itemize}
         \item Strengths
         \item Weaknesses
    \end{itemize}
    \item \textbf{External Factors:}
    \begin{itemize}
         \item Opportunities
         \item Threats
    \end{itemize}
    \item \textbf{Application:} Use the SWOT matrix to develop matching strategies:
    \begin{itemize}
         \item SO--Strategies, ST--Strategies, WO--Strategies, WT--Strategies.
    \end{itemize}
    \item \textbf{Example:} The Lufthansa SWOT matrix (as provided in the slides).
\end{itemize}

%%%%%%%%%%%%%%%%%%%%%%%%%%%%%%%%%%%%%%%%%%%%%%%%%%%%%%%%%%%%%%%%%%%%%%%%%%%%%%%
\chapter{Strategy Formulation in the Marketing Planning Process}
%%%%%%%%%%%%%%%%%%%%%%%%%%%%%%%%%%%%%%%%%%%%%%%%%%%%%%%%%%%%%%%%%%%%%%%%%%%%%%%

\section{Defining Strategy}
\begin{itemize}
    \item \textbf{Strategy as:}
    \begin{enumerate}[label=\alph*.]
         \item \textbf{Plan:} Formulated in advance.
         \item \textbf{Ploy:} Specific maneuvers to outperform competitors.
         \item \textbf{Pattern:} Emerging consistency in actions.
         \item \textbf{Position:} Finding a niche or unique market stance.
         \item \textbf{Perspective:} The organization’s unique view or culture.
    \end{enumerate}
    \item \textbf{Mintzberg’s 5 P’s:} A framework to understand the multifaceted nature of strategy.
\end{itemize}

\section{The Strategic Marketing Planning Process}
\begin{itemize}
    \item \textbf{Key Questions:}
    \begin{enumerate}[label=\arabic*.]
         \item What business are we in?
         \item Where are we today? (situational analysis)
         \item Where do we want to go? (mission/vision)
         \item How do we get there? (strategy and tactics)
    \end{enumerate}
    \item \textbf{Components:}
    \begin{itemize}
         \item Mission statement and vision.
         \item Situational analysis (SWOT, PESTEL, Porter).
         \item Strategic objectives (financial, customer, operational, people/learning).
         \item Selection of strategic options (market penetration, product development, diversification, etc.).
    \end{itemize}
\end{itemize}

\section{Strategic Objectives and Measurement}
\begin{itemize}
    \item \textbf{Criteria for Objectives:}
    \begin{itemize}
         \item Specific performance dimensions.
         \item Appropriate measures and target values.
         \item Clear time frames.
    \end{itemize}
    \item \textbf{Examples:}
    \begin{itemize}
         \item Increase revenue by 10\% annually.
         \item Expand market share in the mid--market segment.
         \item Improve customer retention metrics.
    \end{itemize}
\end{itemize}

\section{Generic and Competitive Strategies}
\subsection{Ansoff’s Generic Strategies for Growth}
\begin{itemize}
    \item \textbf{Market Penetration:} Increase share in existing markets.
    \item \textbf{Product Development:} Innovate or improve products for current markets.
    \item \textbf{Market Development:} Enter new markets with existing products.
    \item \textbf{Diversification:} Venture into new products and new markets.
\end{itemize}

\subsection{Porter’s Generic Strategies}
\begin{itemize}
    \item \textbf{Cost Leadership:} Achieve the lowest production and distribution costs.
    \item \textbf{Differentiation:} Offer unique products commanding premium prices.
    \item \textbf{Focus Strategies:} Concentrate on a narrow market segment with either cost focus or differentiation focus.
    \item \textbf{Key Insight:} “Don’t get stuck in the middle.”
\end{itemize}

\subsection{Portfolio Management Models}
\begin{itemize}
    \item \textbf{BCG Growth--Share Matrix:} Classify business units/products as Stars, Question Marks, Cash Cows, or Dogs.
    \item \textbf{McKinsey Model:} Consider market attractiveness versus competitive position.
\end{itemize}

\section{Offensive and Defensive Strategies, and CSR}
\begin{itemize}
    \item \textbf{Offensive Strategies:}
         \begin{itemize}
             \item Direct attacks on competitors.
             \item Indirect maneuvers to capture market share.
         \end{itemize}
    \item \textbf{Defensive Strategies:}
         \begin{itemize}
             \item Actions designed to deter potential challengers.
             \item Shaping competitor expectations regarding profitability.
         \end{itemize}
    \item \textbf{Corporate Social Responsibility (CSR):}
         \begin{itemize}
             \item Integrates social and environmental concerns into operations.
             \item Acts as a tool for brand differentiation and risk management.
             \item \textbf{Drivers:} Consumer awareness, legislation, globalization, and investor expectations.
         \end{itemize}
\end{itemize}

%%%%%%%%%%%%%%%%%%%%%%%%%%%%%%%%%%%%%%%%%%%%%%%%%%%%%%%%%%%%%%%%%%%%%%%%%%%%%%%
\chapter{Market Segmentation, Targeting, and Positioning (STP)}
%%%%%%%%%%%%%%%%%%%%%%%%%%%%%%%%%%%%%%%%%%%%%%%%%%%%%%%%%%%%%%%%%%%%%%%%%%%%%%%

\section{Market Segmentation}
\begin{itemize}
    \item \textbf{Definition:} Dividing a broad market into subgroups with shared characteristics.
    \item \textbf{Purpose:} Identify segments likely to be most profitable or have growth potential.
    \item \textbf{Criteria:} Demographics, geographic, behavioral, and psychographic factors.
\end{itemize}

\section{Targeting Strategies}
\begin{itemize}
    \item \textbf{Undifferentiated Marketing:} A single marketing mix for all consumers.
    \item \textbf{Differentiated Marketing:} Different products or messages for different segments.
    \item \textbf{Concentrated Marketing:} Focus on one or a few segments.
    \item \textbf{Micromarketing:} Customization based on local or individual differences.
\end{itemize}

\section{Market Positioning}
\begin{itemize}
    \item \textbf{Goal:} Influence how target customers perceive a brand or product relative to competitors.
    \item \textbf{Tools:}
    \begin{itemize}
         \item Perceptual Maps to visualize competitive positioning.
         \item Positioning statements based on a formula:
         \begin{enumerate}[label=\arabic*.]
             \item For (target customer)
             \item Who (statement of need)
             \item (Brand/Product) is the only (category)
             \item That (offers differentiation/benefit)
         \end{enumerate}
    \end{itemize}
    \item \textbf{Examples:} 
         \begin{itemize}
             \item Harley--Davidson’s positioning as a lifestyle brand.
             \item Volvo’s positioning focused on safety.
         \end{itemize}
\end{itemize}

%%%%%%%%%%%%%%%%%%%%%%%%%%%%%%%%%%%%%%%%%%%%%%%%%%%%%%%%%%%%%%%%%%%%%%%%%%%%%%%
\chapter{Marketing Mix in the Marketing Planning Process}
%%%%%%%%%%%%%%%%%%%%%%%%%%%%%%%%%%%%%%%%%%%%%%%%%%%%%%%%%%%%%%%%%%%%%%%%%%%%%%%

\section{Overview of the 4 P's}
\begin{itemize}
    \item \textbf{Product:} What is offered; includes features, branding, and quality.
    \item \textbf{Price:} How much customers pay; requires an understanding of value and competitive positioning.
    \item \textbf{Place:} Distribution channels and logistics.
    \item \textbf{Promotion:} Communication and advertising strategies.
\end{itemize}

\section{Product and Service Decisions}
\begin{itemize}
    \item \textbf{Product Levels:}
         \begin{enumerate}[label=\alph*.]
             \item \textbf{Core Product:} The essential benefit.
             \item \textbf{Actual Product:} Tangible features, design, brand image.
             \item \textbf{Augmented Product:} Additional services, warranties, or support.
         \end{enumerate}
    \item \textbf{New Product Development:} Stages from idea generation to commercialization.
    \item \textbf{Product Life Cycle:} Introduction, growth, maturity, and decline.
\end{itemize}

\section{Pricing, Distribution, and Communication}
\begin{itemize}
    \item \textbf{Pricing Strategies:}
         \begin{itemize}
             \item Penetration pricing, skimming, competitive pricing.
         \end{itemize}
    \item \textbf{Distribution (Place):}
         \begin{itemize}
             \item Direct vs. indirect channels, channel management, geographic considerations.
         \end{itemize}
    \item \textbf{Promotion (Communication):}
         \begin{itemize}
             \item Advertising, public relations, digital marketing, and integrated campaigns.
         \end{itemize}
\end{itemize}

\section{Customer value in the context of marketing decisions}
\subsubsection{Integration of the 4 P's with customer needs}
Explain how each element of the marketing mix is tailored to meet specific customer needs. For example, designing a product that satisfies customers, pricing it appropriately to reflect perceived value, ensuring convenient distribution, and effective promotion that communicates the benefits.

\subsubsection{Dynamic customer value}
Elaborate on how customer value isn’t static but evolves as customer expectations change, and companies must continuously refine their marketing strategies to maintain satisfaction.

\subsubsection{Win-win situation}
Detail that a win-win occurs when the company's competitive advantage translates into customer loyalty and satisfaction, which, in turn, supports long-term profitability and market positioning.
(Pages 40--42)

%%%%%%%%%%%%%%%%%%%%%%%%%%%%%%%%%%%%%%%%%%%%%%%%%%%%%%%%%%%%%%%%%%%%%%%%%%%%%%%
\chapter{Case Studies, Examples, and Group Work Exercises}
%%%%%%%%%%%%%%%%%%%%%%%%%%%%%%%%%%%%%%%%%%%%%%%%%%%%%%%%%%%%%%%%%%%%%%%%%%%%%%%

\section{Group Work and In--Class Exercises}
\begin{itemize}
    \item \textbf{SWOT Analysis Exercises:} 
         \begin{itemize}
             \item Use templates (e.g., the Lufthansa SWOT matrix) for hands--on practice.
         \end{itemize}
    \item \textbf{Competitive Strategy Group Work:}
         \begin{itemize}
             \item Compare market--driven and market--driving strategic approaches.
         \end{itemize}
    \item \textbf{CSR Case Study:}
         \begin{itemize}
             \item Analyze companies such as Coca--Cola, Volkswagen, or Nikin regarding their CSR initiatives.
         \end{itemize}
\end{itemize}

\section{Illustrative Examples}
\begin{itemize}
    \item \textbf{Burger King’s Mission Statement:} “We offer reasonably priced quality food, served quickly, in attractive, clean surroundings.”
    \item \textbf{Harley--Davidson’s Positioning:} Emphasizes a unique lifestyle and strong brand identity.
    \item \textbf{Porter’s Five Forces in Real Situations:} Detailed examples highlight how market entry, buyer power, and competitive rivalry interact.
\end{itemize}

%%%%%%%%%%%%%%%%%%%%%%%%%%%%%%%%%%%%%%%%%%%%%%%%%%%%%%%%%%%%%%%%%%%%%%%%%%%%%%%
\chapter{Consolidated Reflections and Conclusions}
%%%%%%%%%%%%%%%%%%%%%%%%%%%%%%%%%%%%%%%%%%%%%%%%%%%%%%%%%%%%%%%%%%%%%%%%%%%%%%%

\section{Integrated Marketing Management}
\begin{itemize}
    \item Comprehensive marketing management is a dynamic interplay among:
    \begin{itemize}
         \item Relationship building
         \item Rigorous situational analysis
         \item Strategic planning and execution
         \item Tactical adjustments based on performance gaps
    \end{itemize}
\end{itemize}

\section{Key Takeaways for the Exam}
\begin{itemize}
    \item Every phase—from internal audits to external market analyses—impacts strategic decisions.
    \item The synergy of customer orientation, innovative product strategies, and CSR initiatives can create a sustainable competitive advantage.
    \item Continuous evaluation (via performance measurement and feedback loops) is critical to closing the planning gap.
\end{itemize}

\section{Future Directions and Research}
\begin{itemize}
    \item Emerging trends: Digital transformation, ethical consumerism, and globalization.
    \item Importance of lifelong learning: Always update strategic tools and frameworks as market dynamics evolve.
\end{itemize}

%%%%%%%%%%%%%%%%%%%%%%%%%%%%%%%%%%%%%%%%%%%%%%%%%%%%%%%%%%%%%%%%%%%%%%%%%%%%%%%
\chapter{Appendices and Supplementary Material}
%%%%%%%%%%%%%%%%%%%%%%%%%%%%%%%%%%%%%%%%%%%%%%%%%%%%%%%%%%%%%%%%%%%%%%%%%%%%%%%

\appendix

\chapter{Detailed Definitions and Key Concepts}
\begin{itemize}
    \item \textbf{Core Competencies:} Should be difficult to duplicate, provide access to a broad range of markets, and significantly add to customer value. (Prahalad and Hamel, 1990)
    \item \textbf{PESTEL Components:} Detailed definitions for each factor and its implications for strategic decision--making.
    \item \textbf{The Competitive Triangle:} How resources, market conditions, and capabilities interact.
\end{itemize}

\chapter{Full Transcriptions of Selected Slides (Excerpted)}
\begin{itemize}
    \item \textbf{Slide Excerpts:}
    \begin{itemize}
         \item ``Production orientation: Production capabilities are the basis for company activity.'' 
         \item ``Marketing orientation: Customer needs are the basis for all company activity.'' 
         \item ``Sales (Push) vs. Marketing (Pull): Transactional marketing versus relationship--driven approaches.'' 
         \item Detailed breakdown of the relationship ladder: Awareness, Exploration, Expansion, Commitment, Dissolution.
    \end{itemize}
\end{itemize}

\chapter{Group Work Templates and Discussion Prompts}
\begin{itemize}
    \item SWOT analysis templates.
    \item Guidelines for group presentations on CSR strategies and competitive analysis.
    \item Template for perceptual mapping exercises.
\end{itemize}

\end{document}
