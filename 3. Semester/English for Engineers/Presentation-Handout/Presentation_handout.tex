\documentclass{article}

\usepackage{Paper}
\pdftitle{Frongillo-Moos-EENG-Handout}

% === TITLE ===
\title{\textbf{Gravity Energy Storage Systems (GESS) \\ EENG Handout Presentation, HS25}}
\author{Matteo Frongillo, Michael Moos}
\date{\today}

% === TEXT ===
\begin{document}
\newgeometry{
    a4paper,
    total={170mm,257mm},
    left=20mm,
    top=0mm,
    bottom=20mm
}
\hypersetup{citecolor=black}
\maketitle
\linespread{1.2}\selectfont

In Switzerland, potential energy storage is widely used for producing
green energy in the form of pumped hydro plants. Due to the limited
space in cities, Gravity Energy Storage Systems (GESS) are an efficient
alternative for the generation of clean power \parencite{franklin2022}.
In order to make this type of storage possible, the company Energy Vault$^{\textregistered}$
developed the G-VAULT\texttrademark, using excess green electricity to lift heavy masses
and later releasing it by lowering them to generate electricity. This process is based on
the conversion between potential and electrical energy \parencite{energyvault2023}.

\vspace{-0.3cm}
\begin{wrapfigure}{l}{.5\textwidth}
    \includegraphics[width=.5\textwidth]{media/EV_prototype.png}
    \caption{Prototype 1 in Arbedo-Castione, Switzerland}
\end{wrapfigure}

\phantom{}

Storage systems based on gravity rely on the reversible conversion between
potential and electrical energy. The amount of energy that can be stored depends
directly on the mass lifted and the height difference, therefore,
larger masses and greater lifting heights result in higher energy.
With renewable energy, electric motors lift the mass, whether it is solid blocks or
a fluid. During periods of high demand, the mass is lowered, producing kinetic
and consequently electrical energy.
\wrapfill

\vspace*{-1.5cm}

\begin{wrapfigure}{r}{.4\textwidth}
    \vspace*{-5cm}
    \includegraphics[width=.4\textwidth]{media/EVx.png}
    \caption{EVu\texttrademark\ prototype in China}
\end{wrapfigure}

\vspace*{-0.3cm}
Energy Vault$^{\textregistered}$ developed the G-VAULT\texttrademark, one
of the first large-scale implementations of gravity energy storage. The system
uses a crane tower structure that allows 35-tonne blocks to be lifted and
lowered. Prototype 1 in Arbedo-Castione has demonstrated the feasibility
of the concept under real operating conditions, reaching high round-trip
efficiencies of around 80\% and long lifespans with minimal environmental
impact \parencite{mombelli2020}.
\wrapfill

\vspace*{-1.25cm}
Modern GESS installations, such as the EVu\texttrademark\ prototype currently
under construction in China, represent an important step toward large scale
gravity storage capable of supporting large energy grids \parencite{EnergyVault2024Rudong}.
With the advent and widespread use of renewable energy worldwide, GESS represents
a sustainable and viable option for generating electricity in cities with high
green energy demand or in countries with limited grid accessibility.

\newpage
\restoregeometry
\pagestyle{empty}
\section*{Word Bank}
\begin{table}[ht!]
    \renewcommand{\arraystretch}{2}
    \centering
    \begin{tabularx}{\textwidth}{|l|X|}
        \hline \textbf{Term} & \textbf{Definition}\\
        \hline Potential energy & Energy stored in a system due to its mass and position\\
        \hline Kinetic energy & Energy associated with the motion of an object due to its velocity\\
        \hline Electrical energy & Energy carried by electric charges moving through a conductor or field\\
        \hline Reversible conversion & Bidirectional energy transformation with minimal losses\\
        \hline Energy storage & The process of capturing energy for later use\\
        \hline GESS & Gravity Energy Storage System that stores energy by lifting and lowering masses\\
        \hline Height difference & The vertical distance between two points that enables potential energy change\\
        \hline Crane tower structure & A vertical mechanical system that lifts and lowers masses\\
        \hline Green energy & Energy produced from sources that minimize environmental harm\\
        \hline Renewable energy & Energy that derives from natural resources that replenish on a human timescale\\
        \hline Clean power & Electricity generated with minimal greenhouse gas emissions\\
        \hline Electric motor & A device that converts electrical energy into mechanical motion\\
        \hline Solid & A state of matter with fixed shape and volume\\
        \hline Fluid & A substance in a liquid or gaseous state that flows and takes the shape of its container\\
        \hline Pumped hydro plants & Facilities that store energy by pumping water to different heights to generate electricity\\
        \hline Real operating conditions & The actual environmental and technical circumstances under which a system operates\\
        \hline Round-trip & The complete charge-discharge cycle of a system\\
        \hline Lifespan & The period during which a system remains functional according to standards\\
        \hline Environmental impact & The effect of a technology on natural ecosystems and resources\\
        \hline Energy grid & The interconnected network that transports and distributes electrical power from producers to consumers\\
        \hline
    \end{tabularx}
\end{table}

\newpage
\section*{Declarations on the use of AI tools}
\begin{itemize}
    \item ``DeepL'' and ``ChatGPT 5'' have been used as a spell-checker \\
        \url{https://www.deepl.com/}\\
        \url{https://www.chatgpt.com/}
\end{itemize}

\listoffigures

\setlength{\bibitemsep}{1.2\baselineskip}
\printbibliography[title={References}]

\end{document}
