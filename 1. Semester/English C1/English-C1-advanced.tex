\documentclass{article}

\usepackage[fleqn]{amsmath}
\usepackage{amssymb}
\usepackage{hyperref}
\usepackage{url}
\usepackage{graphicx}
\usepackage{geometry}
\usepackage{babel}
\usepackage{enumitem}
\usepackage{parskip}
\usepackage{chemfig}
\usepackage{pdfpages}
\usepackage{xcolor}
\usepackage{tikz}
\usepackage{fancybox}
\usepackage{makecell}
\usepackage{pgfplots}
\usepackage{soul}
\usepackage{ulem}
\usepackage{wrapfig}
\usepackage{subcaption}
\usepackage[T1]{fontenc}
\usepackage{esvect}
\usetikzlibrary{arrows}
\usetikzlibrary{decorations.pathreplacing}
\pgfplotsset{compat=1.17}

\geometry{
    a4paper,
    total={170mm, 257mm},
    left=20mm,
    top=20mm
}

\hypersetup{
    colorlinks=true,
    linkcolor=black,
    urlcolor=blue,
    pdftitle={System engineering in EES}
}

\newcommand{\figbox}[1]{ 
    \begin{figure*}[ht!]        
        \begin{center}            
            \fbox{#1}        
        \end{center}    
    \end{figure*}
}

\newcommand{\wrapfill}{
    \par
    \ifnum \value{WF@wrappedlines} > 0
        \addtocounter{WF@wrappedlines}{-1}%
        \null\vspace{
            \arabic{WF@wrappedlines}
            \baselineskip
        }
        \WFclear
    \fi
    \phantom{}
}

\newcommand{\cfig}[1]{%
  \begin{figure*}[ht!]%
    \centering%
    #1%
  \end{figure*}%
}

\newcommand{\difference}{\,\backslash\,}
\newcommand{\rem}{\underline{Remark}: }
\newcommand{\nots}{\underline{Notation}: }
\newcommand{\prf}{\underline{Proof}: }
\newcommand{\exs}{\underline{Example}: }
\newcommand{\defs}{\underline{Definition}: }
\newcommand{\wrn}{\underline{Warning}: }
\newcommand{\sht}{\ |\ }
\newcommand{\pph}[1]{\paragraph{#1} \phantom{}\\}
\newcommand{\tox}{$\longrightarrow$\ }
\newcommand{\red}[1]{%
    {\color{red}{#1}}%
}

% === TEXT ===
\title{\textbf{English C1 Advanced\\ HSLU, Semester 1}}
\author{Matteo Frongillo}
\date{\today}

\begin{document}

\maketitle
\tableofcontents
\pagebreak

\section{Course overview}
\subsection{Exam overview}
\begin{itemize}
    \item Use of English and Reading: \textbf{MEP (30\%)};
    \item Writing: \textbf{MEP (30 \%)};
    \item Listening: \textbf{Semester performance (20\%)};
    \item Speaking: \textbf{Semester performance (20\%)}.
\end{itemize}


\newpage
\section{Past tenses}

\subsection{Simple tenses}
\subsubsection{Past simple}
Past simple is used for:
\begin{itemize}
    \item Finished past event at a specific past point in time. 
\end{itemize}

\subsection{Continuous tenses}
Are focused on ongoing actions at past or present time.

\subsubsection{Past continuous}
Past continuous is used to:
\begin{itemize}
    \item Express something was ongoing at a specific past point;
    \item Focus on an ongoing action in the past that is ``crossed''
        by another past action.
\end{itemize}

\subsection{Perfect tenses}
Most often an action stretching over time that started in the past
and continues either up until now (present perfect) or up until a
past point (past perfect)

\subsubsection{Past perfect simple}
Past perfect simple is used to:
\begin{itemize}
    \item Express something happened before another past time;
    \item Compare two past events.
\end{itemize}

\subsection{Perfect and continuous tenses}
\subsubsection{Past perfect continuous}
Past perfect continuous is used for:
\begin{itemize}
    \item Past actions in progress up until another past point;
    \item Past actions in progress that are interrupted or unfinished.
\end{itemize}

\section{Passive forms}
Passive is used to say what happened to the subject.

Passive sentences are formed with ``to be'' in the appropriate tense
+ the past participle (+ed).

\subsection{Impersonal passive}
\subsubsection{Form}
When we use one of these verbs:

believe, claim, report, say, think, understand, know, consider,
estimate, expect, be rumoured, be reputed, allege;

we have to structure the sentence as follow:\\

\begin{flushleft}
    \begin{tabular}{|m{0cm}>{\centering\arraybackslash}m{1cm}|>{\centering\arraybackslash}m{4.5cm}|>{\centering\arraybackslash}m{6cm}|>{\centering\arraybackslash}m{2cm}|}
        \hline \rule{0pt}{15pt}
        & It + & \textbf{"be"} + (required tense) & \textbf{reporting verb} + (past participle) & that... \\ 
        \hline
    \end{tabular}
\end{flushleft} \phantom{}

\begin{flushleft}
    \begin{tabular}{|>{\centering\arraybackslash}m{1.5cm}|>{\centering\arraybackslash}m{4cm}|>{\centering\arraybackslash}m{5.5cm}|>{\centering\arraybackslash}m{4cm}|}
        \hline
        subject + & \textbf{"be"} + (required tense) & \textbf{reporting verb} + (past participle) & main verb (to-infinitive $\rightarrow$ present or past) \\ 
        \hline
    \end{tabular}
\end{flushleft}

\section{Linking words}

\section{Inversion}
After certain words and phrases the \textbf{word order is inverted}.
This kind of inversion is mainly found in formal speech and writing.

\subsection{Application of the inversion}
\subsubsection{Negation adverbs}
The negative adverbs \textbf{never (before/again), rarely, seldom,
barely/hardly/scarcely...when/before, no sooner...than, nowhere,
little (with a negative meaning)}.

\subsubsection{Negation}


\section{Formal letters}

\newpage
\section{Reported speech}
\subsection{Verb tenses}
We usually change the tense of the original verb so that it moves
further back in the past. We also change time expressions and pronouns
as necessary:

\textit{``We spoke to him yesterday'', they said.\tox
They said that they'd spoken to him the day before.}

We \textbf{do not} change the tense if the situation we are reporting still
exists and if the reporting verb is in the present tense:

\textit{``She's currently working in London'' \tox
He says she's currently working in London.}

\subsection{Modal verbs in reporting}
We usually change modal verbs in reported speech:
\begin{itemize}
    \item \textbf{will \tox would};
    \item \textbf{can \tox could};
    \item \textbf{may \tox might};
    \item \textbf{needn'n \tox didn't have to};
    \item \textbf{must \tox had to}.
\end{itemize}

We \textbf{do not} change modal verbs if the situation we are
reporting still exists and if the reporting verb is in the present
tense:

\textit{``We need to visit oue cousin'' \tox She says we need to
visit our cousin.}

Modal verbes are often reported using other verbs:
\begin{itemize}
    \item must, should, ought to \tox advised, urged;
    \item let's \tox suggested.
\end{itemize}

\textit{``You should ask for help'' \tox He advised me to ask for help.}

\subsection{Reported questions}
\subsubsection{Reported Yes/No questions}
When there is no question word in the direct speech question, we use
\textbf{if/whether}. The word order is the same as in the statement.
The verb tense and other changes are the same as for other types of
reported speech:

\textit{``Could I borrow your notes'' she asked \tox
She asked / wondered / wanted to know \textbf{if / whether she could
borrow} my notes.}

\subsubsection{Reported wh- questions}
The \textbf{wh-} word is followed by normal word order (subject + verb).
The verb tense and other changes are the same as for other types of
reported speech:

\textit{``Why did you leave that job?'' She asked him \tox
She asked \textbf{him why he had left} that job.}

\subsection{Difference between review and report}
\subsubsection{Review}
A review is an unasked paragraph pointed to the customers of a local
or a object and is normally written with an informal language.

\subsubsection{Report}
Is generally an asked paragraph pointed to the manufactor of a product or the
owner of a local, and talks about the quality and what can be improved.

\subsection{Summary reports}
We can use some reporting verbs to summarize what was said:

\begin{enumerate}
    \item \textit{``Don't come back -- or else'' \tox They \textbf{threatened} us};
    \item \textit{``It was me. I did it'' \tox He \textbf{confessed}}.
\end{enumerate}

Some verbs, such as \textbf{speak}, \textbf{tell} and \textbf{thank}, are only used in summary
reports, not with direct or indirect speech:

\begin{enumerate}
    \item \textit{She \textbf{spoke} briefly to reporters.}
    \item \textit{I \textbf{talked} to Kevin about the problem and he \textbf{thanked} me.}
\end{enumerate}

We can use reporting verbs such as \textbf{boast} or \textbf{lie} + \textbf{about}
to create a summary report:

\begin{enumerate}
    \item \textit{He \textbf{boasted about} his win};
    \item \textit{He \textbf{lied about} how he did it}.
\end{enumerate}

Other verbs used like this include: \textbf{complain, explain, inquire,
joke, protest, speak, write}.


\newpage
\section{Conditionals}

\subsection{Conditional 0}
Conditional zero is used to express a fact, something that is always true.

In conditional zero we can use either ``\textbf{If}'' or ``\textbf{When}'' as preposition,
only if the probability is 100\%:
\figbox{If/When + Present simple \tox Present simple}

i.g.:
\begin{enumerate}
    \item If demand for a product \textit{rises}, its price \textit{rises} too;
    \item When demand for a product \textit{rises}, its price \textit{rises} too.
\end{enumerate}


\subsection{Conditional 1}
Conditional 1 is used to express a present/future situation of highly probability. 

The highly probability is gived in the ``if'' clause, not in the ``consequence'' clause.
\figbox{If + Present simple \tox Future simple (\textit{will do})}

ig.:
\begin{enumerate}
    \item If I \textit{see} her tomorrow, I \textit{will speak} to her;
    \item I \textit{will not let} them in if they \textit{are} late again.
\end{enumerate}

\subsection{Conditional 2}
Conditional 2 is used to express a present/future situation of low probability.

It is used to express a zero probability sentence (hypotesis).
\figbox{If + Past simple \tox would do}

i.g.:
\begin{enumerate}
    \item If I \textit{saw} her tomorrow, I \textit{would speak} to her;
    \item If I \textit{had} some time, I \textit{would tidy up} my office.
\end{enumerate}

\wrn{The correct form of the verb ``\textbf{to be}'' in Conditional 2
is always ``\textbf{were}''.}

\subsection{Conditional 3}
Conditional 3 is used to express a past situation, when is too late to change
something or there are zero possibilities to change it.
\figbox{If + Past Perfect \textit{had done} \tox would have done}

i.g.:
\begin{enumerate}
    \item If I \textit{had seen} her, I \textit{would have spoken} to her;
    \item I \textit{wouldn't hate let} them in if they \textit{had been} late.
\end{enumerate}

\newpage
\subsection{Special cases}
\subsubsection{Use of ``would'' for politeness}
If you follow me please, I'll show you your room.

\hspace*{3.6cm}$\downarrow$

If you \textit{would follow} me, I'll show you your room.

\subsubsection{Imperative and requests in Conditional 1}
If you see John, please give him this book.

\hspace*{3.6cm}$\downarrow$

\textit{Could} you give John this book please if you see him?

\subsubsection{Use of ``may, might, could''}
These can be used in the ``consequence'' clause to reduce the centainty
of the action in that clause:

\pph{Compare (Conditional 1)}
\wrn{``\textbf{may}'' is used only in Conditional 1 sentences.}

If I see her tomorrow, I will speak to her.

\hspace*{3cm}$\downarrow$

If I see her tomorrow, I \textit{may} speak to her.

\pph{Compare (Conditional 2)}
If they were late again, I would not let them in.

\hspace*{3.5cm}$\downarrow$

If they were late again, I \textit{might} not let them in.

\pph{Compare (Conditional 3)}
If I had had some time last week, I would have tidied up my office.

If I had had some time last week, I \textit{could} have tidied up my office.

\newpage
\section{Emphasis}
\subsection{Giving emphasis with ``it is ... that''}
\figbox{\textit{It is / was} + {\color{red}{emphasis}} + (that) + message}

i.g.:

Rob ate my biscuits.

\hspace*{2cm}$\downarrow$

\textit{It was} {\color{red}{Rob}} that (or who) ate my biscuits.\\
\textit{It was} {\color{red}{my biscuits}} that Rob ate.\\
\textit{It was} {\color{red}{yesterday}} that Rob ate my biscuits.

\subsubsection{Present sentences}
\textit{It is} {\color{red}{me}} that does all the work.

\subsubsection{Questions}
\textit{Was it} {\color{red}{you}} that told him?

\subsubsection{Negative sentences}
\textit{It wasn't} {\color{red}{me}} that told him.

\subsubsection{Formal sentences}
For formal sentences, we can use ``\textbf{I}'' insted ``me'':

\textit{It wasn't} {\color{red}{I}} who told him.

\subsection{Giving emphasis with ``what''}
\subsubsection{Emphasise noun}
\figbox{\textit{What / All} + understood info + \textit{is / was} + {\color{red}{emphasis}}}

i.g.:

I hated most insects everywhere.

\hspace*{3cm}$\downarrow$

\textit{What} I hated most \textit{was} {\color{red}{the insects everywhere}}.

and more:

\textit{What} I need now \textit{is} {\color{red}{a holiday}}.\\
\textit{All} I want for Christmas \textit{is} {\color{red}{you!}}.\\

We can use the inversion of subject and object complement:

{\color{red}{Dollar}} \textit{is what} I need.

\newpage
\subsubsection{Emphasise verb}
\figbox{\textit{What / All} + subject + do/does/did + \textit{is / was} {\color{red}{verb}}}

i.g.:

I only touched the shower.

\hspace*{2cm}$\downarrow$

\textit{What} I did \textit{was} {\color{red}{touch}} the shower.

\subsection{Emphasis the whole sentence}
\figbox{\textit{What happens / happened} + \textit{is / was} + \color{red}{clause}}

i.g.:

We got the hotel and realised that our room had been double booked.

\hspace*{5.5cm}$\downarrow$

\textit{What happened was} {\color{red}{we got the hotel and realised that our room had been double booked.}}

\newpage
\section{Future tenses}
\subsection{Summary}
\subsubsection{Formal use}
We use formal use to talk about events in the future. This is often
used by journalists:
\figbox{be + fully infinitive}

i.g.:
\begin{enumerate}
    \item The prime minister \red{is to open} a new factory;
    \item The motorway \red{is to shut} for maintenance;
    \item The actor \red{is to be} awarded for his services to theater.
\end{enumerate}

\subsubsection{Scheduled events}
\figbox{be due + full infinitive}

i.g.:
\begin{enumerate}
    \item Ling's train {\color{red}{is due to arrive}} at 9:37;
    \item Jay's parents \red{are due to leave} this evening;
    \item Ivana's exam\red{'s due to finish} at noon.
\end{enumerate}

\subsubsection{Certainty about the future}
\figbox{be + sure / bound + full infinitve}

i.g.:
\begin{enumerate}
    \item James\red{'s sure to be} late;
    \item Lenu \red{was bound to} win;
    \item It'\red{'s bound to} rain tomorrow.
\end{enumerate}

\subsubsection{Imminent events}
\figbox{sentence + be + on the verge / brink of + verb}

i.g.:
\begin{enumerate}
    \item The volcano is \red{on the verge of} erupting;
    \item The minister is \red{on the brink of} resigning;
    \item The countries are \red{on the verge of} war. 
\end{enumerate}

\subsubsection{Future meaning with present tenses}
We often use a present tense with a future meaning after verbs such as:
\textbf{hope, plan, aim, intend, want} and \textbf{propose}.
The verb that follows is in the infinitive:
\figbox{Subject + \red{present tense verb (hope, plan, intend, ...)} + to + base verb (infinitive) + ...}

\begin{enumerate}
    \item Elif \red{hopes to finish} her studies and find a job next year;
    \item Caterina \red{intends to buy} a house after saving for a few years;
    \item Jorge \red{plans to live} abroad.
\end{enumerate}

\newpage
\section{Relative clauses}
\subsection{Defining relative clauses}
Defining relative clauses add \textbf{essential} information about the
subject of the sentence. They define the \textbf{person, time} or \textbf{thing}
that we are talking about. If we remove the clause, the sentence does not make sense.

\figbox{Noun + \textbf{relative pronoun} + rest of the clause}

e.g.:
\begin{enumerate}
    \item The woman \textbf{who found my wallet} handed it in to reception;
    \item The student \textbf{whose dog has run away} has gone to look for it;
    \item I remember the day \textbf{when we first met}.
\end{enumerate}

\subsection{Non-defining relative clauses}
Non-defining relative clauses add \textbf{extra information} which are
not essential. If we remove the clause, the sentence still makes sense.
This type of clause is more common in written English.

\figbox{Noun + \textbf{, non-relative pronoun,} + rest of the clause}

e.g.:
\begin{enumerate}
    \item My friend's birthday, \textbf{which was last weekend}, was great fun;
    \item My current girlfriend, \textbf{who I love very much}, calls me every night.
\end{enumerate}

\subsection{Notes}

\subsubsection{Replacing the relative noun}
In informal communication, relative pronouns, such as \textbf{who} and
\textbf{when}, are commonly replaced with \textbf{that} in defining
relative clauses.

e.g.:
\begin{enumerate}
    \item The woman \textbf{that} called last night was very polite;
    \item Do you remember the time \textbf{that} you first met?
\end{enumerate}

\subsubsection{Leaving out the relative pronoun}
When using defining relative clauses in informal speech and writing,
the relative pronoun can be left out completely if it refers to the
object of the relative clause.

e.g.:
\begin{enumerate}
    \item This is the shirt \st{that} I bought;
    \item The girl \st{who} I lie isn't here yet.
\end{enumerate}

\subsubsection{Spoken English}
The relative pronoun ``who'' is used when referring to people. However,
in formal written and spoken English, if the pronoun refers to the
object of the clause, we use \textbf{whome} instead.

e.g.:
\begin{enumerate}
    \item My German teacher, \textbf{whom} I really admired, retired last year;
    \item The person \textbf{whom} I called this morning was my secretary.
\end{enumerate}









\end{document}
