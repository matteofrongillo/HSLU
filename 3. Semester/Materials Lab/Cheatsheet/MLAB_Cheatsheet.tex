\documentclass[9pt, landscape, a4paper]{article}

\usepackage{fontspec}
\renewcommand{\normalsize}{\fontsize{9}{11}\selectfont}
\usepackage[english]{babel}
\babelfont{rm}{Noto Sans}
\babelfont{sf}{Noto Sans}

\usepackage{amsmath}
\usepackage{amssymb}
\usepackage{hyperref}
\usepackage{url}
\usepackage{graphicx}
\usepackage{geometry}
\usepackage{enumitem}
\usepackage{parskip}
\usepackage{chemfig}
\usepackage{pdfpages}
\usepackage{xcolor}
\usepackage{tikz}
\usepackage{fancybox}
\usepackage{makecell}
\usepackage{pgfplots}
\usepackage{ulem}
\usepackage{wrapfig}
\usepackage{subcaption}
\usepackage{esvect}
\usepackage{siunitx}
\usepackage{mathtools}
\usepackage{varwidth}
\usepackage{multicol}
\usepackage{titlesec}
\usepackage{bm}
\usepackage{forest}

\usetikzlibrary{arrows}
\usetikzlibrary{arrows.meta}
\usetikzlibrary{decorations.pathreplacing}
\pgfplotsset{compat=1.18}
\definecolor{darkgreen}{rgb}{0.0, 0.6, 0.0}

% ============= GEOMETRY =============
\geometry{
    a4paper,
    landscape,
    left=0.5cm,
    right=0.5cm,
    top=1.5cm, 
    bottom=0.5cm,
    footskip=0cm
}

\setlist{nosep, leftmargin=*}

\AtBeginDocument{
  \setlength{\abovedisplayskip}{3pt}
  \setlength{\belowdisplayskip}{3pt}
  \setlength{\abovedisplayshortskip}{0pt}
  \setlength{\belowdisplayshortskip}{0pt}
}

% ============= HEADER & TITLES =============
\usepackage{fancyhdr}
\pagestyle{fancy}
\fancyhf{}
\rhead{\small \vspace*{.1cm}Materials Lab Cheatsheet}
\lhead{\small \vspace*{.1cm}Matteo Frongillo}
\cfoot{\small \vspace*{.4cm}\thepage}
\renewcommand{\headrulewidth}{0pt}
\setlength{\headsep}{4pt}

\titlespacing*{\part}{0pt}{2pt}{1pt}
\titlespacing*{\section}{0pt}{2pt}{1pt}
\titlespacing*{\subsection}{0pt}{2pt}{1pt}
\titlespacing*{\subsubsection}{0pt}{1pt}{1pt}

\titleformat{\part}{\large\bfseries\color{black}}{\thepart.}{0.5em}{}
\titleformat{\section}{\bfseries\color{red!80}}{\thesection.}{0.5em}{}
\titleformat{\subsection}{\small\bfseries\color[HTML]{007AFF}}{\thesubsection}{0.5em}{}
\titleformat{\subsubsection}{\small\bfseries\color[HTML]{007355}}{\thesubsubsection}{0.5em}{}

% ============= META DATA =============
\newcommand{\pdftitle}[1]{\hypersetup{
  colorlinks=true,
  linkcolor=black,
  urlcolor=blue,
  pdftitle={#1},
  pdfauthor={Matteo Frongillo}
}}

% ============= RENEW COMMANDS =============

\makeatletter
\renewcommand{\author}[1]{%
  \def\@author{%
    #1\\
    \parbox{\textwidth}{%
      \centering
      \small Last update: \today
    }
  }%
}
\makeatother

\let\emptyset\varnothing

\makeatletter
\pgfmathdeclarefunction{arctan}{1}{%
    \begingroup
    \pgfmathparse{rad(atan(#1))}%
    \endgroup
}
\makeatother

% ============= NEW COMMANDS =============

\newcommand{\figbox}[1]{ 
    \fbox{\begin{varwidth}{\textwidth} #1 \end{varwidth}}        
}

\newcommand{\wrapfill}{
    \par
    \ifnum \value{WF@wrappedlines} > 0
        \addtocounter{WF@wrappedlines}{-1}%
        \null\vspace{
            \arabic{WF@wrappedlines}
            \baselineskip
        }
        \WFclear
    \fi
    \phantom{}
}

\ExplSyntaxOn
\DeclareDocumentCommand{\integral}{d[] d[] d[] d[]}
{
    \IfNoValueTF{#3}{
        \IfNoValueTF{#4}{
            \displaystyle \int #1\,d#2
        }{
            \text{Error: either 2 or 4 arguments must be provided}
        }
    }{
        \IfNoValueTF{#4}{
            \text{Error: either 2 or 4 arguments must be provided}
        }{
            \displaystyle
            \int\limits
                \IfNoValueF{#1}{\c_math_subscript_token{#1}}
                ^{\IfNoValueTF{#2}{}{#2}}
                #3\, d#4
        }
    }
}
\ExplSyntaxOff

\newcommand*\circled[1]{\tikz[baseline=(char.base)]{
            \node[shape=circle,draw,inner sep=1pt] (char) {\small #1};}}

\makeatletter
\newcommand*{\rom}[1]{\expandafter\@slowromancap\romannumeral #1@}
\makeatother

\newcommand{\difference}{\,\backslash\,}
\newcommand{\rem}{\underline{Remark}: }
\newcommand{\nots}{\underline{Notation}: }
\newcommand{\note}{\underline{Note}: }
\newcommand{\prf}{\underline{Proof}: }
\newcommand{\exs}{\underline{Example}: }
\newcommand{\defs}{\underline{Definition}: }
\newcommand{\wrn}{\underline{Warning}: }
\newcommand{\sht}{\ |\ }
\newcommand{\pph}[1]{\paragraph{#1} \phantom{}\\}
\newcommand{\dm}{\displaystyle}
\newcommand{\rad}{^{\mathrm{c}}}
\newcommand{\dsum}[2]{\displaystyle \sum_{#1}^{#2}}
\newcommand{\dx}[2]{\dfrac{{\mathrm{d}}^{#1}{#2}}{\mathrm{d}x^{#1}}}
\newcommand{\cel}{$^\circ$C}

% ============= DOCUMENT CONTENT =============

\begin{document}
\pdftitle{MLAB-Cheatsheet}
\vspace*{-.5cm}
\setlength{\columnsep}{0.5cm}

\begin{multicols*}{3}
\vspace*{-.5cm}

\part{Physical metallurgy}
\section{Classes and properties}
\begin{center}
    \begin{tabular}{|c|l|}
    \hline Class & Typical properties\\
    \hline \makecell{Metal\\Alloys} & \makecell[l]{1) Thermal + electric conductivity\\2) Ductility\\3) Castable\\4) Reflective}\\
    \hline Ceramics & \makecell[l]{1) High T resistance (High E, low $\alpha$)\\2) High compression strength\\3) Thermal + electric insulator\\4) Wear resistance}\\
    \hline Polymers & \makecell[l]{1) Cheap\\2) Thermal + electric insulator\\3) Corrosion resistance\\4) Moldable}\\
    \hline
    \end{tabular}
\end{center}

\subsection{Structural model of metals}
\[\phi = \dfrac{\text{Volume occupied by atoms in unit cell}}{\text{Total volume of unit cell}}\]
\vspace*{0cm}

\subsubsection{FCC (Face-centered cubic)}
\begin{minipage}{0.3\linewidth}
    \centering
    \includegraphics[width=.65\linewidth]{media/fcc.png}
\end{minipage}%
\begin{minipage}{0.7\linewidth}
    \begin{itemize}
        \item Packing efficiency: $\phi \approx 74\%$
        \item Many slip systems (12)
        \item Closest pack direction
    \end{itemize}
\end{minipage}

\subsubsection{BCC (Body-centered cubic)}
\begin{minipage}{0.3\linewidth}
    \centering
    \includegraphics[width=.65\linewidth]{media/bcc.png}
\end{minipage}%
\begin{minipage}{0.7\linewidth}
    \begin{itemize}
        \item Packing efficiency: $\phi \approx 68\%$
        \item Many slip systems (6)
        \item Not closest pack direction
        \item Cottrell atmosphere
    \end{itemize}
\end{minipage}

\subsubsection{HCP (Hexagonal close-packed)}
\begin{minipage}{0.3\linewidth}
    \centering
    \includegraphics[width=.65\linewidth]{media/hcp.png}
\end{minipage}%
\begin{minipage}{0.7\linewidth}
    \begin{itemize}
        \item Packing efficiency: $\phi \approx 74\%$
        \item Very few slip systems (3)
        \item Closest pack direction
    \end{itemize}
\end{minipage}

\subsection{Structural model of ceramics}
\begin{itemize}
    \item Ionic bonding, complex crystal structures (ceramics), amorphous (glasses)
    \item Brittle, but high chemical and thermal resistance
    \item Insulators
    \item Wear-resistant (e.g. ferro-/piezoelectricity)
\end{itemize}

\subsection{Structural model of polymers}
\begin{itemize}
    \item Macromolecules (10$^3$ to 10$^5$ C atoms)
    \item Weaker intermolecular bonds but strong atomic bond in molecular chain
    \item Electrically and thermally insulating
    \item Cheap, moldable, massive waste problem, pollutant
    \item Matrix for many composite materials (no recycling)
\end{itemize}

\subsection{Amorphous and crystalline materials}
\begin{center}
    \includegraphics[width=.9\linewidth]{media/mono-poly-amorph.png}
\end{center}

\subsubsection{Amorphous materials}
\begin{itemize}
    \item No crystal lattice (e.g. quartz glass, polymers)
    \item Atomic distances defined by chemical bonds
    \item Bond angles are variable
\end{itemize}
\vspace*{-.2cm}
\textbf{Field}: inorganic classes (also Gorilla glass), metallic glasses (ferrous
transformer sheet metal), amorphous plastics (PMMA - plexiglass, COC, ...)

\subsubsection{Crystalline materials}
\begin{itemize}
    \item Crystal lattice (e.g. metals, ceramics, quartz)
    \item Atomic distances and bonding angles are defined
\end{itemize}

\subsubsection{Monocrystalline materials}
Only for special applications, expensive.\\
\textbf{Field}: single-crystal turbine blades (T $>$ 1000$^\circ$C, creep-resistant),
semiconductors, MEMS components made of silicon (gyroscopes in smartphones, accelerometers),
optical elements (laser crystals, $\lambda$/4 plates, crystals for frequency doubling of lasers).

\subsubsection{Polycrystalline materials}
Most metal components are polycr. (made of many grains/crystals).

\subsection{Directional dependence}
\subsubsection{Anisotropy and Isotropy}
\begin{itemize}
    \item \textbf{Anisotropic}: properties depend on the direction\\(eg. single crystals, wood, composites)
    \item \textbf{Isotropic}: properties do not depend on direction\\(eg. polycr. metals, amorphous materials)
    \item \textbf{Quasi-isotropic}: properties microscopically depend on the direction but not macroscopically
\end{itemize}

\subsection{Polymorphism (Allotropy)}
\subsubsection{Iron}
\begin{itemize}
    \item $\alpha$-Fe (ferrite, BCC) $\rightarrow$\ below 911\cel
    \item $\gamma$-Fe (austenite, FCC) $\rightarrow$\ 911\cel to 1392\cel
    \item $\delta$-Fe (ferrite, BCC) $\rightarrow$\ 1932\cel to 1536\cel
\end{itemize}

\subsubsection{Carbon}
\begin{center}
    \includegraphics[width=\linewidth]{media/carbon_poly.png}
\end{center}

\subsubsection{Other materials}
\begin{itemize}
    \item Titanium: HCP $<$ 880\cel, BCC $>$ 880\cel
    \item Shape memory allory (eg. NiTi)
    \item Zirconia (high crack resistance due to phase transformation toughening)
    \item Ferro- and piezoelectric materials (eg. PZT, quartz, ...)
\end{itemize}

\subsubsection{Shape memory alloy (SMA)}
\begin{minipage}{0.5\linewidth}
    \begin{center}
        \includegraphics[width=.9\linewidth]{media/Nitinol.png}
    \end{center}
\end{minipage}
\begin{minipage}{0.5\linewidth}
    NiTi is a SMA used for screen lock of tablet notebooks, medtech and
    spectacle frames.
\end{minipage}%

\subsection{Microstructure and phases}
\subsubsection{Homogeneous microstructure}
They have only one phase and crystal structure. A phase can be
either crystalline or amorphous (eg. only iron crystals).

\subsubsection{Heterogeneous microstructure}
They have multiple phases and many types of crystal structures
(eg. graphite and iron).

\subsection{Alloys}
An alloy is a metallic material of at least 2 types of atoms:
\vspace*{-.2cm}
\begin{itemize}
    \item Metal + metal (iron-nickel, gold-silver, tin-lead, aluminum-copper)
    \item Metal + non-metal (iron-carbon, nickel-phosphorus)
\end{itemize}

\subsubsection{Microstructure of alloys}
\begin{itemize}
    \item \textbf{Solid solution crystal}: homogeneous, single-phase, only one type of crystal
    \item \textbf{Mix of different crystal types}: heterogeneous, multi-phase
\end{itemize}

\section{Metal structures and crystal lattice defects}
\subsection{Lattice defects}
\subsubsection{0-dimensional defects}
Point defects due to vacancies and impurity atoms.

\subsubsection{1-dimensional defects}
Line defects due to dislocations. Edge dislocations insert an
extra half-plane of atoms in the crystal, distorting the nearby
planes of atoms.

\subsubsection{2-dimensional defects}
Surface defects due to grain boundaries. Crystal growth starts at
multiple locations within the molten metal\\
\textbf{Crystallization from a melt}: (1) homogeneous melt, (2)
nucleation of crystals, (3) crystal growth surrounded by residual melt,
(4) fully solidified polycrystalline structure with grain boundaries.

\subsubsection{3-dimensional defects}
Volume defects due to precipitations, inclusions, voids, cracks in
the crystal structure.

\section{Elastic and plastic deformation}
\subsection{Elastic deformation}
The coefficient of thermal expansion $\alpha$ is inversely proportional
to: Young's Modulus $E$, Bonding energy, Melting temp.

\subsection{Elastic constants of isotropic materials}
\subsubsection{Elastic stress, strain, and Young's modulus}
\[\varepsilon_x = \frac{\sigma_x}{E} \Longleftrightarrow \sigma_x = \frac{E}{\varepsilon_x}\]

\subsubsection{Poisson's ratio $\nu$}
When a material is stretched in x, it tends to contract in y, z:
\[\nu = -\frac{\varepsilon_{y,z}}{\varepsilon_x}\]

\subsubsection{Relation between 3 isotropic elastic constants $G$}
\[G = \frac{E}{2(1+\nu)} = \frac{\sigma}{2\varepsilon_x(1+\nu)}\]

\subsection{Plastic deformation in metals}
The plastic deformation is permanent and non-reversible

\subsubsection{At room temp T$_\text{R}$}
\begin{itemize}
    \item Dislocations move on densely packed slip planes in densely packed directions
    \item Smaller slip distances require less external force or energy
\end{itemize}

\subsubsection{At high temps}
The metal creeps, leading to diffusion of atoms, especially at grain boundaries.

\subsection{Simplified dislocation slip model}
\begin{center}
    \includegraphics[width=.84\linewidth]{media/simplified_dislocation_slip_model.png}
\end{center}

\subsection{Slip systems}
\subsubsection{Slip systems in FCC metals}
FCC metals have 12 close packed systems, making them \textbf{soft and highly ductile}
(eg. Au, Ag, Cu, Al, $\alpha$-Fe).

\subsubsection{Slip systems in HCP metals}
HCP metals are closely packed but deform on only one slip plane with 3 slip systems,
resulting in \textbf{limited ductility} (eg. Ti, Zn, Mg).

\subsubsection{Slip systems in BCC metals}
BCC metals have 48 slip systems but are less closely packed, leading
to \textbf{higher strength and lower ductility} (eg. $\alpha$-Fe, Cr,
W, Mo, Ta, Nb).

\subsection{Metals crystal structure vs Ductility}
\begin{center}
    \includegraphics[width=\linewidth]{media/structure_vs_ductility.png}
\end{center}

\subsection{Cottrell atmospheres and Dislocation pinning}
\begin{itemize}
    \item In $\alpha$-Fe with a BCC structure (ferrite), the sites for interstitial atoms are much smaller than $\gamma$-Fe (austenite)
    \item Carbon atmos in ferrite diffuse into the distortion fields near dislocation lines, forming Cottrell atmospheres
    \item Causes upper yield point (R$_\text{eH}$) in tensile tests and brittle fracture at low temps
    \item During plastic deformation, dislocations must first break freefrom the Cottrell atmosphere. 
\end{itemize}

\section{Strenghtening mechanisms}
\begin{center}
    \includegraphics[width=\linewidth]{media/strengthening.png}
\end{center}

\subsection{Solid-solution hardening}
\begin{itemize}
    \item Impurity atoms in a solid solution create lattice distortion fields that impede dislocation motion
    \item Interstitial atoms cause stronger lattice distortions than substitutionals atoms, leading to a greared strengthening effect
    \item A larger atomic radius mismatch and higher impurity concentration both increase the stremgthening effect
    \item \textbf{Result}: increased strength but reduced ductility
\end{itemize}

\subsubsection{0-d: SSH application field}
Al-Mb and Al-Mg alloys (5000 and 3000) for automotive sheet metal,
airplane outer skin, beverage cans;
Structural and stainless steels; Gold jewerly (Au with Ag, Cu, Ni,
Pt, Pd, ...)

\subsection{1-d: Strain hardening (Work hardening)}
\begin{itemize}
    \item Cold work increases dislocation density $\rightarrow$\ dislocation entanglement
    \item Cold working / cold forming: below the recrystallization temp
    \item Hot forming: above the recrystallization temp
    \item Recrystallization temp: $(T_{\mathrm{recr}} \approx 0.4,T_m\ [\text{K}])$
    \item \textbf{Result}: strength increases but ductility decreases
\end{itemize}

\subsubsection{SH application field}
Any cold-formed parts: cold rolled sheet metal, neckline holes in
sheet metal of a washing machine drum, cold pressed steels.

\subsection{2-d: Grain boundary hardening}
\begin{itemize}
    \item Smaller grains increase strength: grain boundaries hinder dislocation slip
    \item Not suitable at high temp: grain growth reduces strength and ductility
    \item \textbf{Result}: strength and ductility both increased (statistical grain orientation, Schmid's law)
\end{itemize}

\subsubsection{GBH application field}
High-strength fine-grain or even Q+T structural steels:
cranes, oil platforms, bridges, shipbuilding.

\subsection{3-d: Precipitation hardening}
\begin{itemize}
    \item Increase in strength by coherence of the precipitates and size and number of the precipitates
    \item \textbf{Result}: increase in strength but decrease in ductility
\end{itemize}

\subsubsection{Effect of precipitate coherence on strength}
\textbf{Coherent}: lattice planes match, strong strain fields in the matrix $\rightarrow$\ large strength increase.\\
\textbf{Semi-coherent}: misfit dislocations at the interface reduce lattice strain $\rightarrow$\ strength decreases.\\
\textbf{Incoherent}: lattice planes mismatch; little matrix distortion\\$\rightarrow$\ small strength increase.

\subsubsection{PH application field}
High-strength aluminum alloys for aerospace, maraging steels, stainless PH steels,
highly temp-resistant Ni base superalloys for turbine blades.
\vfill\phantom{}

\newcolumn
\part{Strength and Ductility}
\section{Properties of material}
\begin{center}
    \includegraphics[width=\linewidth]{media/propertiesofmaterial.png}
\end{center}

\subsection{Failure hypotesis and Material testing methods}
\begin{center}
    \includegraphics[width=\linewidth]{media/failures.png}
\end{center}

\section{Tensile test}
\subsection{Engineering stress and stress conditions}
\subsubsection{Normal vs Engineering vs Shear stress}
Engineering stress is the force $F$ acting on the original
cross-sectional area $S_0$.
Normal stress is the normal force $F_N$ that acts
perpendicularly to $S_0$.
The shear stress $\tau$ is the force $F_Q$ parallel to $S_0$.
\[\sigma=\frac{F}{S_0}\quad;\quad \sigma_N = \frac{F_N}{S_0}\quad;\quad \tau = \frac{F_Q}{S_0}\quad;\quad \sigma = E\cdot \varepsilon\]

\subsubsection{Engineering strain $\varepsilon$}
Engineering strain is the ratio of the change in length to the
original length of the material under load:
\[\varepsilon = \frac{\Delta L}{L_0}\]

\subsection{Stress-strain behavior of metals}
\begin{center}
    \includegraphics[width=\linewidth]{media/stressstrain.png}
\end{center}

\newcolumn
\subsubsection{Yield stress R}
\begin{itemize}
    \item \textbf{R$_\text{eH}$}: Upper Yield point
    \item \textbf{R$_\text{p0.2}$}: 0.2\% Yield Stress
    \item Without max yield point: $\sigma_\text{max}$ = R$_\text{p0.2}$
\end{itemize}

\subsubsection{Yound's modulus}
Slope of the linear-elastic region: \[E=\frac{\Delta\sigma}{\Delta\varepsilon}\]

\subsubsection{Tensile strength R$_\text{m}$}
It correspondes to the stress at the max of the stress-strain curve.
With max yield point: $\sigma_\text{max}$ = R$_\text{m}$; upper yield point $R_\text{eH}$.

\subsection{Fracture strain A}
Plastic strain at fracture is defined with respect to the initial
specimen length $L_0$. (eg. $L_0 = 50$mm $\rightarrow A_{50\text{mm}}$)

\subsubsection{Uniform strain A$_\text{g}$}
It corresponds to the plastic strain at maximum load before necking
begins.

\subsubsection{Contraction at fracture Z}
\[Z = \frac{\Delta S}{S_0}\]

\section{Other quasi-static mechanical tests}
\subsection{Bending test}
A specimen is placed on two supports, loaded in bending (typically at midspan) until failure, and the maximum bending stress is calculated.

\subsection{Torsion test}
A cylindrical specimen is clamped and twisted by applying torque until yielding or fracture, and the torsion strength is taken from the peripheral shear stress.

\subsection{Creep and relaxation tests (High temps)}
\begin{itemize}
    \item Creep: time-dependent deformation under constant stress
    \item Relaxation: decreasing stress under constant strain
    \item At high T, strength becomes time- and strain-rate-dependent
    \item Strain changes with time under constant load
    \item Creep/relaxation tests assess heat-resistant materials for high-T applications (eg. steels and Ni alloys above 400\cel)
\end{itemize}

\part{Steel - technology and applications}
Steel is a Carbon-iron alloy. Has very good properties: strength,
ductility, toughness, formability, machining, weldability. Can be
heat-treated due its polymorphism.

\section{Steel Technology}
\subsection{Blast furnace and Pig iron}
\begin{itemize}
    \item \textbf{Process}: reduction of iron oxide with coke\\
        Fe$_2$O$_3$ + 3C $\rightarrow$\ 2Fe + 3CO
    \item \textbf{Product}: pig iron (3-5\% C, also Mn, Si, S, P)
\end{itemize}

\subsection{Crude steel production}
\begin{itemize}
    \item Pig iron refined to crude steel by reducing carbon content
    \item Oxygen-Blown Converter (OBC): refining with oxygen
    \item Electric Furnace (EF): melts scrap steel or direct-reduced iron (sponge iron) to crude steel
\end{itemize}

\subsection{Secondary metallurgy}
\begin{itemize}
    \item Crude steel further purified and adjusted to final composition in a ladle furnace
    \item Process:
    \begin{itemize}
        \item Deoxidation (chemical or vacuum)
        \item Advanced purification (electroslag remelting (ESR), vacuum arc remelting (VAR), powder metallurgy (PM))
        \item Alloying additions
    \end{itemize}
\end{itemize}

\subsection{Semi-finished products}
\begin{itemize}
    \item Produced mainly by continuous casting (slabs, billets)
    \item Formed into sheets, plates, wires, rods, pipes, and profiles by hot or cold rolling/drawing
\end{itemize}

\subsection{Important alloying elements}
\begin{center}
    \begin{tabular}{|l|l|}
        \hline \textbf{Function} & \textbf{Alloying elements}\\
        \hline Hardenability & Mn, Cr, Ni, Mo, V, Si\\
        \hline Grain refinement & Al, V, Ti, Nb\\
        \hline Corrosion resistance & Cr (>12\%), Cu, Ni, Si, Mn, Mo, N\\
        \hline Wear/heat resistance & Cr (>1\%), Mo, V, Al, Ti, Nb, Mo\\
        \hline Scale resistance & Cr, Al, Si\\
        \hline 
    \end{tabular}
\end{center}

\section{Microstructure formation}
\subsection{Polymorphism of iron}
\begin{center}
    \includegraphics[width=\linewidth]{media/iron_polymorphism.png}
\end{center}

\subsection{Metastable iron-iron carbide phase diagram}
\subsubsection{Phase diagram}
\begin{center}
    \includegraphics[width=\linewidth]{media/phasediagram_explaination.png}
\end{center}

\newcolumn
\begin{center}
    \includegraphics[width=\linewidth]{media/iron-castiron_diagram.png}
\end{center}

\subsection{Microstructure formation of Steel (Slow Cooling)}
\begin{center}
    \scriptsize
    \begin{tabular}{|c|c|l|l|}
        \hline Hypoeutectoid & < 0.8 & \makecell[l]{Primary ferrite\\+pearlite} &
        \makecell[l]{ductility $\uparrow\uparrow$\\weldability $\uparrow\uparrow$\\formability $\uparrow\uparrow$\\hardness $\downarrow$}\\
        \hline Eutectoid & 0.8 & \makecell[l]{Pearlite (ferrite\\+cementite)} &
        \makecell[l]{hardness $\uparrow$\\toughness $\uparrow$\\ductility $\downarrow$}\\
        \hline Hypereutectoid & > 0.8 & \makecell[l]{Primary cementite\\+pearlite} &
        \makecell[l]{hardness $\uparrow\uparrow$\\wear $\uparrow\uparrow$\\toughness $\downarrow$}\\
        \hline
    \end{tabular}
\end{center}

\subsubsection{Eutectoid steel final microstructure at T$_\text{R}$}
The microstructure is 100\% pearlite, consisting of ferrite lamellae
(soft, ductile phase forming the light layers), and cementite lamellae
(hard, brittle phase forming the dark layers).

\subsubsection{Hypereutectoid steel final microstructure at T$_\text{R}$}
Final components are grain boundary cementite (hard and brittle,
formed before eutectoid reaction) and pearlite (lamellar mixture of
ferrite and cementite).

\subsection{Faster cooling / Quenching}
Fast cooling of steels changes the transformation behavior compared
to equilibrium cooling.

\subsubsection{Fast cooling}
\begin{itemize}
    \item Transformation temps shift to lower values
    \item Pearlite become finer and forms over a wider Temp range
    \item Martensite and bainite can form
\end{itemize}

\subsection{Ferrite}
\begin{itemize}
    \item $\alpha$-iron (BCC), stable at low temps
    \item Very low carbon solubility $\rightarrow$ carbon tends to form cementite
    \item Soft and ductile $\rightarrow$ good formability and toughness
\end{itemize}

\subsection{Austenite}
\begin{itemize}
    \item $\gamma$-iron (FCC), stable at higher temps
    \item High carbon solubility $\rightarrow$ carbon dissolves readily
    \item Parent phase for transf. to pearlite, bainite, and martensite
\end{itemize}

\subsection{Pearlite}
\begin{itemize}
    \item Forms from austenite during slow cooling (eutectoid trans.)
    \item Lamellar mixture of ferrite ($\alpha$) and cementite ($\mathrm{Fe_3C}$)
    \item Higher strength and hardness than ferrite, but lower ductility; finer lamellae give higher strength
\end{itemize}

\subsection{Martensite}
\begin{itemize}
    \item Very fast cooling of austenite produces martensite
    \item Diffusion does not occur, so no ferrite or pearlite
    \item Lattice shear of FCC austenite produces BCT
\end{itemize}

\subsection{Bainite}
\begin{itemize}
    \item Produced by a rapid cooling (250-500\cel)
    \item Combined shear and limited diffusion; cementite precipitate
    \item Higher strength than pearlite, more ductile than martensite
    \item Good for high-performance components and tools
\end{itemize}

\subsubsection{Upper bainite}
\begin{itemize}
    \item Forms at higher bainitic T (350-500\cel)
    \item Ferrite needles with cementite lamellae at ferrite boundaries
\end{itemize}

\subsubsection{Lower bainite}
\begin{itemize}
    \item Forms at lower bainitic T (250-350\cel)
    \item Ferrite needles with fine cementite particles inside the ferrite
\end{itemize}

\section{Heat treatments}
\begin{center}
    \includegraphics[width=\linewidth]{media/heat_treatments.png}
\end{center}

\subsection{Process steps}
\vspace*{-.2cm}
\begin{center}
    \footnotesize
    \begin{tabular}{|c|l|l|}
        \hline \textbf{Process step} & \textbf{\makecell[l]{Microstructure after\\each process step}} & \textbf{\makecell[l]{Strength and toughness\\after each process step}}\\
        \hline Austenitizing & Austenite & \makecell[l]{Low strength (low R$_\text{p0.2}$),\\high toughness,\\high energy absorbed.}\\
        \hline Quenching & Martensite & \makecell[l]{Very high strength,\\brittle.}\\
        \hline \makecell{Temperting\\400-700\cel} & \makecell[l]{Very small ferrite and\\cementite crystals} & \makecell[l]{High strength,\\high toughness.}\\
        \hline 
    \end{tabular}
\end{center}

\subsection{Homogeneization annealing}
\begin{itemize}
    \item Elimination of segretations
    \item Often performed together with hot forming
\end{itemize}

\subsection{Normalizing}
\subsubsection{Austenization + slow cooling}
\begin{itemize}
    \item Produces a uniform, fine-grained microstructure
    \item Austenitize: transform ferrite to austenite
    \item Slow, controlled cooling
    \item \textbf{Result}: fine grains $\rightarrow$\ higher strength and toughness
\end{itemize}

\subsection{Martensitic hardening and Q+T}
\subsubsection{Martensitic hardening}
Austenitize $\rightarrow$\ quench $\rightarrow$\ low-temp temper (200\cel)

\subsubsection{Quench + temper}
Austenitize $\rightarrow$\ quench $\rightarrow$\ temper (350-700\cel) $\rightarrow$\ tempered martensite

\subsubsection{Bainitic transformation}
Continuous cooling of low-alloy steels to form bainite.

\subsubsection{Graphical representation}
\begin{center}
    \includegraphics[width=\linewidth]{media/martensitic_hardening.png}
\end{center}

\subsection{Case hardenable steels}
Typically contain less than 0.25\% carbon before carburizing.
After case hardening and quenching, martensite forms at the surface,
improving wear resistance and fatigue resistance.

\section{Hardening of the surface overview}
\subsection{Without thermochemical diffusion}
\subsubsection{Surface hardening}
For C > 0.3\%, quenching forms martensite at the surface.

\subsection{With thermochemical diffusion}
\subsubsection{Case hardening}
For C < 0.25\%, first diffuse C (carburizing) or C+N (carbonitriding),
then quench to form surface martensite.
\subsubsection{Nitriding}
Diffuses N to form a hard nitride layer (or carbonitrades in nitrocarburizing).
In works for any steel, highest hardness in nitriding steels with Al/Mo/C.

\subsection{Hardening Depth}
\begin{itemize}
    \item \textbf{SHD}: at 0.8 $\cdot$ HV$_\text{surface}$ (80\% of SH)
    \item \textbf{CHD}: at 550 HV1 (limit hardness line at 550 with HV1 load)
    \item \textbf{NHD}: at HV$_\text{lower line}$ + 50HV0.5 
\end{itemize}

\part{Hardness and Toughness}
\section{Hardness}
It measures resistance against localized plastic deformation.

\subsection{Hardness testing}
\begin{center}
    \footnotesize
    \begin{tabular}{|l|l|}
        \hline \textbf{Testing method} & \textbf{Application}\\
        \hline Vickers (HV) & Universal application\\
        \hline Rockwell (HCR) & For hard steels\\
        \hline Brinell (HB) & For soft steels and aluminum\\
        \hline Berkovich & For nanoindentation\\
        \hline Shore A, D & For rubber and plastic\\
        \hline 
    \end{tabular}
\end{center}

\subsection{Hardness Testing procedures}
\subsubsection{Indentation depth}
Hardness is based on the indenter penetration depth.

\subsubsection{Indentation area}
Hardness is based on the indentation surface area.

\section{Notch Impact Toughness}
\subsection{Impact Notch Toughness test (Charpy)}
\begin{itemize}
    \item Applied mainly to structural steel (BCC)
    \item Shows the transition temp from ductile to brittle fracture
    \item Determines the absorbed impact energy (``notch toughness'')
    \item Failure hypotesis: crack propagation under sudden loads
\end{itemize}

\subsection{Toughness}
\subsubsection{Material behavior}
\begin{itemize}
    \item \textbf{Tough material}: absorb high energy before fracture
    \item \textbf{Brittle material}: fracture with little energy absorbtion
\end{itemize}

\subsubsection{Energy criterion}
\begin{itemize}
    \item \textbf{Ductile fracture}: if absorbes > 27 J of the impact energy
    \item \textbf{Brittle fracture}: if absorbes < 27 J of the impact energy
\end{itemize}

\subsubsection{Stress State dependence}
\begin{itemize}
    \item Monoaxial (tensile test): material can yield in lateral ways
    \item Biaxial (pressure vessels): yield possible in one direction
    \item Triaxial (notches): no yielding possible, leads to brittle failure
\end{itemize}

\subsection{BCC brittle behavior at low temps}
\subsubsection{Cottrell atmosphere (BCC $\alpha$-Fe)}
Small interstitial sites, C/N diffuse to dislocation stress fields and cluster there,
forming atmospheres that lock dislocations.

\subsubsection{Dislocation pinning}
Cottrell locking produces an upper yield point R$_\text{eH}$, plastic flow starts
only after dislocations break away.

\subsubsection{Low temperature brittleness (BCC)}
At low temp or high strain rate, breakaway is difficult\\
$\rightarrow$\ stronger pinning, reduced ductility, and brittle behavior.

\subsubsection{Absorbed Notch Impact energy}
\begin{center}
    \includegraphics[width=\linewidth]{media/energyabsorbed.png}
\end{center}

\subsection{Absorbed energy - Temperature graph}
\begin{center}
    \includegraphics[width=.745\linewidth]{media/energy_temperature_impact_graph.png}
\end{center}

\part{Aluminum - Wrought and Cast alloys}
\section{Properties and application of aluminum alloys}
\begin{center}
    \footnotesize
    \begin{tabular}{|l|l|}
        \hline \textbf{Property} & \textbf{Application}\\
        \hline Heat conductivity & Heat exchangers\\
        \hline Electrical conductivity & High voltage lines\\
        \hline \makecell[l]{Corrosion resistance\\(< 10 pH only)} & \makecell[l]{Electronic appliace\\housing, architecture}\\
        \hline Non-magnetic & Electronic appliace housing\\
        \hline Light-weight & Aerospace, automotive industry\\
        \hline 
    \end{tabular}
\end{center}

\section{Designation of alloys and conditional designations}
\subsection{Numerical Designation System (DIN EN 573-1)}
\begin{center}
    \includegraphics[width=.95\linewidth]{media/aluminum_designation.png}
\end{center}

\subsection{Condition Designation (DIN EN 515)}
\begin{center}
    \includegraphics[width=.8\linewidth]{media/aluminum_condition.png}
\end{center}

\subsection{Cold-working H}
\begin{center}
    \includegraphics[width=.9\linewidth]{media/cold-working.png}
\end{center}

\subsection{Precipitation hardening (age hardening)}
\subsubsection{Age hardening steps}
\begin{enumerate}
    \item Solution annealing (W): dissolves existing precipitates into alloy
    \item Quenching: rapidly cools to form a supersaturated solid solution without precipitates
    \item Aging: small, coherent precipitates form, alloy strengthening
\end{enumerate}

\subsubsection{Aging types}
\begin{itemize}
    \item Natural aging: at room temps, moderate strength, lower hardness, higher ductility
    \item Artificial aging: elevated temps (120-200\cel), higher strength and hardness, lower ductility
\end{itemize}

\begin{center}
    \includegraphics[width=.88\linewidth]{media/age_hardening.png}
\end{center}

\subsubsection{Artificial aging}
\begin{center}
    \includegraphics[width=.8\linewidth]{media/artificial_aging.png}
\end{center}

\section{Aluminum Wrought Alloys}
\subsection{Pure aluminum}
\subsubsection{Properties}
\begin{itemize}
    \item Excellent electrical conductivity (best weight/cost balance)
    \item High thermal conductivity (good heat dissipation)
    \item Good corrosion resistance, formability, and weldability
    \item Low inherent strength; strengthened by cold working and grain refinement
    \item Very suitable for surface finishing/polishing/anodizing
\end{itemize}

\subsubsection{Applications}
\begin{itemize}
    \item Electrical/electronics (busbars, bonding wires, HV cables)
    \item Heat exchangers/heat sinks; avoid Al-Cu contact to limit galvanic corrosion
    \item Corrosion-resistant cladding on high-strength sheets for aircraft/automotive
    \item Food packaging (foils, dishes, coffee capsules)
\end{itemize}

\subsection{Type of wrought alloys system}
\subsubsection{EN AW-3XXX: (Al-Si)}
\begin{itemize}
    \item Stronger than pure Al, easy to form/weld
    \item Good corrosion resistance
    \item Tanks/chemical equipment, cookware, gas pipes
\end{itemize}

\subsubsection{EN AW-4XXX: Al-Si}
\begin{itemize}
    \item Mainly used as cast alloys/coating
    \item Low eutectic melting point
    \item Cast parts, corrosion-resistant coatings, welding wires
\end{itemize}

\subsubsection{EN AW-5XXX: Al-Mg}
\begin{itemize}
    \item Excellent weldability and corrosion resistance
    \item Ship hulls, tanks, cookware, beverage can lids
\end{itemize}

\subsubsection{Over-saturated 5XXX (>3\% Mg, quenched)}
\begin{itemize}
    \item High strength
    \item Risk of intergranular corrosion and stress-corrosion cracking
    \item High-strength rods/profiles, avoid hot/chloride exposure
\end{itemize}

\subsubsection{EN AW-6XXX: Al-Mg-Si}
\begin{itemize}
    \item Good strength, good processing properties
    \item Formability + weldability + corrosion resistance + anodizing
    \item Extruded profiles, doors/windows, bicycle frames
\end{itemize}

\subsubsection{EN AW-2XXX: Al-Cu}
\begin{itemize}
    \item High strength and fatigue performance (T3/T4/T6)
    \item Poor fusion weldability and corrosion resistance
    \item Aircraft, high-strength structural parts
\end{itemize}

\subsubsection{EN AW-7XXX: Al-Zn-Mg (-Cu)}
\begin{itemize}
    \item Very high strength; poor corrosion resistance and weldability
    \item SCC/exfolian risk: often use over-aged tempers
    \item Aircraft parts, climbing hardware, lightweight housing
\end{itemize}

\section{Aluminum cast alloys}
\begin{center}
    \includegraphics[width=.5\linewidth]{media/aluminum_cast_alloys.png}
    \includegraphics[width=\linewidth]{media/tectics.png}
\end{center}

\subsection{Hypoeutectic cast alloys}
\begin{itemize}
    \item Solidify as primary $\alpha$-Al + remaining eutectic
    \item At T$_\text{R}$: primary $\alpha$-Al crystal + residual eutectic
    \item Mg/Cu/Fe additions allow precipitation hardening of $\alpha$
    \item Typical alloys: EC AC-465000, EC AC-42000, EN AC-435000
    \item Uses: EV motor housing, chassis parts
\end{itemize}

\subsection{Eutectic cast alloys}
\begin{itemize}
    \item Lowest melting point: excellent castability, no freezing range
    \item Na/Sr modify and refine eutectic; Ti/TiB$_2$ suppress Si needles
    \item Typical alloys: EN AC-44100, EN AC-44300, EC AC-47000
    \item Uses: ribbed casting, pump/engine housing, cylinder heads
\end{itemize}

\subsection{Hypereutectic cast alloys}
\begin{itemize}
    \item Primary Si crystal: very high wear resistance and reduced thermal expansion
    \item Uses: pistons (eg. EN AC-48000 (T6))
\end{itemize}

\newcolumn
\section{Surface technology - corrosion protection}
\subsection{Natural oxide layer on aluminum}
\begin{itemize}
    \item Aluminum forms an immediate, compact Al$_2$O$_3$ oxide layer (passivation)
    \item Properties: hard (ceramic), corrosion-resistant, stable at pH 5-8, only a few nm thick
    \item Thickening: chemical (chromating/chromitizing, boiling water) or electrochemical (anodizing)
\end{itemize}

\subsection{Chemical conversion coatings}
\begin{itemize}
    \item Remove the natural oxide by pickling/etching, then convert the surface into a thin conversion layer (oxide/chromate/chromite/phosphate)
    \item Coating is very thin, may show microcracks
    \item CCC (chromate, Cr VI): transparent/yellow/blue; good paint/primer adhesion; alternatives use Cr III (chromitizing) or chromous-acid systems
    \item Phosphating: phosphoric-acid based; adhesion primer for paint; common in the food industry
\end{itemize}

\subsection{Electrochemical anodizing}
\begin{itemize}
    \item GS (sulfuric gas): oxide up to 30$\mu$m; mainly decodative
    \item GSX (sulfuric + oxalic): 80-150$\mu$m; very good wear protection
    \item CAA (chromic acid): thin. flexible, low-crack layer; very high corrosion resistance; mainly aerospace (Cr(VI) hazard)
    \item TSA (tartaric-sulfuric): safer, more environmentally friendly
\end{itemize}

\part{Non-ferrous metals and applications}
\section{Titanium and Titanium alloys}
\subsection{Key properties of pure Ti}
\begin{itemize}
    \item $E \approx 110$ GPa (low compared to steel)
    \item Very good corrosion resistance, very good biocompatibility
    \item Interstitial (O, Fe) high strength but low ductility
\end{itemize}

\subsection{Crystal structure/polymorphism}
\begin{itemize}
    \item $\alpha$-Ti ar T$_\text{R}$; $\beta$-Ti above 882\cel
    \item Hot-forming easier in $\beta$; cold-forming limited in $\alpha$
    \item Service temperature limited by phase changes
\end{itemize}

\subsection{Alloy types}
\begin{itemize}
    \item c.p. Ti: best corrosion + weldability, lower strength
    \item $\alpha$ alloys: solid-solution strengthened, not precip.-hardenable
    \item $\alpha+\beta$ alloys: best all-round, heat-treatable
    \item $\beta$ alloys: highest strength, usually lower corrosion resistance
\end{itemize}

\subsection{Application}
\begin{itemize}
    \item Aerospace/structures and landing-gear parts
    \item Chemical industry: heat exchangers, pipes, pumps/valves
    \item Medical: implants and instruments
\end{itemize}

\section{Magnesium and Magnesium alloys}
\subsection{Key properties of pure Mg}
\begin{itemize}
    \item Very low density
    \item HCP structure; very low stiffness. $E\approx 44$ GPa
    \item Very un-noble, corrosion sensitive
\end{itemize}

\subsection{Main limitations}
\begin{itemize}
    \item Low strength; low fatigue/wear/creep resistance
    \item HCP: few slip systems $\rightarrow$\ limited cold-formability
    \item Machining risk: chips/dust can ignite
\end{itemize}

\subsection{Alloy types}
\begin{itemize}
    \item Mg-Al-Zn: common low-cost cast alloys
    \item AMZ40: 4\% Al + minor Mn/Zn (die-cast parts)
    \item Rare-earth cast alloys: improved high-T/creep performance
\end{itemize}

\subsection{Application}
\begin{itemize}
    \item Automotive die-cast parts: brackets, steering
    \item Lightweight housing: laptops, cameras
    \item Aerospace casting: very light gearbox housings
\end{itemize}

\section{Nickel and Nickel alloys}
\subsection{Key properties of pure Ni}
\begin{itemize}
    \item FCC crystal structure: very ductile even at low temps
    \item Self-passivating: very good corrosion resistance
    \item High melting point: good high-temps capability
    \item If precipitation-hardened: higher strength, creep resistance
\end{itemize}

\subsection{Alloy families}
\begin{itemize}
    \item Ni-Cu: solid-solution strengthened; very corrosion resistant
    \item Ni-base superalloys: high-temp + oxidation/corrosion
    \item Soft-magnetic Ni: very high permeability, magnetic losses
    \item NiTi: superelasticity + shape-memory effect
\end{itemize}

\subsection{Applications}
\begin{itemize}
    \item Corrosive environment: seawater, chemical plants
    \item High-temp parts: gas turbines, furnaces
    \item Electrical / magnetic: coil cores, sensors
    \item Medical: stents, orthodontic wires
\end{itemize}

\section{Copper and Copper alloys}
\subsection{Key properties of pure Cu}
\begin{itemize}
    \item Very high electrical and thermal conductivity
    \item Medium strength, high ductility
    \item FCC: very well cold-formable (malleable/ductile)
    \item Good atmospheric corrosion resistance
\end{itemize}

\subsection{Refining / purity}
\begin{itemize}
    \item Conductivity requires very high  purity
    \item Industrial route: fire refining then electrolytic refining
    \item Cu dissolves at anode and re-deposits at cathode
\end{itemize}

\subsection{Alloying and applications}
\begin{itemize}
    \item Alloying: $\uparrow$ strength but $\downarrow$ el + therm conductivity
    \item Pure Cu: cables, wires, busbars
    \item Low-alloy Cu: overhead contact lines, spring contacts
    \item Brass (Cu-Zn): good formability + machinability
    \item Bronze (Cu-Sn): springs, wear-resistant parts
\end{itemize}

\section{Zinc and Tin}
\subsection{Key properties}
\begin{itemize}
    \item Zn (pure): sacrificial corrosion protection for steel; cheap, easy to cast
    \item Sn: corrosion-resistand and food-safe
\end{itemize}

\subsection{Crystal structure / formability}
\begin{itemize}
    \item Zn: HCP; at TR already hot-forming, deformation
    \item Sn: low melting point: hot-formable already near TR
\end{itemize}

\subsection{Applications}
\begin{itemize}
    \item Zn: galvanizing, die casting, alloying element for brass
    \item Sn: tin plating for food and beverage, electronics solders 
\end{itemize}

\section{Refractory metals (W, Mo, Ta, Mb)}
\subsection{Key properties}
\begin{itemize}
    \item Very high melting temp: high-T capability, bus expensive
    \item Very good thermal and chemical/corrosion resistance
    \item High stiffness; generally low thermal expansion
\end{itemize}

\subsection{Crystal structure and deformation behavior}
\begin{itemize}
    \item BCC lattice: brittle-ductile transition temperature
    \item W, Mo: often brittle at TR; become ductile at higher T
    \item Ta, Nb: already ductile at TR
\end{itemize}

\subsection{Strengthening concept}
\begin{itemize}
    \item Strength at high T increases mainly by SSH
    \item Grain boundaries can limit creep/ductility at high T
\end{itemize}

\subsection{Applications}
\begin{itemize}
    \item W: TIG welding electrodes, high-T components
    \item Mo: high-T parts, air-lubricated bearing components
    \item Ta: chemical/biomedical components, Ta capacitors
    \item Nb: high-T/corrosion-resistant components, aerospace
\end{itemize}

\section{Precious metals (Pt, Pd, Au, Ag, Ir, Rh)}
\subsection{Key properties}
\begin{itemize}
    \item Very noble: high corrosion/oxidation resistance; stable in air
    \item High value: often used as thin layers
    \item Often excellent catalysts
\end{itemize}

\subsection{Jewelry alloys}
\begin{itemize}
    \item Platinum: typical fineness 950, 900, 800
    \item Gold: common fineness 999, 750, 585, 375
    \item Silver: sterling 925
\end{itemize}

\subsection{Applications}
\begin{itemize}
    \item Pt/Pd/Rh: Catalysts, chemical process
    \item Au: conductors, contact layers, wire bonding
    \item Rh: watch and jewelry industry
    \item Au/Ag/Pt/Pd: finance and investment speculation
\end{itemize}
\vfill\phantom{}

\newcolumn
\vspace*{-.5cm}
\part{Corrosion and corrosion prevention}
\section{Corrosion}
\subsection{Electrochemical corrosion of metals}
\begin{itemize}
    \item Indirect redox: oxidation at the anode + reduction at the cathode, coupled by an electron current
    \item Requirements: less noble anode, more noble cathode, electrical connection, electrolyte
\end{itemize}

\subsection{Galvanic cells}
\begin{itemize}
    \item Two galvanic half-cells form a galvanic element; corrosion is driven by potential diff. and/or ion concentration diff.
    \item Electrode potential at one interface is not measured directly; measure voltage between two half-cells
    \item Standard hydrogen electrode: 1 atm, pH = 0, 25\cel, E$^\circ$ = 0 V
\end{itemize}

\subsubsection{Chemical reactions}
\textbf{Anode/Oxidation}: Zn $\to$\ Zn$^{2+}$ + 2e$^-$\\
\textbf{Cathode/Reduction}: Cu$^{2+}$ + 2e$^- \to$\ Cu\\ 
\underline{Note}: noble metals are never the cathodes!
\subsection{Oxygen and hydrogen corrosion}
\begin{itemize}
    \item Anode: usually the least noble metal/phase in the system
    \item Cathode at neutral/basic pH: oxygen electrode
    \item Cathode at pH < 5 or with chlorides: hydrogen electrode
\end{itemize}

\subsubsection{Oxygen corrosion}
\textbf{Anode}: Fe $\to$\ Fe$^{2+}$ + 2e$^- \quad ; \quad$2Fe $\to$\ 2Fe$^{2+}$ + 4e$^-$\\
\textbf{Cathode}: {\footnotesize H$_2$O + $\frac{1}{2}$O$_2$ + 2e$^- \to$\ 2OH$^-\ ;\ $2H$_2$O + O$_2$ + 4e$^- \to$\ 4OH$^-$} 

\subsubsection{Hydrogen corrosion}
\textbf{Anode}: Fe $\to$\ Fe$^{2+}$ + 2e$^-$\\
\textbf{Cathode}: 2H$^+$ + 2e$^- \to$\ H$_2$

\subsection{Passivation}
\begin{itemize}
    \item Some metals form a dense oxide layer that slows corrosion
    \item Stainless steel: passivation needs about $\geq$ 10.5-12\% Cr; layer is very thin and self-renewing
    \item Al/Ti/Zn oxides are more insulating; passivation can increase risk of localized attack (pitting)
\end{itemize}

\subsection{Area rule}
\begin{itemize}
    \item Small anode + large cathode $\to$\ high corrosion rate
    \item Large anode + small cathode $\to$\ low corrosion rate
\end{itemize}

\section{Types of corrosion}
\subsection{Surface corrosion}
\begin{itemize}
    \item Relatively uniform material loss over large areas
    \item Typical for steels in atmospheric exposure
    \item Main risk: loss of cross-section: reduced load capacity
\end{itemize}

\subsection{Contact corrosion / Selective corrosion}
\begin{itemize}
    \item Contact corrosion: dissimilar metals connected + electrolyte $\to$ less noble dissolves; noble surface acts cathodically
    \item Severity strongly depends on area ratio
    \item {\footnotesize Selective: in multiphase alloy the less noble phase corrodes}
\end{itemize}

\subsection{Pitting corrosion (PREN)}
\begin{itemize}
    \item Local breakdown of passivation: tiny anode + no repassivation inside the pit
    \item Chlorides concentrate: self-accelerating, acidifies in the pit
    \item Higher PREN, better pitting resistance:
    \begin{itemize}
        \item PREN > 32: seawater resistant
        \item PREN > 17: often acceptable for medtech/watches
    \end{itemize}
    \item Mo increases corrosion resistance and sweat resistance
\end{itemize}

\subsection{Intercrystalline corrosion (IC)}
\begin{itemize}
    \item Sensitization in stainless steels: Cr carbides at grain boundaries $\to$\ no local repassivation
    \item Prevention: low C grades (<0.03\%), Ti/Nb stabilized grades, or solution anneal + rapid cooling
    \item Strength $\uparrow$: C$\downarrow$, Ti$\uparrow$, Nb$\uparrow$, Ta$\uparrow$, TiC$\downarrow$
\end{itemize}

\subsection{Crevice corrosion}
\begin{itemize}
    \item Like pitting, but driven by oxygen depletion in a crevice
    \item Non-aerated zone becomes anodic; chloride enrichment is typical; avoid/seal crevices
\end{itemize}

\subsection{Stress corrosion cracking (SCC)}
\begin{itemize}
    \item {\footnotesize 3 factors: susceptible material; specific medium; tensile stress}
    \item Countermeasures: reduce stress, change environment, change materials, coatings/cathodic protection
\end{itemize}

\section{Corrosion prevention}
\subsection{Active}
\begin{itemize}
    \item Sacrifical anode/coating with a less noble metal (Zn,Al,Mg) to protect steel 
    \item Impressed current cathodic protection (external DC source), requires an electrolyte path
\end{itemize}

\subsection{Passive}
\begin{itemize}
    \item Barrier layers: paints, varnish, plastics, oils; oxides, phosphates, enamel; noble metal coatings (Ni,Au)
    \item Design/material measures: avoid crevices/water traps/sharp edges, electrically insulate dissimilar metals, improve surface quality
\end{itemize}

\part{Ceramic and glasses}
\section{Ceramics}
\subsection{Properties and applications of ceramics}
\textbf{Pros}: Heat resistance, hard/wear resistance, high compression
strength, high Young's Modulus $E$, low thermal expansion coefficient,
foot safe, insulator (most ceramics)\\
\textbf{Cons}: Brittle, crack formations, not tensile, cost\\
\textbf{Appl.}: Bearings, construction materials, coatings, medtech

\subsection{Bonding and structure}
\begin{itemize}
    \item Mostly ionic/covalent bonding (eg. oxides, nitrides, carbides, borides)
    \item \underline{Ceramics have crystalline structure}
\end{itemize}

\subsection{Processing route}
Powder production $\to$\ shaping (pressing/extrusion/injection)
$\to$\ debinding $\to$\ sintering (shrinkage + densification) $\to$\ 
finishing (grinding/polishing).

\subsection{Brittle fracture}
\begin{itemize}
    \item Strength controlled by defects (pores/cracks): ``weakest link'' behavior, crack initiation often at surface/defects
    \item Size effect: larger volume $\to$\ higher defect probability $\to$\ lower fracture strength
\end{itemize}

\subsection{Weibull distribution}
\begin{itemize}
    \item As for brittle fracture, size effect also for Weibull
    \item Higher Weibull modulus $m$ $\to$\ less scatter (best quality)
    \item Depends on the material composition (purity), grain size, porosity, volume, and manufacturing process
    \item Large $m \to$\ narrow fracture strength distribution $\to$\ good microstructure quality
\end{itemize}

\subsection{Fracture toughness K$_\text{Ic}$}
\[\text{K}_\text{Ic} = f \sigma_c \sqrt{\pi a}\]
$a$ = crack size, $\sigma_c$ = fracture stress, $f$ = geometry factor
\begin{itemize}
    \item K$_\text{Ic}$: critical mode-I stress intensity (crack becomes unstable $\to$\ brittle fracture)
    \item Higher K$_\text{Ic}$: more crack tolerance
    \item Ceramics: measured often in notched bendings
\end{itemize}

\subsection{Ceramic classes}
\subsubsection{Classic ceramics (silicate or natural)}
\begin{itemize}
    \item Raw materials: kaolin + quartz + feldspar
    \item Cordierite: low coefficient of thermal expansion and good thermal shock resistance
    \item Stearite: ten times lower dielectric loss factor tan $\delta$ compared to electroporcelain
    \item Products: porcelain, stoneware, sanitary
    \item Glazes reduce open porosity
    \item Applications: sanitary ceramics, dishes, construction
\end{itemize}

\subsubsection{Refractory ceramics}
\begin{itemize}
    \item High-T linings for furnaces/steel
    \item Chamotte: for heat treatment ovens
    \item Magnesia: for fireproof refractory stones in steel ladles
    \item May show softening interval due to mixed phases
\end{itemize}

\subsubsection{High-performance (technical) ceramics}
\begin{itemize}
    \item Chemically processed raw materials
    \item Additives (binders) as raw materials are not moldable
    \item Oxide cemarics, non-oxide ceramics
    \item Applications: technical products (high-tech) 
\end{itemize}

\subsection{Structural high-performance ceramics}
\subsubsection{Alumina (Al$_2$O$_3$)}
\begin{itemize}
    \item Increase of $E$ and strength with degree of purity
    \item Hard, wear resistant, good electrical insulator, corrosion resistant
    \item Applications: insulators, melting pot, abrasive powder
    \item Single crystalline (sapphine): transparent, scratch-resistant, thermoshock resistant. For watch glasses
\end{itemize}

\subsubsection{Zirconia (ZrO$_2$, stabilized)}
\begin{itemize}
    \item Higher fracture toughness K$_\text{Ic}$; increases in partially stabilized
    \item Polymorphism: stabilized (Y$_2$O$_3$/MgO/CaO) controls phases
    \item Can conduct oxygen ions (sensor applications)
    \item Applications: forming tools, knives, implants
\end{itemize}

\subsubsection{Silicon carbide (SiC)}
\begin{itemize}
    \item Very hard, high thermal conductivity, low thermal expansion
    \item High temperature and wear resistance
    \item Applications: bearings, fireproof ceramics, heating elements
\end{itemize}

\subsubsection{Silicon nitride (Si$_3$N$_4$)}
\begin{itemize}
    \item Improved toughness via microstructure (crack deflection)
    \item Good wear and temeprature resistance
    \item Applications: pipes, molds, wire production
\end{itemize}

\subsection{Functional ceramics}
\subsubsection{Piezoelectric ceramics}
\begin{itemize}
    \item Require non-centrosymmetric structure
    \item Piezo effect disappears above Curie temperature
    \item Applications: sensors/actuators
\end{itemize}

\subsubsection{Ferrites}
Soft magnetic ceramics (eg. induction applications)

\subsubsection{Superconducting ceramics}
\begin{itemize}
    \item High critical temp relative to metallic superconductors\\
        (eg. Al, Pb, Nb$_3$Sn, NbTi, MgB$_2$)
    \item Enables liquid nitrogen cooling (77K) for strong-field appl.
    \item Uses: magnetic resonance imaging, strong magnetic fields
\end{itemize}

\subsubsection{Optical ceramics}
Single crystal ceramics. For short pulse lasers.

\section{Glass and Glass-ceramic}
\subsection{Glass-ceramics}
\begin{itemize}
    \item Cast like glass, then controlled crystallization $\to$\ very low thermal expansion and high thermal-shock resistance
    \item Applications: cooking hobs, precision mirror materials
\end{itemize}

\subsection{Glass basics}
Network concept: network formers (SiO$_2$, B$_2$O$_3$) + 
modifiers (Na$_2$O, K$_2$O) that reduce viscosity and the glass
transition temperature.\\
\underline{Glasses have amorphous structure (transparents)}.

\subsubsection{Float glass surface}
\begin{itemize}
    \item Lower surface: contacts liquid tin bath $\to$\ very smooth
    \item Upper surface: flame-polished in the float proces $\to$\ very smooth
\end{itemize}

\subsubsection{Production of glass}
\begin{itemize}
    \item Hollow glass: glass blowing
    \item Flat glass: float glass process, drawing, pressing
\end{itemize}

\subsection{Main glass types}
\subsubsection{Quartz glass}
\begin{itemize}
    \item High melting point, high viscosity, high temperature resistance, low thermal expansion, chemical resistance
    \item Applications: laser optics, lamps, fireproof stove windows
\end{itemize}

\subsection{Other glasses}
\subsubsection{Soda-lime glass}
Cheap, easy to form, limited thermal-shock resistance. Used for 
windows and bottles.

\subsubsection{Borosilicate glass}
Low thermal expansion gives high thermal-shock and chemical resistance.
Used for labware and cookware.

\subsubsection{Lead oxide containing glass}
High refractive index and density (good optics/radiation shielding)
but heavier and more toxic/regulated.

\subsection{Glass strengthening}
\subsubsection{Thermal toughening}
Rapid cooling creates compressive surface stresses.

\subsubsection{Chemical toughening}
Ion exchange (Na$^+ \to$\ K$^+$) creates compressive surface
layer (display glasses). As a result, compressive stress increases

\part{Polymers, Recycling, and FRP}









































































































































\end{multicols*}

\end{document}