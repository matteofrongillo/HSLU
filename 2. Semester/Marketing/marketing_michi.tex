\documentclass[12pt]{article}
\usepackage[a4paper,margin=2.5cm]{geometry}
\usepackage{enumitem}
\usepackage{titlesec}
\usepackage{amsmath}
\usepackage{graphicx}
\usepackage{fontawesome5}
\usepackage{lmodern}
\usepackage{hyperref}
\hypersetup{
    colorlinks=true,
    linkcolor=black,
    urlcolor=blue
}
\titleformat{\section}{\normalfont\Large\bfseries}{\thesection}{1em}{}

\begin{document}

\begin{center}
    \LARGE\textbf{Marketing Cheat Sheet: Weeks 1–7}
\end{center}
\hrule
\vspace{1em}

\section*{\faChartBar\hspace{0.5em} Week 1–2: Marketing Foundations \& Relationship Marketing}

\textbf{Definition of Marketing}:
\begin{itemize}[leftmargin=*]
    \item The process of creating, communicating, delivering, and exchanging offerings that have value for customers, clients, partners, and society.
\end{itemize}

\textbf{Transactional vs. Relationship Marketing}:
\begin{itemize}[leftmargin=*]
    \item \textbf{Transactional}: Focus on one--time sales, product--centric, short--term.
    \item \textbf{Relationship}: Focus on long--term engagement, customer--centric, trust--based.
\end{itemize}

\textbf{Drivers of Relationships}:
\begin{itemize}[leftmargin=*]
    \item Trust, Commitment, Satisfaction, Gratitude
\end{itemize}

\textbf{Relationship Ladder}:
\begin{enumerate}[leftmargin=*]
    \item Awareness
    \item Exploration
    \item Expansion
    \item Commitment
    \item Advocacy
\end{enumerate}

\textbf{Customer Value}:
\begin{itemize}[leftmargin=*]
    \item Perceived benefits vs. perceived costs.
    \item Value = Quality + Service + Image − Price
\end{itemize}

\section*{\faSearch\hspace{0.5em} Week 3: Situational Analysis}

\textbf{SWOT Analysis}:
\begin{itemize}[leftmargin=*]
    \item \textbf{Strengths}: Internal advantages
    \item \textbf{Weaknesses}: Internal limitations
    \item \textbf{Opportunities}: External trends to leverage
    \item \textbf{Threats}: External challenges
\end{itemize}

\textbf{PESTEL Framework}:
\begin{itemize}[leftmargin=*]
    \item \textbf{Political}: Laws, regulations, trade policies
    \item \textbf{Economic}: Inflation, interest rates, economic growth
    \item \textbf{Social}: Demographics, culture, lifestyle trends
    \item \textbf{Technological}: Innovation, R\&D, tech infrastructure
    \item \textbf{Environmental}: Climate, sustainability issues
    \item \textbf{Legal}: Labor laws, consumer protection
\end{itemize}

\textbf{Porter’s Five Forces}:
\begin{enumerate}[leftmargin=*]
    \item Threat of New Entrants
    \item Bargaining Power of Suppliers
    \item Bargaining Power of Buyers
    \item Threat of Substitutes
    \item Competitive Rivalry
\end{enumerate}

\section*{Week 4–5: Strategic Planning \& STP}

\textbf{Strategic Planning Process}:
\begin{enumerate}[leftmargin=*]
    \item Define Mission and Vision
    \item Conduct SWOT Analysis
    \item Set Objectives (SMART)
    \item STP: Segmentation, Targeting, Positioning
    \item Strategy Implementation
    \item Control and Evaluation
\end{enumerate}

\textbf{Market--Driven vs. Resource--Driven Strategy}:
\begin{itemize}[leftmargin=*]
    \item \textbf{Market--driven}: Responsive to market needs and trends.
    \item \textbf{Resource--driven}: Based on internal strengths and assets.
\end{itemize}

\textbf{Segmentation Criteria}:
\begin{itemize}[leftmargin=*]
    \item Geographic, Demographic, Psychographic, Behavioral
\end{itemize}

\textbf{Targeting Strategies}:
\begin{itemize}[leftmargin=*]
    \item Undifferentiated, Differentiated, Concentrated, Micromarketing
\end{itemize}

\textbf{Positioning}:
\begin{itemize}[leftmargin=*]
    \item How a brand is perceived in the minds of customers relative to competitors.
    \item Perceptual mapping helps visualize positioning.
\end{itemize}

\section*{\faLeaf\hspace{0.5em} Week 6–7: The Marketing Mix – Product Focus}

\textbf{4Ps $\rightarrow$ 4Cs}:
\begin{itemize}[leftmargin=*]
    \item \textbf{Product $\rightarrow$ Customer solution}
    \item \textbf{Price $\rightarrow$ Customer cost}
    \item \textbf{Place $\rightarrow$ Convenience}
    \item \textbf{Promotion $\rightarrow$ Communication}
\end{itemize}

\textbf{Product Levels}:
\begin{enumerate}[leftmargin=*]
    \item Core Product (basic need satisfied)
    \item Actual Product (design, features, brand)
    \item Augmented Product (services, warranty, support)
\end{enumerate}

\textbf{Product Lifecycle}:
\begin{enumerate}[leftmargin=*]
    \item Introduction
    \item Growth
    \item Maturity
    \item Decline
\end{enumerate}

\textbf{B2C vs B2B Buying Behavior}:
\begin{itemize}[leftmargin=*]
    \item \textbf{B2C}: Emotional, influenced by brand/lifestyle, short buying cycle
    \item \textbf{B2B}: Rational, formalized process, long buying cycle, derived demand
\end{itemize}

\textbf{Buyer Roles in B2B}:
\begin{itemize}[leftmargin=*]
    \item Initiator, Influencer, Decider, Buyer, User
\end{itemize}

\section*{\faGraduationCap\hspace{0.5em} Exam Tips}

\begin{itemize}[leftmargin=*]
    \item Always apply theory to a real--world example.
    \item Know how to draw and label a perceptual map.
    \item Be prepared to conduct a SWOT or PESTEL analysis in short form.
    \item Understand how marketing strategy links to the value proposition.
    \item Remember: Marketing is about delivering value \textbf{better than the competition}.
\end{itemize}

\end{document}
