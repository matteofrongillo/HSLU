\documentclass{article}

\usepackage[fleqn]{amsmath}
\usepackage{amssymb}
\usepackage{hyperref}
\usepackage{url}
\usepackage{graphicx}
\usepackage{geometry}
\usepackage{babel}
\usepackage{enumitem}
\usepackage{parskip}
\usepackage{chemfig}
\usepackage{pdfpages}
\usepackage{xcolor}
\usepackage{tikz}
\usepackage{fancybox}
\usepackage{makecell}
\usepackage{pgfplots}
\usepackage{soul}
\usepackage{ulem}
\usepackage{wrapfig}
\usepackage{subcaption}
\usepackage[T1]{fontenc}
\usepackage{pgfplots}
\usepackage{esvect}
\usetikzlibrary{arrows}
\usetikzlibrary{decorations.pathreplacing}
\pgfplotsset{compat=1.17}

\geometry{
    a4paper,
    total={170mm, 257mm},
    left=20mm,
    top=20mm
}

\hypersetup{
    colorlinks=true,
    linkcolor=black,
    urlcolor=blue,
    pdftitle={Math refresher course}
}

\newcommand{\figbox}[1]{ 
    \begin{figure*}[ht!]        
        \begin{center}            
            \fbox{#1}        
        \end{center}    
    \end{figure*}
}

\newcommand{\wrapfill}{
    \par
    \ifnum \value{WF@wrappedlines} > 0
        \addtocounter{WF@wrappedlines}{-1}%
        \null\vspace{
            \arabic{WF@wrappedlines}
            \baselineskip
        }
        \WFclear
    \fi
    \phantom{}
}

\newcommand{\difference}{\,\backslash\,}
\newcommand{\rem}{\underline{Remark}: }
\newcommand{\nots}{\underline{Notation}: }
\newcommand{\prf}{\underline{Proof}: }

% === TEXT ===
\title{\textbf{Maths refresher course \\ HSLU, Semester 1}}
\author{Matteo Frongillo}

\begin{document}

\maketitle
\tableofcontents
\pagebreak

\part{Week 1}
\section{The set theory}
\subsection{Definition of a set}
A set is a collection of objects or elements.

\rem{The collection of all sets is not a set.}

\subsection{Logical symbols}
\subsubsection{Definition}
Braces and the definition symbol ``$:=$'' are used to define a set giving all its elements:
\figbox{$A:=\left\{a,b,c,d,e\right\}$}

\subsubsection{Equal}
In this case, the equal symbol means that the set $A$ is equal to the set $B$:
\figbox{$A=B$}

\subsubsection{Belongs to}
The symbols $\in$ and $\ni$ describe an element which is part of the set:
\figbox{$a \in A \Longleftrightarrow A \ni a$}

\subsubsection{Does not belong to}
The symbols $\notin$ mean that an element does not belong to the set:
\figbox{$f \notin A$}

\subsubsection{Inclusion and contains}
The symbols $\subset$ and $\supset$ mean that a set has another set included in its set:
\figbox{$\mathbb{N} \subset \mathbb{Z} \Longleftrightarrow \mathbb{Z} \supset \mathbb{N}$}

\subsubsection{For all/any}
The symbol $\forall$ means that we are considering any type of element:
\figbox{$\forall x \in \mathbb{R},\ x>0$}

In this case, we've defined a new set.

\newpage
\subsubsection{Implication}
The symbol $\Rightarrow$ means that by setting a rule, we imply an event or an action:
\figbox{if $x=1 \Longrightarrow x \in \mathbb{N}$, but if $x \in \mathbb{N}$ we do not know if $x=1$}

The symbol $\Leftarrow$ is called inference (it is inferred).

\subsubsection{If and only if}
The symbol $\Leftrightarrow$ means that two events happen simultaneously (double implication):
\figbox{$x \in \mathbb{N},\ x \neq 0 \Longleftrightarrow x \in \mathbb{N}^*$}

\prf{Let’s prove that $2x=10 \Longleftrightarrow x=5$}
\begin{center}
    \begin{align*}
        2x&=10 \Rightarrow x=5\\
        \frac{2x}{2}&=\frac{10}{2}\\
        x&=5
    \end{align*}
\end{center}

\newpage
\subsection{Numerical sets}
\begin{itemize}
    \item $\mathbb{N} :=$ Natural numbers (including 0);
    \item $\mathbb{Z} :=$ Integer numbers;
    \item $\mathbb{Q} :=$ Rational numbers;
    \item $\mathbb{R} := \text{Real numbers} := \mathbb{Q} \cup \left\{\text{irrational numbers}\right\}.$
\end{itemize}

\nots{The ``$^*$'' symbol means that the set does not include 0.}

\subsubsection{Inclusion of sets}
\figbox{$\mathbb{N} \subset  \mathbb{Z} \subset \mathbb{Q} \subset \mathbb{R} \subset \mathbb{C}$}

$
    B := \left\{\pi, 1,-1,0\right\}; \\
    C := \left\{\pi, 1\right\}; \\
    D := \left\{\pi\right\}.
$

Then we write some examples: $\pi \in B,\ D \subset B,\ C \subset B,\ B \not\subset C,\ 0 \in B,\ 0 \notin C$.

\section{The real line}
\vspace*{.5cm}
\begin{center}
    \begin{tikzpicture}
        \draw[->] (0,0) -- (6,0);
        \draw[<-] (-6,0) -- (0,0);
        \node at (0,.5) {$\mathbb{R}$};
        
        \foreach \x/\label in {-5/{$-\pi$}, -4/{$-e$}, -3/{$-\sqrt{2}$}, -2/{$-1$}, -1/{$-\frac{1}{2}$}, 0/{$0$}, 1/{$\frac{1}{2}$}, 2/{$1$}, 3/{$\sqrt{2}$}, 4/{$e$}, 5/{$\pi$}} {
            \draw (\x,0.1) -- (\x,-0.1) node[below] {\label};
        }
        \node[below] at (-6,-0.1) {$-\infty$};
        \node[below] at (6,-0.1) {$+\infty$};
    \end{tikzpicture}
\end{center}

\end{document}
                                                                                